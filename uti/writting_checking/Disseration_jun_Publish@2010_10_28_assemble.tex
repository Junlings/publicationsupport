

%  ========================================================================
%  Copyright (c) 2006-2008 The University of Washington
%
%  Licensed under the Apache License, Version 2.0 (the "License");
%  you may not use this file except in compliance with the License.
%  You may obtain a copy of the License at
%
%      http://www.apache.org/licenses/LICENSE-2.0
%
%  Unless required by applicable law or agreed to in writing, software
%  distributed under the License is distributed on an "AS IS" BASIS,
%  WITHOUT WARRANTIES OR CONDITIONS OF ANY KIND, either express or implied.
%  See the License for the specific language governing permissions and
%  limitations under the License.
%  ========================================================================
%

% Documentation for UW thesis document style for LaTeX

% by Jim Fox
% fox@washington.edu
%
%    Revised for version 2008/04/15 of uwthesis.cls
%    To help you ignore the unusual things I do with this sample document
%    I try to use the notation
%
%    % --- sample stuff only -----
%    special stuff for my document, but you don't need it in your thesis
%    % --- end-of-sample-stuff ---

%    Printed in twoside style now that that's allowed
%

\documentclass [11pt,oneside] {ucfthesis}
\usepackage{chapterbib}

\usepackage{graphicx}        %% use graphy

\usepackage[dvips,bookmarks=true,citebordercolor={1 1 1},filebordercolor={1 1 1},linkbordercolor={1 1 1},bookmarksnumbered=true]{hyperref}  %% provide the pdf bookmark when converting

\usepackage{amssymb,amsmath}

\setcounter{tocdepth}{2}  % Print the chapter and sections to the toc


\begin{document}

% ==========   Preliminary pages
%
\prelimpages

%
% ----- title page
%
\Title{ULTRA-HIGH PERFORMANCE FIBER REINFORCED CONCRETE IN BRIDGE DECK APPLICATIONS}
\Author{JUN XIA}
\PreviousDegrees{B.S. Zhejiang University, 2003 \linebreak
M.S. Zhejiang University, 2006}
\Year{2010}
\Degree{Doctor of Philosophy}
\Department{Civil, Environmental, and Construction Engineering}
\College{Engineering and Computer Science}
\Term{Fall Term}
\Advisor{Kevin R. Mackie}
\titlepage
\copyrightpage

% --- end-of-sample-stuff ---

%
% ----- signature page (put real names in these)
%

\Chair{Dr Kevin R. Mackie}{Assistant Professor}{Department of Civil and Environmental Engineering}

\Signature{Dr. F. Necati Catbas}
\Signature{Dr. Shiou-San Kuo}
\Signature{Dr. Bruce M. Butler}
\Signature{Dr. Hae-Bum Yun}
%\signaturepage


%
% ----- abstract
%

\currentpdfbookmark{ABSTRACT}{abstract}
%\setcounter{page}{-1}
\abstract{The research presented in this dissertation focuses on the material characterization of ultra-high performance fiber reinforced concrete (UHP-FRC) at both the microscopic and macroscopic scales. The macroscopic mechanical properties of this material are highly related to the orientation of the steel fibers distributed within the matrix. However, the fiber orientation distribution has been confirmed to be anisotropic based on the flow-casting process. The orientation factor and probability density function (PDF) of the crossing fiber (fibers crossing a cutting plane) orientation was obtained based on theoretical derivations and numerical simulations with respect to different levels of anisotropy and cut planes oriented arbitrarily in space. The level of anisotropy can be calibrated based on image analysis on cut sections from hardened UHP-FRC prisms. Simplified equations provide a framework to predict the mechanical properties based on a single fiber-matrix interaction rule selected from existing theoretical models. Along with the investigation of the impacts from different curing methods and available post-cracking models, a versatile parameterized uniaxial stress-strain constitutive model was developed and calibrated.

The constitutive model was implemented in a finite element analysis software program, and the program was utilized in the preliminary design of moveable bridge deck panels made of passively reinforced UHP-FRC. This deck system was among the several alternatives to replace the problematic steel grid decks currently in use. Based on experimental investigations of the deck panels, failure occurred largely in shear rather than flexure during bending tests. However, this shear failure is not abrupt and usually involves large deformation, large sectional rotation, and wide shear cracks before loss of load-carrying capacity. This particular shear failure mode observed was further investigated numerically and experimentally. Three-dimensional FEM models with the ability to reflect the interaction between rebar and concrete were created in a commercial FEM software to investigate the load transfer mechanism before and after bond failure. Small-scale passively reinforced prisms were tested to verify the conclusions drawn from simulation results. In an effort to improve the original design, several shear-strengthened deck panels were tested and evaluated for effectiveness. Finally, methods and equations to predict the ultimate shear capacity were calibrated.

A two-dimensional frame element based complete moveable bridge finite element model was built for observation of bridge system performance. The model contained the option to substitute any available deck system based on a subset of pre-calibrated parameters specific to each deck type. These alternative deck systems include an aluminum bridge deck system and a glass fiber reinforced plastic (GFRP) deck system. All three alternatives and the original steel grid deck system were evaluated based on the global responses of the moveable bridge, and the advantages and disadvantages of adopting the UHP-FRC deck system are quantified.
}




%
% ----- dedication
%
\dedication{
I dedicate this dissertation to my dear wife Chun He, my daughter Vivienne, and my parents. The work presented here can not be finished without the support and encouragement I received from my wife. Her hardworking and determination rolls the family forward. My parents always love me unconditionally and sacrificed themselves for my future goodness.
}

%
% ----- acknowledgments
%
\currentpdfbookmark{ACKNOWLEDGMENTS}{ACKNOWLEDGMENTS}
\acknowledgments{
The author would like to express sincere appreciation to all the people help make this dissertation possible. First I would like to thank my advisor, Dr Kevin R. Mackie, who inspired me to work on the research presented here and helped me at various stages of the writing. His knowledge and enthusiasm about our research field greatly encouraged me and build a role model for me to follow. I also would like to thank all my committee members, Dr. Necati Catbas, Dr. Shiou-San Kuo, Dr. Bruce M. Butler, and Dr. Hae-Bum Yun for providing me with valuable guidance on my research work. The author also thank our research partner at Florida International University, Dr. Amir Mirmiran and Muhammad Azhar Saleem, who worked closely with the author to make all these research work happen. Special thank to Dr Lei Zhao, who gave me the opportunity to start the study and research here at University of Central Florida. The thanks also go to all my friends and colleagues working in the structure's labs, Michael Olka, Zachary Haber, Jignesh Vyas, Ricardo Zaurin, Melih Susoy, Kevin O'Neill, Robert Slade, Elie El Zghayar, Yulin Xiao, Shanjun Helian and our technical support and lab manager Mr. Juan Cruz, who helped design the experiment setups and coordinate the lab equipments.
}

%
% ----- contents & etc.
%
\currentpdfbookmark{TABLE OF CONTENT}{TABLE OF CONTENT}
\tableofcontents



\listoffigures
\listoftables

%
% ----- abbreviations and acronyms (old glossary)
%
\chapter*{LIST OF ABBREVIATIONS}      % starred form omits the `chapter x'
\addcontentsline{toc}{chapter}{LIST OF ABBREVIATIONS}
\thispagestyle{plain}
%
\begin{glossary}
\item[UHP-FRC] Ultra-high performance fiber reinforced concrete
\end{glossary}

\begin{glossary}
\item[UHPC] Ultra-high performance concrete
\end{glossary}

\begin{glossary}
\item[RC] Reinforce Concrete
\end{glossary}

\begin{glossary}
\item[FRC] Fiber Reinforce Concrete
\end{glossary}


\begin{glossary}
\item[OpenSees] Open System for Earthquake Engineering Simulation
\end{glossary}

\begin{glossary}
\item[HPFRCC] High performance fiber reinforced cement composite
\end{glossary}

\begin{glossary}
\item[RPC] Reactive Powder Concrete
\end{glossary}

\begin{glossary}
\item[SNT] Statistical nanoindentation technique
\end{glossary}






%
% end of the preliminary pages



%
% ==========      Text pages
%

\textpages
% ========== check Chapter 1

\chapter{INTRODUCTION ON UHP-FRC}

    \section{DEFINITIONS}

        %\subsection{History and definitions of UHP-FRC}
            %% ====text====
% Last updated @ [2010/10/27] by Mackie
% Last updated @ [2010/10/20]
% Last updated @ [2010/10/5]
% Discuss about the definition of UHP-FRC
% this section is the general about FRC
Although the concept of fiber reinforced concrete appeared much earlier, the research on FRC did not take off until the 1960s when metal fibers became more commonly available commercially \cite{60004}. 

%The definition of fiber reinforced concrete from ACI Committee 544, 1996 
% shouldnt that be a proper citation?
%was `concrete made primarily with hydraulic cement, aggregates and discrete reinforcing fibers.' 

The original purpose of adding fibers to the weak and brittle cement matrix was to control the crack width. Later, researchers found that the additional fibers could increase the first crack strength as well. For post-crack strength, most of the early FRCs exhibited strain softening behavior, while strain hardening behavior during the uniaxial tension test or third point bending test was observed in several recently developed FRC. They have superior ductile behavior compared to plain concrete matrices or the FRCs with strain softening. A FRC which exhibits a strain hardening effect is classified as high performance fiber reinforced cement composites (HPFRCC) \cite{25808}. Depending on the different hardening behaviors, HPFRCC can be subdivided into two categories as tension strain hardening and deflection strain hardening. All tension hardening materials will exhibit deflection hardening effects. Although the ductilities of these HPFRCCs is improved greatly when compared to the normal FRC, the compressive strength of these material is usually at the same level as normal concrete.

% this section is the development of UHP-FRC
Although there are different definitions of ultra-high performance concrete, it is commonly characterized as exhibiting a compressive strength higher than 150 MPa and high ductility. The high compressive strength of the matrix is achieved by using only fine aggregates that ensure good homogeneity and compactness  \cite{22068}. The appropriate granular mixture also reduces the entrapped air and creates a rigid structural skeleton. The water-binder ratio is usually about 0.2 and the fiber volume faction is about 2\% for typical UHP-FRC mixtures. The short fibers are added into the cement matrix as micro-reinforcement and the high bond strength between them is ensured by the treatment of fibers and use of silica fumes. The good ductility of UHP-FRC comes from the bridging effects of fibers and leads to a strain hardening response with high pre- and post-crack tensile strength. The existence of fibers also changes the brittle compressive failure mode to a more ductile manner and prevents the sudden explosive compressive failure \cite{22006}.

Although there is still debate over the definition of UHP-FRC (UHPC), both the strength and ductility are treated as critical criteria. The design code made by French civil engineering association \cite{22005} defines ultra-high performance concrete (UHPC) as concrete requiring the compressive strength greater than 150 MPa (21.8 ksi), tensile strength over 7 MPa (1 ksi) and ductile behavior under mechanical load. 

    \section{INGREDIENT AND PRODUCTION OF UHP-FRC}

        \subsection{Ingredient and production}
            %% ====text====
% Last updated @ [2010/10/27] by Mackie
% Last updated @ [2010/10/20] by Jun
% Last updated @ [2010/10/5] by Jun
% Discuss about the definition of UHP-FRC
Several types of UHP-FRC are listed in Table~\ref{Table_available_UHPFRC}. However, only System 1 is commercially available as an off-the-shelf product and thus will be used in this research. Ductal$^{\small\textregistered}$-FM contains 2\% volume steel fibers and the constituent materials are listed in Table~\ref{Table_constitution}.

The typical Metallic fiber used with System 1 has a nominal diameter of 0.2 mm (0.008 in.) and a nominal length of 12.7 mm (0.5 in.). Its yielding stress is 3150 MPa and ultimate stress is 3250 MPa. The modulus of elasticity of the fiber is 210 GPa. The critical fiber volume fraction to achieve a homogeneous distribution is about 0.4\% to 0.5\%, according to Vodicka \cite{Vodicka}. Taerwe et al. \cite{Taerwe} also achieved the homogeneous distribution for up to 8\% fiber volume fraction when using short steel fibers. Therefore, 2\% of steel fiber volume fraction is a suitable amount to achieve a homogeneous distribution without compromising the workability of the cement paste.

The fiber pull out test results on high compressive strength mortar (150 MPa) show that the interfacial bond strength was greatly increased due to the high compressive strength \cite{23204}. The additional silica fume was found to be directly related to the interfacial bond strength of the high compressive strength reactive powder matrix. The interfacial shear stress with 30\% weight ratio of silica fume is about 5.5 MPa (0.8 ksi) \cite{23205}.

Due to the self-consolidating properties of UHP-FRC, the flow cast method is commonly used when casting UHP-FRC members. Usually, mild or even no external vibration is necessary. The good flow ability of the cement paste also ensures good quality members because the paste fills out the form space without entrapped air bubbles. For System 1, the pre-mix bags contain the ingredients in the first four rows in Table ~\ref{Table_constitution}. Steel fibers and super plasticizer ship separately. Ice cubes are added during the casting to control the mixing temperature and working properties and are necessary if the environmental temperature is higher than 25 degrees Celsius. Usually, the cement paste was dumped at one end of the form and allowed to spread to the remaining spaces. If needed, additional paste material can be added behind the flow front as shown in Fig.~\ref{fig_flowtable}. Multi-layer casting techniques were usually not used, although the cold joint is not a issue if the waiting time is less than an hour. The viscosity of the cement paste was tested via the flow table test following ASTM standard \cite{ASTM-C230} with the equipment shown in Fig.~\ref{fig_flowtable}. 

            %% ====table====



            %% ====figure====


        \subsection{Investigation on microstructures}
            %% ====text====
% Last updated @ [2010/10/27] by Mackie
% Last updated @ [2010/10/20] by Jun
% Last updated @ [2010/10/5] by Jun
% Discuss about the microscale research on UHP-FRC
Research on FRC can be divided based on the scale of interest as listed in Table~\ref{Table_research_scales}. Different targets and instruments were used for each level to investigate the material and structural properties. To investigate the responses of the material at the structural level, the basic material constitutive relations should be derived first. For UHP-FRC, which is a composite material in nature, both the properties of individual material and the interface need to be investigated.

At the molecular and micro levels, the constitution of different mechanical parts within the cement paste and the interface between these parts were investigated. For traditional FRC, the matrix of the composites can divided into three components \cite{60005} 
% I do not follow the 3 components, see my edit below?
as continuous phase cement matrix, discontinuous phase grains, pores. Fibers act as micro-reinforcement. Constantinides \cite{23101} found there are two different kinds of Calcium-Silicate-Hydrate (C-S-H) that exist in the cement-based material. The two types of C-S-H have different elastic properties and thus the properties of overall cement paste are related to the percentage of these two portions. For UHP-FRC, based on the research of Sorelli \cite{23103} using the statistical nanoindentation technique (SNT), it was concluded that there is 86\% C-S-H of the overall volume of UHP-FRC. This is much higher than normal concrete with water-cement ratio equal to 0.5. Sorelli also found that the interface of fiber and matrix is not softer than the matrix and has uniform composition of the hydration products around the fiber, which is different from other type of material classified as HPFRCC. This lack of a soft interfacial zone explains the high bond strength between fibers and matrix and thus high tensile strengths.

At the meso-scale level, the orientation and spatial distribution of fibers within the matrix are the focus of research using optical microscopes or other quantification equipment, such as x-ray images or electron microscopes. One general mechanical process of image analysis was performed by Wuest \cite{23467} to detect the fiber orientation distributions based on the oval shape of the cut sections of individual fibers.


            %% ====table====



    \section{MECHANICAL PROPERTIES}

        \subsection{Compressive strength and modulus of elasticity}
            %% ====text====
% Last updated @ [2010/10/27] by Mackie
% Last updated @ [2010/10/20] by Jun
% Last updated @ [2010/10/5] by Jun
% Discuss about the compressive test on UHP-FRC
The compressive tests on System 1 in Europe were conducted on 7 cm diameter, 14 cm long cylinders with both ends ground. Based on the test results from 196 specimens, a mean strength of 228 MPa was recorded, giving a characteristic value as 197 MPa with 95\% confidence \cite{22005}. The design strength is set as 180 MPa accordingly. The results on modulus of elasticity were between 61.4 GPa and 57 GPa, which is close to the number provided by the manufacturer \cite{22001}. The test results of compressive tests showed a linear behavior until failure.

The tests done by Graybeal \cite{22009} show that the presence of fibers can prevent the explosive failure and held the concrete together after the cylinder crushed in compression. A high loading rate of 1 MPa/second (150 psi/second) was used compare to the standard 0.24 MPa/second (35 psi/second) for normal concrete cylinders. The uses of higher loading rate was justified by the test results and help shrink the test duration. The paste of System 1 needs 12-24 hours to set. Two hours after initial setting, according to Graybeal \cite{22009}, the compressive strength of the cylinders in the lab environment can reach 70 MPa (10 ksi). After 24 hoours, the rate of strength gain will slows. The high early age strength is another very beneficial property in the precast industry.



        \subsection{Tensile properties, first crack and post-crack strength}
            %% ====text====
% Last updated @ [2010/10/27] by Mackie
% Last updated @ [2010/10/22] by Jun
% Last updated @ [2010/10/5] by Jun
% Discuss about the compressive test on UHP-FRC
The tensile strength of System 1 mainly refers to two aspects: the first crack strength and post-crack behavior. While the fiber content has limited impact on the first crack strength of System 1 according to Chanvillard \cite{22026}, it dominates the post-crack behavior by providing the bridging over the cracks. Four different tensile tests, flexural prism test, split cylinder test, direct tension test, and mortar briquette test were performed by Graybeal \cite{22009}. These four tests gave a best estimation of first crack strength of System 1 as 9 MPa for heat treated material.
Among the four test methods, the standard three-point flexural test was specified in French code \cite{22005} to catch the first crack strength. The strength value is corrected to account for the influence of the stress gradient based on the specimen dimensions. The corrected characteristic tensile strength is $f_t$=8.1 MPa, which is based on the tests of 196 4 cm x 4 cm x 16 cm prisms \cite{22005}.

For the post-crack behavior, the 70 mm x 70 mm x 280 mm prism with a 10 mm deep notch was used in the flexural test with concentrated load at the center according to Chanvillard \cite{22026}. The crack opening of the notch as well as the applied load were recorded. The assumed crack width versus stress relation was used to calculate the projected section moment based on different levels of crack width. Then the load versus crack width curve was correlated to its counterpart from real experiments. The crack width-stress relation then can be calibrated by the fitting of the two curves. The crack width was then correlated to the strain value to gain the stress-strain relation. 



        \subsection{Bond strength with UHP-FRC and tensile reinforcement}
            %% ====text====
% Last updated @ [2010/10/27] by Mackie
% Last updated @ [2010/10/22] by Jun
% Last updated @ [2010/10/5] by Jun
% Discuss about the bond between UHP-FRC and rebar
Bond properties between UHP-FRC and MMFX2 rebar are very important to the structural behavior of the passively reinforced UHP-FRC beams. Holschemacher \cite{04-8-5} investigated bond tests between pure UHP-FRC matrix and ribbed normal strength rebar by using pull-out specimens. The local bond stress was around 40-70 MPa for US {\#}3 rebars with 45 mm concrete cover. Some of the specimens with 25 mm cover failed in concrete splitting. The UHP-FRC used in the test had no fibers in the mix design and none of the specimens were subjected to heat treatment. Lubbers performed anchorage tests on UHP-FRC (Ductal$^{\textregistered}$) \cite{27027}. Low-relaxation, 12 mm diameter, 1862 MPa prestressing strands were embedded in UHP-FRC with a minimum bond length of 305 mm and no prestress force applied. All strands fractured during the pullout test, therefore the high bond strength was confirmed. For MMFX2 rebar, beam splice tests performed by Ansley \cite{27014} on US {\#}6 and US {\#}8 MMFX2 rebars show that the bond length required to yield the rebar is forty-five times the rebar diameter. After yielding, the nonlinear ductile response of the rebar material reduced the bond strength and changed the commonly brittle splice failure to a gradual and more ductile failure.

Adhesive bond strength between UHP-FRC and other materials was also investigated \cite{08-S3-3}. The glued connection between UHP-FRC pre-cast panels and steel truss chords was tested under bending and good performed was observed. This composite structure was used in a real pedestrian bridge project in Europe. The interface between UHP-FRC and metallic plates was also investigated, A 56 mm thick UHP-FRC concrete layer was poured on the 12 mm thick steel plate connected by sprinkle-in bauxite aggregate. With the neutral axis within the UHP-FRC layer, tensile cracking was expected. The rebar in the UHP-FRC layer optimized the crack location and postponed the initiation of the softening stage. Four point bending tests on this multi-layer plate were carried out. The results showed a simplified perfect yielding model may be accurate enough for modeling the UHP-FRC element if it is reinforced by steel rebars or steel plate.


    %\section{DURABILITIES}
        %% ====text====
% Last updated @ [2010/10/27] by Mackie
As UHPC is relatively new and construction projects utilizing the material have only existed for short periods of time, the research on durability of UHPC is primarily based on accelerated lab experiments. A series of durability tests, including rapid chloride ion penetrability test, chloride penetration test, scaling resistance test, abrasion resistance test, freeze-thaw resistance test,  and alkali-silica reaction test were conducted by Graybeal \cite{22009}. Results confirmed the high durability of System 1 against chemical as well as physical attack from environments typical for UHPC applications.


    \section{NUMERICAL MODELS AND SPECIFICATION}

        \subsection{Description of numerical models}
% Last updated @ [2010/10/27] by Mackie
% Last updated @ [2010/10/22] by Jun
% Last updated @ [2010/10/6]
% Discuss about the general numerical methods

A lot of research has been done on modeling and calibrating the structural responses of the UHP-FRC material at all scale levels, as shown in Fig.~\ref{Fig_modeling_scale_work}. The microstructure-based numerical models are usually based on the single fiber matrix interaction and consider the fiber contribution at different orientations and embedment lengths. The global response of all fibers can be accumulated based on the single fiber response plus a consideration of group effects. Based on the different failure modes assumed for the fiber-matrix interaction, this approach can lead to different results.

%This methodology has been successfully applied to other types of HPFRCC, such as engineered cementitious composites
% which methodology?
% such as  what




At the macro-scale level, the rule of mixture is a straight forward method to estimate the composite material properties based on the properties of individual components. However, this methods can only provide point estimation on the modulus of elasticity or the ultimate tensile strength as shown by Chanvillard \cite{22026}. By adopting the rule of mixture at the macro level, a spring and friction model \cite{83008,83009}, named the think model, was developed. The complete stress-strain curve for UHP-FRC can be obtained based on the individual stress-strain relation for cement matrix and fibers. The parameters in the model that represent the properties of the two components were calibrated based on the flexural test results. The model used in the French code \cite{22005} is a semi-curve fitting model based on the flexural test results. The general shape of the stress-strain curve was first determined by defining several typical stress-strain points on the curve, then these critical points were back calculated to make sure that the simulation results matched the experiment results.

Due to the fact that the response of UHP-FRC under multi-dimensional complicated stress states has not been fully calibrated, the development of the multi-dimensional material constitutive model always involves assumptions about the material responses. Therefore, the uniaxial think model was expanded to multi-dimensional level \cite{83009} using plasticity theory and a different yielding surface under different stress states.




        \subsection{Design specifications}
            %% ====text====
% Last updated @ [2010/10/27] by Mackie
% Last updated @ [2010/10/22] by Jun
% Last updated @ [2010/10/5]
% Discuss about the existing regulations on the design of UHP-FRC
Australia published the design recommendations on UHP-FRC \cite{PP-1} in 2000. It followed Australian Standard for Concrete structures AS 3600-1994 and provided the design methods for prestress concrete beams made of UHP-FRC.

In 2002, Association Fran\c{c}aise de G\'{e}nie Civil(AFGC) published the interim design guidelines on UHP-FRC. It includes the experiments on material properties as well as the engineering practices in France. The code is based on the French 'BPEL' code (BAEL 91 limit state reinforced concrete rules, 1999 revision), and combined with the 'BAEL' codes (BPEL 91 limit state prestressed concrete rules,1999 revision). The French code covered the material properties for service and ultimate limit states. The design methods for the flexural, shear, and torsion resisting members are stated. The standard test methods to calibrate the material properties using uniaxial compressive test and flexural prism tests are also included in the appendix.

In 2004, the concrete committee of Japan Society of Civil Engineers published the Recommendations for design and construction of UHP-FRC structures, which is called 'UFC' for short in Japan. The code was based on the knowledge and engineering experience gained through the construction of the Sakata-Mirai bridge and also used the French code as main reference. The English version \cite{PP-2} of this code was published in 2006.

Based on the French code and two phase material model developed at MIT, Davila proposed the design philosophy \cite{83008} regarding pre-stressed girders constructed with UHP-FRC. The design procedure follows the AASHTO LRFD requirement. 


    \section{APPLICATIONS}
        \subsection{Bridge deck application}
            %% ====text====
% Last updated @ [2010/10/22] by Jun
% Last updated @ [2010/10/5]
% Discuss about the existing regulations on the design of UHP-FRC
The advantage of using UHP-FRC material is obvious, it will help accelerate the construction due to precast members and high early strength if casted in field. It can also ensure a better durability over the 100 years life span. The drawback on the considerable high initial material cost can be eased by analyzing the life cycle cost which is expected to be comparable to the existing cast in place concrete deck system.

Graybeal \cite{22014} proposed a conceptual UHP-FRC two way waffle deck and performed a design verification based on the simplified material properties. The illustrated deck panel is 203 mm (8 in.) thick with ribs spacing at 610 mm (24 in.) going both directions. The thickness of the slab portion is 63.5 mm (2 1/2 in.) and the minimum width of the web is 76.2 mm (3 in.). The pre-stressing strands used as reinforcement were 12.7 mm (1/2 in.) in diameter at both directions.

Perry \cite{08-13-1} developed a UHP-FRC precast bridge deck and implemented it on the bridge in Canada at Rainy Lake, near Fort Francis, Ontario. The deck panels featured GFRP rebar on the top to prevent corrosion form the top deicing materials and mild steel rebar at bottom. The total height of the deck is 225 mm (8 7/8 in.) and UHP-FRC was cast in the pocket left in the panel in the field to form the composite actions between the deck and girder. The joint of two panels longitudinally was also filled by UHP-FRC to form the continuity of the decks. The width of the longitudinal deck joint is 210 mm based on the development length of GFRP rebars in the UHP-FRC from pull out tests. According to the author, no crack was found on this deck joint due to shrinkage partly because the width of this deck was only one third of the conventional deck joints. The shorten of the deck joint is largely attribute to the superior bond strength between GFRP rebar and UHP-FRC. No pre- or post-tensioning is used in this type of decks and screw feet were used for leveling.

Toutlemonde \cite{22021} performed a series of fatigue tests on the two way ribbed UHP-FRC decks in Europe. The height of the deck section is 380 mm (15 in.) with slab portion as 50 mm (2 in.) thick, the width of the rib is 100 mm (4 in.) at top and 70 mm(2 3/4 in.) at bottom, the height of the ribs are the same for both directions. The length of the deck segment is 2.5 m limited by transportation. Two pretension strands were used in the transverse direction, while the post tension was applied in the longitudinal direction to assemble the deck segments. The deck joint was filled by UHP-FRC casting on site. The experiment shown the fatigue initiation was consistently associated with the situation when the tensile stress exceed the linear limit. This deck also experienced the punching shear force \cite{22023} with a load zone as 400 mm by 400 mm (15.7 in. by 15.7 in.) on top of one honeycombs, no punching shear failure happened under this truck load. The punching shear failure was observed when using small load zone and it was found that the mean shear stress along the load surface was close to the tensile strength of UHP-FRC which justified the design method for punching shear. Furthermore, the anchor blocks for post-tension tendons \cite{22022} were also tested and validated to use in this deck system.

Punching shear properties of UHP-FRC was calibrated by Harris with results included in the report \cite{22003}. Several punching shear experiments were performed on twelve 1143 mm (45 in.) by 1143 mm (45 in.) small slabs with thickness as 50.8 mm, 76.2 mm and 101.6 mm (2 in., 2.5 in. and 3 in.), respectively. Small load zones (1 in. by 1 in. and 3 in. by 3 in.) were used in the tests. Several formulas predicting the punching shear strength for conventional concrete were verified for UHP-FRC and the results were compared to the experiment results. The modified ACI equation provided the best estimation and was recommended to use. Besides, three larger slabs were tested by the standard wheel tyre load and no punching shear failure was observed. The analysis verified by the experiment secured a minimum thickness of 25.4 mm (1 in.) of the slab to preventing the punching shear failure under the 203.2 mm by 508 mm (8 in. by 20 in.) wheel load patch.



        \subsection{Other structural applications}
            \subsubsection{Pedestrian bridge}
            %% ====text====
% Last updated @ [2010/10/27] by Mackie
% Last updated @ [2010/10/22] by Jun
% Last updated @ [2010/10/5]
% Discuss about the existing applications
Sherbrooke bridge in Quebec, Canada, is the first pedestrian bridge made of UHP-FRC. It has a 30 mm top slab and two inclined tube webs made of stainless steel and encased UHP-FRC. The bridge is 60 m long assembled by six prestressed segments without using any passive reinforcement.

% careful about commercialism

The Seonyu footbridge in Korea and the Sakata Mirai footbridge in Japan were both built in 2002. Japan built a second bridge using UHP-FRC in 2004. The Yamagata footbridge is built using the principle of a 35.3 m long, 3.5 m wide, and 0.95 m high square box girder frame. The Papatoetoe footbridge \cite{08-13-1} in New Zealand was designed by VSL Australia, using prestressed segments with 50 mm thick deck.

% Note, this is a repeat from the other file
%The world's first UHP-FRC traffic bridge was Valence bridge \cite{04-2-1} built in France in 2001 with BSI concrete. The world first traffic bridge made of UHP-FRC is Shepherds bridge in Australia. The first UHP-FRC traffic bridge made in North America is the Wapello Country Mars Hill Bridge.

Based on the test results, UHP-FRC has proven to be one possible solution to provide reduced maintenance costs and to improve durability for the highway bridge system. An experimental UHP-FRC bridge at FHWA was tested \cite{22008}. This optimized bridge is made of two $\Pi$ shaped girders; 21.3 m long by 2.44 m wide, with pre-tensioned strands. The design of these girders was optimized through an in-depth study conducted at M.I.T. (Massachusetts Institute of Technology) \cite{83009}. The material model and corresponding FEM analysis were performed to assess and predict the behavior the UHP-FRC prestressed girders.

            \subsubsection{Traffic bridge}
             %% ====text====
% Last updated @ [2010/10/27] by Mackie
The world's first UHPC traffic bridge was Valence bridge, built in France in 2001 with BSI concrete. Since then, there are another four bridges built in France using structural UHPC components. These are another bridge of Bourg-l��s-Valence, south of Lyon, the Saint-Pierre-la-Cour bridge in Mayenne, the N 34 bridge on the A51 motorway, and the Pinel bridge in Rouen \cite{08-12-4}. The world's first traffic bridge made of System 1 is Shepherds bridge in Australia. The first traffic bridge made of System 1 in North America is the Wapello County Mars Hill Bridge.The Kuyshu Bridge will be the first traffic bridge in Japan.



\clearpage
\newpage

\bibliographystyle{elsarticle-num}
\bibliography{../../lib/UHPC} 

% ========== check Chapter 2

% Last updated @ [2010/10/7]
% To do list
% 1 Fill the single-fiber interaction
% 2 Add variable engagement length modeling (sequential cracking consideration)
% 3 Add discussion on the multi-cracking scheme, in this research, assume one crack within critical length region
% 4 Add image analysis on the fiber orientation distributions
% 5 spatial distribution K function
% 6 standard spatial distribution backbone of F and G function from simulation results
% 7 In simulation, dose spatial size affect the orientation factor variance?


\chapter{MICRO/MESO LEVEL MATERIAL MODELS}

%% main text
\section{FIBER SPATIAL AND DIRECTIONAL DISTRIBUTIONS}

    \subsection{Introduction}
        %% ====text====
% Last updated @ [2010/10/6]
% Discuss about the introduction
In UHP-FRC, the fibers play an important role in resisting external load and thus their properties and distributions in the cement matrix are critical to the macro-mechanical properties of UHP-FRC. Usually the fiber volume fraction is 2\% for UHP-FRC and the material is treated as homogeneous and isotropic by assuming both the spatial and orientation distribution of fibers are uniform. But in real production, the fiber orientations are inevitably affected by the casting method and the boundaries of the molds. Thus, quantifying the degree of anisotropy of fiber orientation distribution is necessary in order to estimate the mechanical properties of the material, which will enable a safe but economic design approach when using UHP-FRC materials.

By casting UHP-FRC in a specially designed U-shaped sag box and analyzing the fiber orientation via image analysis, Patrick \cite{23412} confirmed that the fibers tend to align with the cement paste flow direction and the degree of alignment is related to the flow ability of cement paste. The influence of form boundaries on fiber alignment was utilized by Bernier \cite{23413} to create prisms with fibers aligned at a particular orientation. A plate was cast by temporarily placing parallel thin metal sheets along the flow direction with distance equal to the fiber length during the casting. After curing, several prisms were cut from that plate with different angles with respect to the flow direction. The mechanical tests on these prisms showed the distinct effects of the fiber alignment on the bending responses. The anisotropic distribution of fiber orientation was also confirmed by non-destructive electrical resistivity measurements on two slabs with different casting methods \cite{23401}. The flow directions matched the pattern of the measured anisotropic axis. Alternating current-impedance spectroscopy (AC-IS) was also applied as a non-destructive method to characterize the fiber orientation of cement-based fiber reinforced composites including UHP-FRC \cite{23402}. Good correlation was found between the results from AC-IS and those from image analysis regarding the anisotropic distribution of fiber orientations \cite{23452}.

To account for the effects of anisotropic distributed fibers in UHP-FRC, a reduction factor equal to 1.25 for normal cases, was introduced by the French guidelines \cite{22005} for the strength variation due to adverse fiber orientation distributions. However, this factor is mainly based on limited experimental results rather than theoretical derivations. There is an obvious lack of knowledge to link the type and degree of anisotropic fiber orientation distribution with the mechanical properties of the material. Currently, there is no way to quantify the anisotropic fiber orientation distribution other than to distinguish it from an assumed uniform distribution scenario.

Statistical analysis on the orientation distribution of short fibers in cement matrices were started in the early 1960s. Naaman \cite{25801} derived the distribution of single fiber pullout strength based on the assumption that the spatial and orientation distributions are statistically independent. In his theory, the probability density function (PDF) of the angle between the fiber and the normal vector of the cut plane is a sinusoidal function. Based on this conclusion, the orientation factor, which was originally introduced to calculate the number of fibers having an intersection with a unit area cut plane, was derived for 2D uniform and 3D uniform cases, and are equal to $2\diagup\pi$ and 0.5 respectively. Later some researchers introduced additional parameters, such as the fraction ratio of load-carrying fibers \cite{23408} and orientation efficiency factors \cite{23462} to count the stress effectiveness of the fiber with orientation other than zero degrees. However, some other researchers blended the effectiveness of fibers into the orientation factor term, which leads to discrepancies in the reported values for the same fiber orientation distribution. For example, an orientation factor of 0.375 for the 3D uniform scenario was used in one study \cite{23445} while its value was reported to range from 0.41 to 0.82 in a different study \cite{23457}. In this paper, the orientation factor will be strictly defined according to particular fiber orientation distributions. The value of the orientation factor for some special anisotropic cases affecting by the boundaries were investigated by Parviz \cite{23457}, David \cite{23410}, and Lee \cite{23466} for straight and ring-type fibers; however, these approaches only apply to those specimens with localized alignment of fibers, for example, the region beneath the surface that is affected by the form walls. The global preference of fiber orientation, such as that caused by the cement flow cannot be considered and needs further investigation.

The objective is to create a statistical model of fiber orientation distribution to quantify the global preference of fiber alignment. Although the theory can be applied to any kind of FRC, one typical UHP-FRC material was selected and the deterministic parameters in the analyses were based on this particular material. The micro structure \cite{23103} as well as the structural behavior \cite{22009} of this material have been well investigated. The number of fibers crossing per unit area under several anisotropic distribution scenarios are calculated. The objectives are illustrated using both statistical derivation and numerical simulation. The results were also checked using physical experimental specimens by counting the fibers on the section cuts obtained from two UHP-FRC prisms in the parallel and perpendicular direction with respect to the cast/flow direction. 

    \subsection{Statistical approach and orientation factors}

        \subsubsection{Uniform distribution}
            %% ====text====
% Last updated @ [2010/10/6]
% Discuss about the uniform distribution
Based on previous literature results \cite{25801}, under the uniform distribution case, the distribution of the inclination angle $\Theta$ between the fiber line and any axis has the probability density function (PDF) as shown in Eq.~\ref{eq_uniform_ft}. Generally, the probability of fibers having intersection with the cutting plane perpendicular to the considered axis can be obtained by Eq.~\ref{eq_general_ft_cross}. And for the uniformly distributed case, the results are shown in Eq.~\ref{eq_uniform_ft_cross}.







The total number of fibers within a unit volume $N_V$ and the number of the fibers across a unit area $N_s$ can be expressed in Eq.~\ref{eq_Nv} and Eq.~\ref{eq_Ns_alpha}, respectively.



In which $\alpha _{orient}$ is the fiber orientation factor. Other parameters are the fiber properties and were treated as deterministic. The descriptions and corresponding values for a particular UHP-FRC material are shown in Table~\ref{Table_fiber_para}. The expectation of $N_s$, and thus the orientation factors, can be calculated directly from $f_{\Theta}(\theta)$ by Eq.~\ref{eq_Ns_expression} assuming the fiber orientation and embedment length are statistically independent.


In which $\theta _{crit}  = cos^{ - 1} (2\left| x \right|/l_f )$, and $x$ is the distance from the cut plane to the gravity center of the fiber. The orientation factor for 3D uniformly distribution case is calculated to be 0.5. 
            %% ====table====


        \subsubsection{Anisotropic distributions with principle axis}
            %% ====text====
% Last updated @ [2010/10/6]
% Discuss about the type of anisotropic
Form boundary restriction and the cement flow influence are the two main factors causing an anisotropic fiber orientation distribution. When casting specimens like thin plates that have one relatively small dimension compared to the others, the fibers tend to align in the plane. On the other hand, for slim UHP-FRC prisms, the cement paste is usually dumped into the mold at one end and then allowed to flow to the other end until filling the mold. The fibers tend to align with the flow direction and form the preferred principle direction of the fiber orientations. Although the two anisotropic distributions mentioned are caused by different mechanisms, they have the same anisotropic characteristics with one principle axis and randomly distributed orientation of the fiber projections in a plane perpendicular to that axis. The compliance matrix of a transversely isotropic material is shown in Eq.~\ref{eq_ss2}, and is representative of the mechanical properties of the materials investigated. The different axes configurations is shown in Fig.~\ref{Fig_coord_transverse} for the two anisotropic cases and the name `weak-principle' (WP) and `strong-principle' (SP) are adopted for all subsequent discussions in the paper. There are total of five independent parameters for the three-dimensional stress state. However, based on the loading conditions of interest, the matrices can be further simplified. For example, in the WP case, the t-t plane stress response is of interest, while for the SP case, the p-t plane stress state is usually needed. The corresponding compliance matrices for the two plane stress cases are shown in Eq.~\ref{eq_plane_stress_1} and Eq.~\ref{eq_plane_stress_2}, respectively.




While the fiber anisotropic conditions only have limited impact on the elastic properties, it dominates the fiber bridging effects and thus determines the degradation of the stiffness due to concrete cracking. In order to estimate the pre- and post-crack properties of the material, the properties of fibers crossing any arbitrary cracking plane need be investigated. The orientation factor and the actual orientation distribution are the basis for any modeling approaches. 
            %% ====figure====



        \subsubsection{Derivation of fiber orientation distribution under anisotropic distributions}

            %% ====text====
% Last updated @ [2010/10/6]
% Discuss about the derivations on distributions
By introduce random variable $S_{WP}$ and $S_{SP}$ with PDFs shown in Eq.~\ref{eq_s1s2}, the PDF of the angle between fiber line and principle axis were obtained as shown in Eq.~\ref{eq_fp1_fp2} assuming $\Theta _{p}  = \cos ^{ - 1} (S)$.





The unit area on the cut section plane is defined as the associated unit area (AUA). The fiber orientation distribution for those fibers having intersections with the AUA with the cut section perpendicular to the principle axis can be calculated by Eq.~\ref{eq_general_ft_cross} and PDFs of $\Theta_{crossing}$ are shown in Eq.~\ref{eq_fp1_fp2_crossing}, in which, parameter $c$ is a numerical constant to fulfill the normalization property of the PDF.


The angle between a fiber and any arbitrary axis is denoted as $\Theta_\varphi$ with $\varphi$ representing the angle between the arbitrary axis considered and the principle axis. The coordinate system used in the derivation and the notation for all referenced angles are shown in Fig.~\ref{fig_angle_coordinates}. The known relation between these angles are shown in Eq.~\ref{eq_angle_relation}. $\Theta_\varphi$ can be expressed via $\Theta_{p}$ and $\Theta_{pt}$ by Eq.~\ref{eq_fii_angle}. The PDF of $\Theta_{p}$ is shown in Eq.~\ref{eq_fp1_fp2} for both cases, and $\Theta_{pt}$ is randomly distributed between 0 and $\pi/2$.
Due to the transverse isotropy, the arbitrary axis can be placed in one of the p-t plane with varying $\varphi$ angle as shown in Fig.~\ref{fig_angle_coordinates}.



The explicit expression of  $f_{\Theta_\varphi}({\theta _\varphi})$ is difficult to obtain through integration due to the quantities in the integration. Therefore, the analysis with respect to the transverse axis was performed when $\varphi=\pi/2$. The expression of  $f_{\Theta_t}(\theta_t)$ is derived for both WP and SP cases based on the existing method \cite{60009}. Once the distribution is derived, the distribution of orientations for those fiber having intersections with the AUA can be derived using Eq.~\ref{eq_general_ft_cross}. Some results are summarized in Table~\ref{Table_Transverse_ft}. The parameter $\kappa$ in the table was introduced as a uniformity measurement and equals $1/\beta$. When $\kappa=1$, the orientation distribution is uniform, while when $\kappa=0$, all fibers are aligned in the same direction. 

            %% ====table====

            %% ====figure====






        \subsubsection{Derivation of orientation factors under anisotropic distributions}
            %% ====text====
The orientation factor $\alpha _{orient}$ can be calculated based on the PDF of fibers crossing the AUA using the procedure described in Eq.~\ref{eq_general_ft_cross}. The results are simple for the principle axis, as shown in Eq.~\ref{eq_orient_principle}. The orientation factors versus uniformity parameter $\kappa$ are also plotted in Fig.~\ref{Fig_orient_factor} for both principle and transverse axes. The results with respect to other arbitrary axes were investigated by simulation approaches discussed in the next section.




    \subsection{Simulation approach}
        \subsubsection{Fiber generation}

            %% ====text====
% Last updated @ [2010/10/6]
% Discuss about the simulation method, how to generate the fibers under distribution assumptions
A numerical simulation was programmed (using Matlab\circledR) by randomly placing a particular number of fibers within a cube in 3D space. The fiber length, diameter, and fiber volume fraction are deterministic, thus the total numbers of fibers can be calculated using Eq.~\ref{eq_Nv}. Three uniformly distributed random numbers were generated for each individual fiber to decide the spatial location of the gravity center. Three algorithms were used to generate the angle couple of ($\theta _1 ,\theta _2 $) representing the uniformly distribution, the WP anisotropic case and the SP anisotropic case discussed previously. The coordinate system is shown in Fig.~\ref{fig_angle_coordinates}. For those distributions bounded by $\left[ {0,\pi /2} \right]$, the limits were mapped to $\left[ {0,2\pi } \right]$. The results of the fiber generation are shown in Fig.~\ref{fiber_place_case_all} for the two cases of anisotropic distributions with $\kappa= 0.2$. The fibers having an intersection with the AUA with cut planes perpendicular to the principle and transverse axes are displayed. 

            %% ====figure====

        \subsubsection{Random AUA generation and intersection detection}
            %% ====text====
% Last updated @ [2010/10/6]
% Discuss about the simulation, how to generate the cut plane and calculate the intersections
After all fibers were placed in space under a particular orientation distribution, a circular AUA with axis having angle $\theta_\varphi$ between the positive x-axis was placed in 3D space with its center at the center of the cube. The number of fibers having intersections with AUA will be counted based on the algorithm discussed as follows.
The direction cosines for any vector in 3D space can be expressed as:
in which the numerical realization of parameter $i$ represents the $i$th fiber in space, while when $i = c$, it is implied to represent the normal vector of the AUA. Based on the direction cosines, the analytical expression of the fiber line and cut plane are expressed as follows:


in which point $P_{o,i}  = [x_{0,i} ,y_{0,i} ,z_{0,i} ]$ is the gravity center of the $i$th fiber. The $i$th intersection point $P_{{\mathop{\rm int}} ,i}  = [x_{{\mathop{\rm int}} ,i} ,y_{{\mathop{\rm int}} ,i} ,z_{{\mathop{\rm int}} ,i} ]$ can be calculated by solving Eq.~\ref{eq19}. The results are as follows:


Only those fibers meeting the following two criteria in Eq.~\ref{eq21} are counted towards the total number of fibers crossing the AUA.

The start and the end point of the fiber are

Criterion I ensures that the intersection lies within the AUA, and Criterion II ensures that the intersection lies between the bound of the start and end points on the fiber line. The orientation between the fiber line and cut plane can then be calculated as
The distance from any point $P = [a,b,c]$ to the cut plane is


The shorter distance from the two ends of a given fiber to the cut plane is 

The embedment length can be calculated as

    \subsection{Comparison between simulation results and theoretical results}
        \subsubsection{Simulation verification}

            %% ====text====
The orientation distribution of fibers crossing the AUA was obtained via simulation and compared to the previously derived equations for the case of $\kappa=0.2$ as in Fig.~\ref{fiber_crossing_PDF_case_all}. The two dotted lines show the theoretical PDF curves for the uniformly distributed cases and the special case with respect to the type and degree of anisotropy. The curves from theoretical derivation match with the simulation bar chart. Good correlation was also found between the calculated and simulated orientation factors for the principle and transverse directions, thus validating the simulation procedure. The uncorrelated nature between the fiber orientation distribution and the embedment length distribution was also confirmed. 

        \subsubsection{Complete orientation factors and simplified equations}

            %% ====text====
% Last updated @ [2010/10/6]
% Discuss about the numerical simulation, make sure this is comparable
Based on the simulation results, the orientation factor for an arbitrary cut plane axis with $\varphi$ from zero to 90 degrees with 22.5 degree intervals and various levels of anisotropy with $\kappa=$1, 0.5, 0.2, and 0.01 are plotted in Fig.~\ref{orientation_factors} for both WP and SP cases. A simplified equation was developed for SP case as shown in Eq.~\ref{eq_orient_kappa_case2}. The equation has a very simple form and is close enough to the simulation results when $\kappa>0.2$. The orientation factor from the equations and simulations were compared and the results are shown in Fig.~\ref{simplify_formula}. For case WP, only the transverse plane is of interest, and the orientation of the transverse axis with varying $\kappa$ values can be expressed by linear interpolation using Eq.~\ref{eq_orient_kappa_case1}.



            %% ====figure====





        \subsubsection{Simplified PDF of fiber crossing}

            %% ====text====
% Last updated @ [2010/10/6]
% Discuss about the proposed simplified equations to present the orientation PDF
By observation,it was found that the generalized form of the PDF of the crossing fiber orientation distribution can be expressed as shown in Eq.~\ref{eq_ft_fii}.


in which $C_{sc}$ is a constant and can be expressed as in Eq.~\ref{eq_lamda}.



The values of $r_c$ and $r_s$ are related to the anisotropy type, the $\kappa$ value, and the $\varphi$ angle. The coefficients can be estimated by curve fitting based on simulation results. For case WP, because only the transverse plane is of interest, the curve fitting results are shown in Table ~\ref{Table_ft_cf_case1} for $\varphi=\pi/2$. For case SP, the results corresponding to $\kappa$, $\varphi$ combination are shown in Table ~\ref{Table_ft_cf_case2}. As an illustration, the comparison of curve fitting and simulation results for case SP when $\kappa=0.1$ is shown in Fig.~\ref{Fig_cf_k_01}. 
            %% ====table====




            %% ====figure====


        \subsubsection{The inter-fiber minimum distance with respect to the fiber orientation distributions}

            %% ====text====
% Last updated @ [2010/10/6]
% Discuss about the intra-distance between fibers
The minimum distance between fibers was also investigated based on the algorithm calculating the distance between two finite length straight lines in 3D space. The inter-fiber distance is related to the interfacial stress transfer mechanism before and after the crack happens. The simulation results are shown in Fig.~\ref{fig_inner_distance}. Interestingly, the minimum distances do not change drastically with respect to either the type or the degree of the fiber alignment, which reveals that this quantity is more related to the fiber volume fraction rather than the orientation distributions. Therefore, the interfacial stress transfer model used on the uniform distribution scenario can also be used under anisotropic fiber orientation distributions.

For the case of perfectly aligned continuous fibers with a square packing pattern, the minimum distance $\left( {2R} \right)$ between fibers can be estimated using the following equation \cite{60011}.


In which $2R$ is the minimum distance between the fibers and the estimated value from Eq.~\ref{eq26} is 2.54 mm. It is a conservative estimation and based on the simulation results shown in Fig.~\ref{fig_inner_distance}, $1.2(2R)=3.05$ mm was used as an estimation for all anisotropic cases. 
            %% ====figure====




    \subsection{Experimental estimation of uniformity parameter}
        \subsubsection{Specimen design}
% Last updated @ [2010/10/6]
% Discuss about the specimen design and cut plane schedules
Two UHP-FRC prisms were cast with the above-mentioned UHP-FRC material. The cement paste passed the flow table test, thus satisfying the maximum viscosity specified by the manufacturer. One prism was cast by dumping the cement paste at one end of the form and letting the material flow and fill the voids. As needed, additional material was added behind the flow front. The second prism was cast in the same way as the first one, except that there was a 200-gauss electro-magnetic field imposed on the specimen during the casting, as shown in Fig.~\ref{fig_magnetic_prism}. The magnetic field causes a force in the misaligned steel fibers and was expected to increase the level of anisotropy. After hardening, the prisms were cut at two directions parallel and perpendicular to the flow direction as shown in Fig.~\ref{fig15} in order to applied the image analysis. 

        \subsubsection{Image analysis}
            %% ====text====
% Last updated by Jun @ [2010/10/22]
% Last updated @ [2010/10/6]
% Discuss about the general numerical methods

The fiber dispersion of FRC manufactured by extrusion method was investigated using image process \cite{23453}. After the prism went over the four point bending test, the images were taken by electron microscope at 50x magnification on the two opposite sections near the location of fracture. Two observation windows of 3.05 mm$^2$ area were picked up on the center regions on the two sections to avoid edge effect. And the pattern found on the two sides were intra-confirm. Images were selected and processed to find the coordinates of the center point for each fiber by using the optimas image analysis software (image-pro). The distance between any two points as well as the orientations between them were calculated and the first and second moment statistical analysis were performed based on those results to test if the cluster existed in the material and if the material is isotropic. The functions used to exam the fiber spatial distribution in the statistical analysis were $F_r$, $G_r$, and $K_r$, with the expressions in Eq.~\ref{eq_spatial_functions}. In which, function $N()$ is the number of fibers or grid points that meets the criterion shown in the parenthesis. and function $D_{min}()$ is the minimum distance between two spatial points.Parameter $A$ ia the total sectional area, while $A_r$ is the area of the inner section that was used in the counting to avoid boundary influence. Letter $g$ and $f$ were used to represent the individual grid and fiber point.




% Start with the literature review-- orientation check
Both the image analysis and the AC-impedance spectroscopy were performed \cite{23452} to detect the fiber distribution in concrete. For image capturing, an optical microscope and a high resolution digital camera with a macro lens were used. In order to measure the clumping of the fiber, $K$ function were measured and compared to the value form the simulating passion process \cite{23453}.In order or determine the orientation, the length of the major and minor axis, and the in plane angle is required. The outline of the ellipse image were detected and used to best fit the ellipse using methods discussed in \cite{23455}.


% Work done on UHP-FRC --- uniformity check
The image analysis was also performed in this research in order to confirm the uniformity of the fiber spatial distribution as well quantify the fiber orientation distribution. The sections of prisms were cut using an industry chop-off saw and then polished with fine sand paper. High resolution digital camera was used to take photos of sections with calibration rulers by side. The fibers were bright on the photo due to the reflected light from the smooth metal surface. The photos were calibrated and processed using image-pro plus 6.0. The information of fibers (spatial location, aspect ratio, area and perimeter) were exported. Corresponding analysis program was coded in Python to examine the spatial and orientation parameters. Data filter was setup based on the geometry correlation. The area and perimeter value for each point based on the length of major and minor axis were calculated assuming all points are ellipse. The data points with more than 1\% discrepancy between measured area or perimeter from the calculated value will be discarded. For spatial distribution, three functions were calculated \cite{23453} to demonstrate the spatial distribution, while for fiber orientation, the commonly used orientation factor will be calculated based on the aspect ratio and the distribution of angles for each casting technical group will be created based on multiply observations. The average orientation factor for the section of same distribution condition were also obtained. The image analysis process is shown in Fig.~\ref{Fig_imageanalysis_proc}.

One typical section cut from UHP-FRC prism was used to exam the spatial distribution uniformity. The optical microscope was used to take photos from the small region on the prism, then the photos were jointed together to have the global view on the section. The results on the sectional analysis and the plots of $F$ and $G$ function is shown in Fig.~\ref{Fig_image_uniformity_check}.

% Work done on UHP-FRC --- orientation distribution
%Need to add the orientation distribution discussion here

% Work done on UHP-FRC --- Fiber count
Image analysis was applied to the section cuts of two prisms as shown in Fig.~\ref{fig15}, and the number of fibers per unit area was counted for the principle and transverse directions. The typical photos for two types of cuts are shown in Fig.~\ref{fig16}, and the fiber count results are summarized in Fig.~\ref{fig_exp_fiber_counts} for two UHP-FRC specimens individually. The results from cut-1 of prism 2 are far smaller than the other perpendicular cuts and thus omitted in Fig.~\ref{fig_exp_fiber_counts}. It is obvious from the plot that the fibers in both prisms were aligned with respect to the cast/flow direction. The value of uniformity parameter $\kappa$ can be estimated based on the measured fiber per unit area for both the principle and transverse directions based on Eq.~\ref{eq_orient_kappa_case2}, the results are shown in Fig.~\ref{fig_exp_fiber_counts} as well. This methods use the internal relation of the orientation factors on section cuts with different orientation, thus the results can autocorrelated to each other to confirmed the fiber distribution. It is more straight forward and less equipment demanding than calibrated the orientation distribution by identifying the orientation for individual fiber intersection point.

% Work done on UHP-FRC --- prism test

The mechanical tests were also performed on these two prisms. The load versus displacement curve as well as the fiber arrangement after cracking are shown in Eq.~\ref{Fig_mechanical_res}.




            %% ====figure====
















\section{SINGLE FIBER MATRIX INTERACTION}

    \subsection{Introduction}
        %% ====text====
% Last updated by Jun @ [2010/10/22]
% Last updated @ [2010/10/5]
% Discuss about the simple fiber matrix interaction
The estimation of mechanical properties of UHP-FRC material with different fiber orientation distributions is directly based on the single fiber-matrix interaction rule. The bond strength between fiber and cement matrix comes from three parts: adhesion, mechanical anchorage, and friction. The failure mode is largely depend on the fiber strength, interface conditions, and the cement strength. A lot of researches have been working on this topic in order to build/verify the stiffness and strength of the individual interaction to consider the impact from different embedment length, and incline angles other than zero degree. Various efficacy factor with respect to the embedment length $\eta _l$ and inclined angles $\eta _\theta$ were introduced. The snubbing effects \cite{60004} is introduced in order to reflect the realistic load transfer mode. The constructed model is sensitive to the type of cement matrix and type of fibers added. And the effectiveness of the single-fiber matrix interaction is also dominated by the accumulation approaches of individual interaction in the spatial and sequential domain.

%assembling process
The macro properties of material can be derived based on the assumptions about the spatial and sequential distribution of the individual interaction between fiber and cement matrix. The spatial distribution provides the information about the stress and crack distribution along the loading path, therefore, the displacement based single-fiber matrix relation can be translated to the global strain based stress model. The most commonly practice of this translation is to determine a characterization length, divide by which, the crack width can be converted to strain value assuming there is only one crack developed within this region. The sequential distribution determines the degree of activation of individual fiber in load transfer at specific crack width of the section. The degree of activation related to the combination from the embedment length and inclined angle. it reveals the fact that not all fibers will be fully activated/deactivated at the same time (crack width). This is especially true after the section cracks. However, this impact factor is usually considered by altering the single-fiber matrix interaction rule and assuming all fiber across the section is equally and simultaneously effective. One of the advantages of this simplified approach is that the model can be easily calibrated via the pull out test results using single fiber and fiber groups without considering the details about the sequential distribution. Another important impact factor is the group effect between fibers due to the overlapping of the stress field in the matrix causing by individual interaction. The degree of the influence is dependent largely on the inter-fiber distance and relative stiffness of the two material, which determine the possibilities of the overlapping. This factor is usually considered by introduce the grouping efficacy factor $\eta_g$ into the terms that represent the fiber contribution.


%Practice in this research
The single fiber-matrix interaction rule before and after matrix crack with respect to UHP-FRC will be discussed in the next section. Till now, there is no widely accepted relation established for UHP-FRC, and very limited experiments have been done on the single fiber or fiber group-matrix interaction of the material type investigated in this research. Therefore, commonly used and simplified assumptions will be adopted for the estimation work presented here. the characterization length approach and sequential average approach were used to convert the response from single fiber to fiber groups. The same level of grouping effects ($\eta_g$=1) were considered due to the fact that the inter-fiber distance is not highly sensitive to the fiber orientations, which is the main influence factor investigated. Although simplified accumulation approach is used, the previously derived models to represent the fiber anisotropic orientation distribution is valid for any advanced accumulation models, and is a necessary if need to consider the orientation anisotropy influence within the model framework (accumulation approach plus single fiber-matrix interaction relation).


    \subsection{Interaction between fiber and cement matrix before cracking}

        \subsubsection{Single aligned fibers}
            %% ====text====
% Last updated by Jun @ [2010/10/22]
The estimation of modulus of elasticity will be based on the work of Cox \cite{23208} who assume a perfect elastic interaction between fibers and matrix and equally effectiveness for all fibers in the matrix. The normal stress and interfacial shear stress distribution of an aligned short fiber embedded in the matrix without any debonding or the matrix cracks,  are shown in Fig.~\ref{interfacial_shear}. The theory is referred as shear lag theory, and the effective modulus of elasticity can be estimated based on the average normal stress along the fiber length. The effective modulus of elasticity expression is shown in Eq.~\ref{eq_E_effective}.



in which the efficacy of embedment length is 


$E_f$ is the modulus of elasticity of the fiber, and $r_f $ and $l_f $ are the radius and length of the fiber. $G_m $ is the shear modulus of the matrix. $R$ is the minimum distance of the neighboring fibers. Based on the simulation results in previous section, the $R$ value has small variations with respect to the level of anisotropy and the results based on the uniform distribution was used for all anisotropy cases. By using the typical value of UHP-FRC as listed in Table ~\ref{Table_para_cox}, the numerical value of effective modulus of elasticity of aligned fibers can be obtained. 
            %% ====figure====


            %% ====table====

        \subsubsection{Single inclined fibers}
            %% ====text====
% Last updated by Jun @ [2010/10/22]
% Last updated @ [2010/10/6]
% Discuss about the interaction at single fiber scale
The inclined fiber contribution is greatly reduced when compared to aligned fibers, the reduction is usually linked to the inclined angle as shown in Eq.~\ref{eq_incline_reduction}. Parameter $f_d$ represent the degree of inclination and equals to various number in different proposed models.






    \subsection{Interaction between fiber and cement matrix after cracking}
% Last updated by Jun @ [2010/10/22]
The interested information about the single fiber-matrix interaction after cracking is the load versus displacement relation. One of the simplest approach is adopting an average sense sequential accumulation scheme and create an average stress model with respect to individual fiber condition. By doing this, the ultimate tensile strength can be obtained using the method of point estimation by accumulate the interfacial shear stress of all fibers. Two efficacy factors with respect to embedment length and incline angle were introduced as shown in Eq.~\ref{eq29}. The efficacy factor for incline fibers has a maximum value at $\theta$=90deg, which reveals the fact that fiber with mild inclination has the highest load transfer capacity from single fiber pullout results. Although the test results were for normal strength concrete, the interaction between fiber and UHP-FRC is expected to have the similar responses. 




\section{POINT ESTIMATION ON THE MECHANICAL PROPERTIES}


    \subsection{Estimation of modulus of elasticity}
% Last updated by Jun @ [2010/10/22]
% Last updated @ [2010/10/6]
% Discuss about the estimation of modulus E
The expression of the modulus of the composites is

In which, $N_s$ is the number of fibers crossing per unit area. The expectation of the modulus of elasticity can be estimated based on the simplified equation of orientation factors and the general equation of fiber orientation distribution. The expression is shown in Eq.~\ref{eq_E_expectation}.


with $\eta _\theta$ in Eq.~\ref{eq_eta_theta} based on the general form of $\eta _\theta$ in Eq.~\ref{eq_incline_reduction}, $\eta _l$ in Eq.~\ref{eq_eta_l}, and $\alpha _{orient}$ in Eq.~\ref{eq_orient_kappa_case1} or Eq.~\ref{eq_orient_kappa_case2} for case `WP' and `SP'.


The results for the plane stress state and the WP and SP anisotropic cases are plot in Fig.~\ref{fig_modulus_elasticity} with $f_d=4$. It is seen from the plot that due to the small fiber fraction in the composites, the influence of anisotropic fiber orientation distribution has very limited impact on the modulus of elasticity, especially for case WP. Thus, the elastic properties including modulus of elasticity, Poisson's ratio, and shear modulus can be estimated by using the results for uniformly distributed case for the material investigated in this paper. 
    \subsection{Estimation of ultimate tensile strength}
% Last updated by Jun @ [2010/10/22]
It is believed that the ultimate tensile strength of UHP-FRC is directly determined by fibers bridging force across the cracks. For high strength smooth steel fibers without end hooks, the final failure mode of the fibers is most likely to be fiber pullout. Thus from a macroscopic point of view, the ultimate tensile stress of UHP-FRC can be written as follows for a unit area cross section.


In which $\eta_g$ represents the global effectiveness factor due to group effects (the interaction between fibers), and is mainly related to the volume fraction and type of matrix. The individual fiber ultimate stress $P_{ult} (z_i ,\theta _i )$ is estimated by multiplying the average interfacial stress $\tau (l_{em,i},\theta _{crossing,i} )$ with the embedment length $l_{em,i}$ and the fiber circumference $\pi d_f$. The term in the summation sign can be rewritten as

In which $\tau _{fu}$ is the ultimate interfacial shear stress under the assumptions that the inclination angle to the cut plane is zero and the shear stress is uniformly distributed over the entire embedment length.

The expectation of term in  Eq.~\ref{eq28} can be simplified based on the fact that embedment length is always a uniform distribution between $[0,l_f /2]$ regardless of fiber orientation distributions. And thus,

Insert Eq.~\ref{eq30} into Eq.~\ref{eq27},
in which
and the expectation and variance of $\theta$ are shown in Eq.~\ref{eq32} and Eq.~\ref{Var_t}, respectively.


Based on the simplified equations for orientation factors and the distribution of $\theta_{crossing}$, the ultimate tensile stress for different types and degrees of anisotropy can be obtained. The $\eta _{eff}$ results are shown in Fig.~\ref{fig_ultimate_eff} based on assumptions shown in Eq.~\ref{eq29}. 
\section{CONCLUSIONS AND DISCUSSIONS}
% Last updated by Jun @ [2010/10/22]
% Last updated @ [2010/10/5]
% Discuss about the conclusion and discussion
The macro-mechanical properties of ultra-high performance fiber reinforced concrete (UHP-FRC), particularly ultimate tensile strength, are dependent on the type and alignment of steel fibers embedded in the cement matrix. Based on the statistical derivations and numerical simulations, the estimated fiber orientation factors for anisotropic cases weak-principal (WP) and strong-principal (SP) were derived as shown in Eq.~\ref{eq_orient_kappa_case1} and Eq.~\ref{eq_orient_kappa_case2}. For case WP, the properties in the t-t plane are independent of the cut plane directions, so the orientation factor was only affected by the uniformity parameter $\kappa$. While for case SP, the properties in the p-t plane were of interest and the orientation factor is affected by the direction of cut plane represented by the parameter $\varphi$. Furthermore, the orientation distribution for the fibers having intersections with any particular plane was modeled with generalized expressions shown in Eq.~\ref{eq_ft_fii}.  The parameters listed in Table~\ref{Table_ft_cf_case1} and Table~\ref{Table_ft_cf_case2} can be determined by the type of anisotropy, direction of the cut plane $\varphi$, and the level of uniformity $\kappa$.

By using the material properties for a particular UHP-FRC and particular assumptions on singe fiber matrix interaction rule, it was found that the elastic modulus does not change drastically with either the type or the level of the orientation anisotropy, while the ultimate tensile strength can change significantly. Although one particular UHP-FRC was used for evaluation purpose in this paper, the proposed model can be applied to any kind of FRC with various fiber volume fraction and aspect ratios, as long as the type of orientation anisotropy can be approximately classified by one of the proposed cases. The cement paste viscosity, flow rate, and external influences may have an impact on the orientation distributions and the impact can be quantified using the model proposed in this paper. As an illustration, the uniformity parameter $\kappa$ for two prisms with and without magnetic treatment were calibrated using image analysis by counting fiber numbers per unit area at two perpendicular section cuts. This paper does not provide direct correlation between the casting condition and final mechanical properties; however, the presented results can facilitate such a calibration process by providing the quantification of the anisotropy levels and corresponding statistical fiber distribution status. 
\clearpage
\newpage
\bibliographystyle{elsarticle-num}
\bibliography{../../lib/UHPC}

%% The Appendices part is started with the command \appendix;
%% appendix sections are then done as normal sections
%% \appendix

%% \section{}
%% \label{}

%% References
%%
%% Following citation commands can be used in the body text:
%% Usage of \cite is as follows:
%%   \cite{key}         ==>>  [#]
%%   \cite[chap. 2]{key} ==>> [#, chap. 2]
%%

%% References with bibTeX database:







% ========== check Chapter 3
\chapter{MACRO-LEVEL MATERIAL MODELS}

    \section{DETAILED DISCUSSION ON UNIAXIAL MODELS}
        \subsection{General discussions}
% Last updated by Jun @ [2010/10/22]
% Last updated @ [2010/10/5]
% This is the introduction to the macro-scale modeling
The macro-level mechanical properties of the material are very important for the application of UHP-FRC in the civil engineering projects. The mechanical properties can be simulated based on the micro-scale modeling approach as shown in the previous chapter. They can also, and more directly, obtained by the mechanical tests. Due to the fact that UHP-FRC behaves very differently in the compression and tension, different types of mechanical tests are needed in order to calibrate the material properties under different stress conditions. A lot of experiments, including the uniaxial tensile and compressive tests, have been done to calibrate the material properties. Several numerical models were built to simulate the responses based on these test results. However, the literature existed models have very different post crack stress-strain relations and different ways to consider the size effect. Therefore, these uniaxial models need to be verified for various situations and also need further simplifications in order to be easily implemented in the design equations. Theoretically, all the impact factors discussed in the following sections shall be considered in the numerical models to reflect the appropriate material properties.  

        \subsection{Influence factors}

         \subsubsection{Influence of curing methods}
% Last updated by Jun @ [2010/10/22]
% Last updated @ [2010/10/5]
%
% edited Mackie 2010/10/10 10:47am
%
% Here point out the curing method and general influence
In order to ensure the high quality of UHP-FRC, steam curing at 90 degrees Celsius and 95\% humidity for about 48 hours under stable thermal condition are usually required \cite{22005}. This heat curing process is often prohibitive given specimen size and available resources. Therefore, untreated UHP-FRC structural members are also of interest, particularly given that the strength of untreated UHP-FRC is in the range of 120 MPa and 160 MPa, which is still higher than traditional high strength concrete. According to previous experimental results \cite{22009}, the high temperature and humidity can accelerate the hydration speed and cause the material to achieve the designated high compressive strength right after curing. The UHP-FRC material without curing will still gain strength much faster than the normal strength concrete. However, a lower ultimate compressive strength is expected when compared to the heat treated material.

         \subsubsection{Influence of fiber orientation distribution}
% Last updated by Jun @ [2010/10/22]
% Last updated @ [2010/10/5]
%
% edited Mackie 2010/10/10 10:52am
%
% Here point out the influence of fiber orientation

Vodicka \cite{Vodicka} suggested that the critical fiber volume fraction to achieve a homogeneous distribution is about 0.4\% to 0.5\%. The  homogeneous distribution can be achieved for up to 8\% fiber volume fraction when using short steel fibers \cite{Taerwe}. Therefore, 2\% of steel fiber volume fraction falls within the range where a homogeneous distribution can be assumed, while at the same time not affecting the workability of the cement pastes. However, the assumption of random orientation distribution of individual fibers is usually violated due to the flow casting method. This casting method is widely used with or even without external vibrations by taking advantage of the self-consolidation property of UHP-FRC. By casting UHP-FRC in a specially designed U-shaped sag box and analyzing the fiber orientation via image analysis, Patrick \cite{23412} confirmed that the fibers tend to align with the cement paste flow direction and the degree of alignment is related to the flow ability of the cement paste. The influence of form boundaries on fiber alignment was utilized by Bernier \cite{23413} to create prisms with fibers aligned at a particular orientation. The anisotropic distribution of fiber orientation was also confirmed by non-destructive electrical resistivity measurements on two slabs with different casting procedures \cite{23401}. The flow directions matched the pattern of the measured anisotropic axis. In the previous uniaxial stress and strain relations, the adverse fiber orientation is usually accounted for by introducing a global reduction factor that decreases the post-crack strength. However, this method is not a direct link between the level of anisotropy and the mechanical responses, and can underestimate the adverse effects if the fibers are highly aligned.

         \subsubsection{Shear deformation and shear failure}
% Last updated @ [2010/10/5]
%
% edited Mackie 2010/10/10 10:55am
%
% Discuss about the shear deformations and general way to consider this effect 
Although shear-induced deformation is usually neglected in flexural design of normal reinforced concrete members with small shear span ratio, it is relevant in the design of passively-reinforced UHP-FRC beams without shear reinforcement. In this type of beam, the longitudinal rebar is very effective in increasing the flexural strength, while it can only contribute to the shear resistance after developing significant diagonal cracks. Considering only the flexural response not only over-estimates the load capacity of the beam but also indicates an  unrealistic failure mode. One possible method for considering shear in flexural design is to add additional deformations due to shear to the flexural deformation. This process is usually performed without considering the shear and flexural interaction; however, this correlation directly effects peak strength and is considered in this research.


        \subsection{Available uniaxial numerical models from Literature}

            \subsubsection{French model and its simplified form}
                %% ====text====
% Last updated by Jun @ [2010/10/22]
% Last updated @ [2010/10/5]
% Discuss about the French model
The stress-strain curve of UHP-FRC for the strength limit state in the French code is plotted in Fig.~\ref{model_french} as well as the simplified model used for reliability analysis of UHP-FRC flexural members \cite{35002}. For the original French code model, the elastic cracking strain $\varepsilon _e $ can be calculated from $\varepsilon _e  = f_t /E$. The typical stress-strain value after cracking is dependent on the geometry of the flexural member and can be expressed as $\varepsilon _w  = \varepsilon _e  + \frac{w}{{l_c }}$, in which the strain subscript 'w' may assume values of "0.3","1\%", and "lim", representing the crack width of 0.3 mm, 1\% of the total section height $h$, and one quarter of the fiber length $l_f$, respectively.
The characteristic length $l_c$ can be estimated as two thirds of the total section height. Based on the previous definition, it can be seen that $\varepsilon _{1\% }$ is geometric size independent and has an approximate value of 0.015 based on the expression of $l_c$. The curve may degraded to
a tri-linear shape as noted by Ricardo \cite{83008} when the section height exceeds a critical limit and cause $\varepsilon _{1\% } $ to be larger than $\varepsilon _{\lim }$. The critical section height corresponding to the curve degradation is $h_{critical}  = 317.5$ mm if the fiber length is 12.7 mm. The basis for the French model is the stress-cracking opening relation, which can be obtained and simulated based on experiment results. However, the results directly from the flexural test were usually need some corrections based on the geometry, as shown by Reeves \cite{83017} on processing the results on the 76 mm by 102 mm (3 in. by 4 in.) un-notched prisms. The correction takes into account the size effects and make sure the results from flexural test are comparable to those from the uniaxial direct tension tests.

Eric \cite{35002} proposed the simplified stress-strain model, which was used to perform reliability analysis on UHP-FRC specimens. In his model, a constant stress response was introduced in the medium crack width region between the crack width of 0.3 mm and 1\% of the section height. The nominal values used in the simplified model are listed in Table~\ref{para_french}, which cause a strain softening effect after cracking, as shown in Fig.~\ref{model_french}. 
                   %% ====figure====

                %% ====table====

                \subsubsection{MIT think model}
                %% ====text====
% Last updated by Jun @ [2010/10/22]
% Last updated @ [2010/10/5]
% Discuss about the French model
By treating UHP-FRC as a composite material in macro sense, a spring and friction model \cite{83008,83009}, named the think model, was developed. In one dimensional situation, the equivalent mechanical system, the stress-strain relation for the matrix, fibers, and the composite material are shown in Fig.~\ref{Fig_model_MIT} \cite{22018}. The points on the stress-strain curve of the composite material can be calculated via the six parameters using the following Eq.~\ref{Model_para_MIT} \cite{22018}.



There are total six characteristic parameters used in this model, three of them represent the elastic properties and other three affect the inelastic behavior. The idealized parameters, and typical values used in literature for heat treated UHP-FRC are listed in Table~\ref{Table_para_MIT}. If using these parameters, the shape of the stress-strain curve is very similar to that from the simplified French model. The only difference is that the think model exhibits a 0.7 MPa kink at the onset of cracking. 
                %% ====figure====

                %% ====table====

                \subsubsection{Simplified stress-strain curve with perfect plastic responses}
                %% ====text====
% Last updated @ [2010/10/5]
% Discuss about the simplified elastic-perfect plastic model
A simplified uniaxial stress-strain curve was developed by Graybeal \cite{22008} based on experimental observations from the material test results and tests on prestressed UHP-FRC girders. It has also been used to predict the structural responses of the waffle shape bridge decks \cite{22014}. The model is much simpler than the previously mentioned models by adopting an elastic portion and a perfectly-plastic portion start at stress level of 9 MPa up to the ultimate tensile strain at 0.007. 
    \section{UNIAXIAL MODEL IN OPENSEES}
        \subsection{Material model in OpenSees}
            \subsubsection{Model setup}
            %% ====text====
% Last updated by Jun @ [2010/10/22]
% Last updated @ [2010/10/5]
% Discuss about the modeling of UHP-FRC in OpenSees
The Open System for Earthquake Engineering Simulation (OpenSees) is an open source finite element software package. It allows the user to define the customized material constitutive model in C++ environment, thus is selected as the tool for the analytical work presented in this research. The generalize uniaxial stress-strain material model was built based on the numerical model existed in the OpenSees package for simulating the engineering cementitious composites (ECC). The numerical model was based on the work of Han \cite{25506}. Two additional branches were added to the model on tension and compression sides in order to have the ability to simulated UHP-FRC. The expansion allows the model to be fully defined by five stress-strain points and the ultimate compressive and tensile strains. The uniaxial stress-strain relation is shown in Fig.~\ref{Fig_opensees_mode} with the parameters denoted. The unloading exponential ratio was set as unity for a linear unloading path. Different sets of parameters can be used to represent different types of uniaxial tensile models. 

% Last updated @ [2010/10/5]
%
% edited Mackie 2010/10/10 11:16am
%
% Discuss about the modeling of tensile reinforcement in OpenSees
The uniaxial stress-strain relationship for steel reinforcement was taken from the existing library of material models in OpenSees. The normal steel rebar with a yielding stress of 414 MPa was modeled using the elastic-perfectly plastic (EPP) material model. The Steel02 material model was used to simulate the response of the high strength steel (HSS) rebar with yielding stress as high as 1140 MPa. The experimental uniaxial stress-strain curves for several HSS bars are plotted in Fig.~\ref{Fig_stress_strain_MMFX} along with the numerical model prediction and values of the parameters used in the Steel02 model.                 %% ====figure====


            \subsubsection{Model parameters calibrations-compressive responses}

% Last updated @ [2010/10/5]
%
% edited Mackie 2010/10/10 11:25am
%
% Discuss about the calibrating the compressive side
From tests results reported in Appendix I of the French code \cite{22005}, an average 228 MPa compressive strength was obtained from 197 7 cm diameter, 14 cm long cylinders with both ends ground. The modulus of elasticity was in the range of 57 GPa to 61.4 GPa. The stress-strain curve showed an almost linear behavior until failure. This highly linear stress-strain relation on the compressive side was also observed by Graybeal \cite{22009} who performed a series of compressive tests on cylinders and cubes with different dimensions. The difference of compressive strength due to geometry of the specimen is within 8\%. The different curing regime was found to have a great impact on the compressive strength. The 28 day compressive strength of the 76 mm diameter cylinder with both ends ground are around 190 MPa and 120 MPa for the steam treated specimen and untreated specimens, respectively \cite{22070}. The relation between compressive strength and modulus of elasticity was also established based on the results from both the heat treated and untreated cylinders tests.

Based on the experiment results, a linear compressive stress-strain curve with an ultimate stress plateau is adopted. The resistance factor was not included to better match the real experimental results. The relation between the proposed model parameters and compressive strength are shown in Eq.~\ref{para_comp}.


In which, $E_c$ is the modulus of elasticity and can be directly related to the compressive strength based on $E_c = 46200\sqrt {f'_c}$ \cite{22070}, with both $E_c$ and $f'_c$ in `psi' units. By using constant $\varepsilon _{c1}$ and $\varepsilon _{c2}$ values, the model on the compressive side is only related to the concrete compressive strength $f'_c$.
                %% ====figure====


            \subsubsection{Model parameters calibrations-tensile responses}
% Last updated @ [2010/10/5]
%
% edited Mackie 2010/10/10 11:25am
%
% Discuss about the model calibrating the tension side
The tensile response of UHP-FRC is much more complicated due to the cracking of the cement matrix and the strain hardening post-crack responses. The material stress-strain relation before cracking can be fully calibrated by defining the modulus of elasticity and first crack train of the matrix. The modulus for tension is usually the same as the compressive modulus. The first crack strength of UHP-FRC was investigated by Graybeal \cite{22009} through several different test methods. The results shown an average first cracking strength of UHP-FRC at about 9 MPa. while in French code \cite{22005}, the value used is 8.1 MPa based on the prism flexural test.

After matrix cracking, the stress-crack opening relation is usually of interest and can be obtained by crack width measurements. Regarding the type of specimens, the notched prisms are usually used for both the uniaxial tensile test as well as the flexural test in order to concentrate the post-crack responses within the notched region. Chanvillard \cite{22026} performed both tests using the same size prisms, and the results are shown in Fig.~\ref{Fig_exp_tensile} in grey color. The results obtained from the flexural test can be corrected based on the uniaxial test result to make sure the same stress level at the same crack width. In this paper, the stress versus crack width relation is expressed in Eq.~\ref{s_w_relation} with the general expression from the curve fitting results as shown in Fig.~\ref{Fig_exp_tensile} and calibrated parameter $\gamma$  = 3.3 based on the uniaxial tensile test results at crack width equal to 0.3 mm.

Within the seven parameters in the generalized model shown in Fig.~\ref{Fig_opensees_mode}, $\sigma _{t0}$ and $ \varepsilon _{t0}$ are determined by the first crack strength and the rest are post-crack related and will be determined based on particular material constitutive models.
                %% ====figure====

            \subsubsection{Realization of literature available uniaxial stress-strain relation}

% Last updated @ [2010/10/5]
%
% edited Mackie 2010/10/10 11:29am
%
% Discuss about how the model reflecting the existing model available in literature

% Jun, just be consistent throughout, the generalized model may be misleading, suggest you call it the proposed model or some such thing.  Some people may view existing literature models as being generalized.
The post-crack responses of the size dependent French model can be realized using the generalize material model by changing the parameters with respect to the section height as shown in Eq.~\ref{para_French}. The subscripts of `0.3', `1 \%', and `lim', represent the crack widths of 0.3 mm, 1\% of the total section height $h$, and one half of the fiber length, respectively. The characteristic length $l_c$ is usually equal to two-thirds of the total section height $h$.

By using the stress-crack width relation shown in Eq.~\ref{s_w_relation} and deterministic values for $E_t$ and $f_t$, the section height only has an effect on $\varepsilon _{t1}$,$\sigma _{t2}$, and $\varepsilon _{t3}$. The stress-strain curves are shown in Fig.~\ref{Fig_French_Tension} with various section heights. Beside the size-dependent French model, three size-independent models were also introduced as shown in Fig.~\ref{Fig_model_independent}. The parameters used in the generalized model to represent the three models are shown in Table~\ref{Tabel_sizeindep_para}. Based on the shape of the three proposed models, the names 'Hardening-Plastic (HP)', 'Hardening-Softening (HS)', and 'Elastic-Plastic (EP)' were given to these three models.                 %% ====figure====



        \subsection{Special considerations}
            \subsubsection{Effects form different curing methods}
% Last updated @ [2010/10/5]
%
% edited Mackie 2010/10/10 11:55am
%
% Discuss about how the different affect the material
Usually, heat treatment at 90 degrees Celsius and 95\% humidity is applied to the material right after initial setting and shall continue in a stable thermal condition for 48 hours to fully developed the strength of the UHP-FRC. However, the compressive tests done by Graybeal \cite{22009} show that the same heat treatment applied to the specimens at two weeks after the casting has almost the same effects as the standard heat treatment. The experimental results also confirmed the relation between modulus of elasticity and compressive strength, which is valid for both treated and untreated specimens. Thus, the heat treatment impact can be reflect by the increased compressive strength and can be related to the modulus of elasticity. The typical curves for treated and untreated UHP-FRC used in this research are shown in Fig.~\ref{model_treated_untreated}. It is assumed in this research that the different curing methods have no impact on the compressive ductility due to the lack of experimental data.

Different curing methods are also expected to have an impact on the tensile modulus of elasticity as well as the first crack strength. About 30 percent difference was reported for the first crack strength between heat treated and untreated materials from experimental results based on several different direct tension tests \cite{22005}. In this paper, point estimations were made on the first crack strength with respect to different curing conditions because the complete correlation between $f_t$ and $f'_c$ is not available. For standard heat treated and untreated material, the first crack strengths were set as 9 MPa and 6.2 MPa, respectively. Regarding the post-crack responses, the experimental load versus displacement curves from the flexural tests on various prisms with and without heat curing are very close to each other, the difference is no greater than the variation of the results with the same curing method, which means that the curing method has only limited impact on the tensile post-crack response. This is a reasonable assumption considering that the post-crack strength is largely dependent on the bond strength between fibers and cement matrix, and the different curing methods do not have direct impact on the interfacial properties. The same post crack responses were used for materials with and without heat curing. As an illustration, the tensile stress-strain relations of the size independent 'Hardening-Plastic' model is shown in Fig.~\ref{model_treated_untreated} for the two typical curing conditions. 
                %% ====figure====

            \subsubsection{Effects from anisotropic fiber orientation distributions}
% Last updated @ [2010/10/5]
%
% edited Mackie 2010/10/10 12:13pm
%
% Discuss about orientations
While the assumption of uniform spatial distribution of fibers is usually achievable due to the moderate 2\% fiber volume faction and can be ensured by the appropriate mixing process, the distribution of fiber orientation is hard to control and inevitably affected by the casting method and flow directions of the cement paste. In the French code \cite{22009}, this effect of adverse fiber orientation distribution was considered by introducing a reduction factor $K$ to reduce the post-crack strength at all strain levels, and $K$ equals 1.25 for common load cases. This method is insensitive to the level of anisotropy and aims to provide a conservative estimation for design purposes.  From experimental results, it is concluded that the fiber orientation distribution not only reduces the post-crack strength, but also affects the first crack strength. In this paper, the first crack stress and stress level at a crack width of 0.3 mm were varied with respect to the fiber orientation distribution and can be determined by the following equations.



In which $f_{t,m}$=6.9 MPa represents the tensile strength of the matrix and $f_{t,f}$=2 MPa is the tensile strength of the composite contributed by all the fibers. The two factors $\eta _1$ and $\eta _2$ are shown in Eq~\ref{eq_eta_1} and Eq~\ref{eq_eta_2}.



In which, parameter $\kappa $ represents the degree of alignment with respect to the principal axis. The range of parameter $\kappa $ is zero to one, with $\kappa$=1 representing the case of uniform distribution and $\kappa$=0 representing all fibers aligned to a specific direction. Angle $\varphi $ denotes the direction of the cut section and equals the angle between the normal vector of the cut plane and the alignment axis. The preference parameter $\kappa $ can be estimated by counting the number of crossing fibers $N_s$ at a particular section cut and then back calculating the $\kappa$ value using Eq~\ref{eq_orient_kappa_case2_2}. Parameter $\alpha (\kappa ,\varphi )$ is the orientation factor and equals to $N_s A_f/V_f$ with $A_f$ representing the individual fiber cross-sectional area and $V_f$ is the fiber volume fraction.



            \subsubsection{Consideration of shear deformations}
% Last updated by Jun @ [2010/10/22]
% Last updated @ [2010/10/5]
%
% edited Mackie 2010/10/10 12:19pm
%
% Discuss about how the different affect from shear
In OpenSees, by using the aggregated section model and force-based nonlinear beam column element, the shear deformation can be included in the total deformation by specify the sectional shear rigidity $GA$. However, there is no interaction between the flexural and shear responses. To consider the concrete cracking under a two dimensional normal and shear stress state, an external program
%written in scripting language
was created to trigger a single step analysis in OpenSees and to keep the shear rigidity updated after each displacement increment.
At the element level, the total shear force demand is calculated based on the elastic analysis of the structure and imposed on individual integration points based on their locations. The sectional shear rigidity was determined based on the shear force demand as well as the normal stress levels. The shear rigidity for all integration point along the beam were set as changeable parameters and were updated after each displacement increment if necessary. At the section level, the shear force demand was distributed to all sub-layers to achieve force equilibrium and the shear rigidity for the un-cracked section and residual shear stress for the cracked section were collected. The net shear force was calculated by subtracting the total shear force by the residual shear resistance from those cracked sub-layers, and then distributed proportionally to the remaining sub-layers based on $GA$. This distribution process requires iteration and may eventually fail to achieve equilibrium, which denotes shear failure has occurred in the region where the integration point exists.

At the material level, the maximum allowable shear stress was calculate by Eq.~\ref{eq_ta_allow} based on the normal stress obtained from the OpenSees output file, and the graphical representation is shown in Fig.~\ref{Fig_ta_allowable}. Based on the relation between the shear stress demand and the maximum allowable shear stress, different sub-layer status can be determined as shown on Fig.~\ref{Fig_ta_allowable}. The shear modulus $G$ and stress output are listed in Table~\ref{Table_ta_allow}. On the tension side, $\tau _{\max }$ decreases when the tensile normal stress increases, while on the compression side, $\tau _{\max }$ can increase with the increasing compressive strength. Therefore, zero $G$ value was assumed for tensile cracked sections, and a reduced $G$ value was assumed for compressive side after the shear stress reaching $\tau_{max}$, the expression of this reduced $G$ value is shown in Eq.~\ref{eq_G_compre}, in which $\tau _{\max ,f'_c }$ is the ultimate shear stress the material can provide when the compressive stress reaching $f'_c$. The whole analysis process is shown in Fig.~\ref{model_sheardesign_flow} with the explanation of the procedure at three different scale levels. 







                %% ====figure====




    \subsection{Analytical investigation}
            \subsubsection{Impact from different post-crack models}

% Last updated by Jun @ [2010/10/5]
% Last updated @ [2010/10/5]
% Discuss about how the different post-crack model affect the global load displacement responses
The global sectional responses from different post-crack models were compared for both the un-reinforced and the passively reinforced sections.
The verification sections are all rectangular with unit width and varying section height. The ultimate moment and corresponding curvature for each section are summarized in the following Table~\ref{t_res_unreinforced}. Two typical moment curvature relations are plot in Fig.~\ref{res_unreinforced}. The labels represent the statue when reaching the specific tensile strain values at the extreme fibers. For all unreinforced sections, the compressive stress at the peak moment are all within the linear range. It is clear from the analytical results that for unreinforced sections, different post-crack models have considerable impact on the sectional moment capacity and corresponding curvature. Results from different post-crack models match well with the size dependent model at different section height ranges due to size effects. For cases of sections height less than 200 mm, which is the common range for bridge deck applications, the difference between 'Hardening-Plastic' model results and size dependent French model results are within 10\% regarding the maximum sectional capacity, which means the size effect can be neglected at this height range if 'Hardening-Plastic' model is used.

Similar sectional analysis were also carried out on the reinforced rectangular sections. Three typical sections with height less than 200 mm were selected, and various reinforcement ratios from 1\% to 3\% were applied to the section. Normal strength steel and high strength steel rebars were used and the results are shown in Table~\ref{Table_res_reinforced_EPP} and Table~\ref{Table_res_reinforced_MMFX}, respectively. For the group reinforced with normal steel rebar, it was found that the differences between all models are mostly less than 10\%. And the 'Hardening-Plastic' model is still the best model when compared to the size dependent model. For the group reinforced with high strength steel rebar, it was found that except for the 'Elastic-Plastic' model, which is sometimes too conservative, both 'Hardening-Plastic' model and 'Hardening-Softening' model produce reasonable estimates when compared with the size dependent French model. The results from the two groups also reveal that the higher percentage of the tensile reinforcement used, the less influence of UHP-FRC post-crack strength will have on the overall sectional moment capacity. Use of high strength steel reinforcement with nonlinear stress-strain relations will fully utilize the UHP-FRC material when comparing Fig.~\ref{Fig_res_reinforced_MMFX} with Fig.~\ref{Fig_res_reinforced_EPP}. All the analytical results were based on the assumptions that the perfect bond exists between the concrete and the rebars, under which the fibers were still within the strain hardening range when both rebars reach strain level of 0.002. This explains the reason why the responses are the same for the two hardening models up to the yielding of the reinforcement. 







            %\subsubsection{Impact form different levels of ultimate tensile strain}
                %Need to add
        \subsection{Experimental verifications}

            \subsubsection{Unreinforced UHP-FRC prisms}
% Last updated @ [2010/10/5]
%
% edited Mackie 2010/10/10 12:28pm
%
% Discuss about the experiment verification on the unreinforced prisms, literature results
Some literature results were used in this paper for model verification purposes. Several groups of unreinforced prisms with and without heat treatment were tested \cite{22009} with different specimen geometry and loading configurations. The results were compared to the simulation results based on the 'Hardening-Plastic' material model for treated and untreated specimens. Uniform spatial and orientation distribution were assumed for this comparison.

% This is the prisms with fiber alignment k=0
The influence of form boundaries on fiber alignment was utilized by Bernier \cite{23413} to create prisms with fibers aligned at a particular orientation. A plate was cast by temporarily placing parallel thin metal sheets along the flow direction with distance equal to the fiber length during the casting. After heat curing, several prisms were cut from that plate with different angles with respect to the flow direction. These prisms are ideally corresponding to the model with $\kappa $=0, and various $\varphi $ values. The experiment results on these prisms were used to verify the uniaxial stress-strain relation considering the fiber anisotropic orientation distributions. 
% Last updated @ [2010/10/5]
% Discuss about the experiment verification of unreinforced prisms
The comparison between literature results and the analytical results from the proposed model is shown in Fig.~\ref{Fig_res_unreinforced_lit}. The top solid lines in each comparison groups represents the analytical results for treated prisms. It is concluded from the comparison that the curing condition do not affect the responses for unreinforced prisms and the analytical results are close to the experiment results and on the conservative side.

The analytical results for prisms with fiber alignment at certain orientations were compared to the experiment results as shown in Fig.~\ref{Fig_res_orient}. Although not perfectly matched, the proposed model exhibited the strain-hardening and strain softening responses based on different degree of alignment similar to the response patterns observed from the experiment. 



            \subsubsection{Reinforced UHP-FRC prisms}

% Last updated @ [2010/10/5]
%
% edited Mackie 2010/10/10 12:28pm
%
% Discuss about descriptions on experiment verification on reinforced prism
Two groups of passively-reinforced prisms were cast to verify the proposed model. Each group contains three prisms. The dimensions and loading configurations are shown in Fig.~\ref{res_reinforced_prism_setting}. All prisms were reinforced with US No 3 high strength steel rebars and the rebars were bent at the end into 90 degree end hooks to prevent the anchorage failure.

% Jun, I think we should leave the FRP out of the paper, fine to include for dissertation
Other than the steel reinforcement, three UHP-FRC prisms from the same batch were used to investigate the effectiveness of externally-bonded CFRP reinforcing. While one prism was leave as is without any strengthening, the other two were strengthen with wet-layup bidirectional woven CFRP at the bottom. One of the two reinforced prisms also has shear reinforcement at both sides as shown in Fig.~\ref{prism_frp}. 


% Last updated @ [2010/10/5]
% Discuss about descriptions on experiment verification on reinforced prism
The experiment results as well as the analytical results using the proposed model are shown in Fig.~\ref{res_reinforced_prism}, because the prisms were not heat treated, the typical untreated stress-strain model was used. The CFRP reinforced concrete prisms were also untreated and the comparisons to experiment results are shown in Fig.~\ref{res_FRP_prism}. It was found that the proposed relation can predict the load displacement curve close to those obtained from experiment with respect to different reinforcement types. From the results on the CFRP reinforced prisms, the impact of shear influence is obvious. The prisms strengthen with side CFRP layer has a higher stiffness and higher load capacity.




            \subsubsection{Passively Reinforced full scale deck strips}
% Last updated @ [2010/10/5]
%
% edited Mackie 2010/10/10 12:35pm
%
% Discuss about descriptions on experiment verification on T section beams
A passively-reinforced T-section deck strip was tentatively designed for use as the unit structural component of a newly developed moveable bridge deck system. The deck was made of UHP-FRC and bottom steel reinforcement. The dimensions for two T-section deck strips are shown in Fig.~\ref{Fig_Tsection_sec}. The single unit deck strip was tested with a 1219 mm simply supported span and loaded with the neoprene pad placed at the middle 508 mm region to simulate the tire pressure. The load versus displacement as well as the rebar strain were obtained from experiment. 

% Last updated @ [2010/10/5]
% Discuss about results on experiment verification on T section beams
It was found from the load versus displacement curve that the experimental stiffness and its degradation was between the two simulated loading scenarios, the uniformly distributed load within the loading pad region and the two point loadings at the edges as shown in Fig.~\ref{Model_Tsec}. The real load distribution is depended on the relative stiffness of the neoprene pad and the flexural rigidity of the deck strip itself. The compressive stiffness of the neoprene pad was obtaining by testing two 100 mm by 100 mm pads and the pressure versus deformation curves are shown in Fig.~\ref{res_neoprene}. A simplified model was used to take into account the pad effects as shown in Fig.~\ref{Model_Tsec}. The stiffness of three compression only springs were estimated based on the experiment results and equals to 1.36 MPa/mm based on the contributed area of the pads. The results for the three load cases are compared as shown in Fig.~\ref{Fig_res_Tsec_pad}, it was found that due to the high stiffness of the loading pad, the load distribution situation is more close to the point loading at the pad edges. 


% Last updated @ [2010/10/22]
% Last updated @ [2010/10/5]
% Discuss about results on experiment verification on T section beams include shear
Based on the model with neoprene pad, the load versus displacement responses including the shear influence is obtained as shown in Fig.~\ref{Fig_res_Tsec_shear}, the section status at particular load level was also shown in Fig.~\ref{Fig_res_Tsec_shear}, It was found that although the shear deformation is not that much compared to the flexural, however, they governed the maximum load capacity.



    \section{CONCLUSIONS AND DISCUSSIONS}
% Last updated @ [2010/10/5]
% Discuss about results on experiment verification on T section beams include shear

% about general model and size effect
The generalized uniaxial model for UHP-FRC was built in OpenSees. With different numerical values of the parameters, the model in the literature can be compared under the situation with same elastic properties and different post-crack relations. It was found that size effect can be neglected when the section height is less than 200 mm and if appropriate model is used. The post-crack stress-strain relation has considerable impact on the global responses of un-reinforced member. However, this impact can be neglected when analyzing the passively reinforced flexural members due to the fact that the steel reinforced is the control factor for the tensile force. The use of high strength steel rebar can fully utilized the ultra-high strength of UHP-FRC without adopting high reinforcement ratio, which is more economical and practical than using the normal steel. Based on the analytical results, the Simple model was proposed to use in the future analysis and its effectiveness was verified based on the experiment results. 

% about curing
As the different curing methods only greatly affect the first crack strength, its impact on the tension controlled flexural member can be neglected. However, as the heat treatment greatly enhanced the compressive strength. The compression controlled flexural member, for example, highly reinforced beams, will be greatly affected by the curing methods.       


% about the orientation
By link the stress levels of Simple model to the fiber orientation distribution status. The anisotropic level and direction can be easily considered in to the uniaxial flexural analysis. The uniformity parameter and direction of anisotropy can be obtained via experimental analysis. The results shows the general trend of the responses due to anisotropy that is similar to those based on the experiment results. However, more verification is needed at more wide scope with respect to different levels and directions of the anisotropy. 


% About the shear
It should be noticed that the shear-flexural interaction in this paper has only one way effect, which means only the impact of normal stress on the shear resistance is considered. However, in real situation, after the concrete cracks due to shear induced stress, the contribution to the flexural resistance will also be affected. This effect is neglected in this paper due to the fact that the steel reinforced is the main tensile member and the reduction of flexural due to shear can be neglected. Further more, perfect bond is assumed between the concrete and rebar, therefore, the possible affects of the local debonding is not considered. Bond failure can be critical, because the lose of bond will greatly reduced the compressive strength in the UHP-FRC locally and thus cause the shear failure due to quick loose of shear resist capacity at that particular location. \clearpage
\newpage
\bibliographystyle{elsarticle-num}
\bibliography{../../lib/UHPC}




% ========== check Chapter 4

% Last updated @ [2010/10/7]
% To do list
% 1 Add load demand estimation, moment target based on the AASHTO LRFD code (previous calculation is LFD)
% 2 Update the preliminary design  engine to opensees and add probability analysis to replace sensitive analysis
% 3 Add steel strain versus load plot for all specimens
% 4 Add ordinary Fem RESULTS
% 5 Add results on the interface model
% 6 upate Fig_FEM_Model_wiredisplay to a color mode
\chapter{APPLICATION ON MOVEABLE BRIDGE DECKS}

    \section{PROJECT INTRODUCTION}
        \subsection{Background and objective}
            %% ====text====
% Last updated by Jun @ [2010/10/24]
% Last updated @ [2010/10/7]
% Discuss about the Porject initiation and background
According to the National Bridge Inventory (NBI), there are 137 bascule movable bridges in Florida and 469 national wide. Most of these moveable bridges utilize open grid steel deck systems that have several shortcomings: the riding surface is less skid-resistant when wet; traffic-induced vibration causes noise and sensations of poor ridership \cite{41001}; and the steel deck is corrosion- and damage-prone and costly to maintain. Thus a passively-reinforced ultra-high performance concrete (UHP-FRC) deck system was proposed as an alternative to these decks \cite{Paper-2}. The combination of UHP-FRC and high strength steel (HSS) rebar provides a solution with a light-weight, high-strength deck system that was experimentally proven to meet the crucial selfweight and strength requirements necessary in moveable bridge applications.

Deck systems for moveable bridges have stringent acceptance criteria. The total height of the deck section should be around 127 mm. The selfweight of the deck panel should be less than 1.2 ${\rm{kN/m}}^{\rm{2}}$. The typical deck panel should be compatible for installation on steel girders that are commonly 1219 mm apart. The deck system should be able to resist the AASHTO LRFD \cite{AASHTO} HS-20 truck load, and the maximum deflection should be less than 1/1000 of the span (typically 1.2 mm) under service load levels according to AASHTO LRFD 9.5.2 for bridges with limited pedestrian traffic. The deflection control of the deck itself is to prevent the breaking up of the wearing surface. For newly developed deck systems, it is recommended that this deflection limit is verified experimentally. Several light-weight deck systems made of fiber-reinforced plastic (FRP) \cite{44001} or aluminum \cite{Paper-1} were reported to meet these requirements, while their field application are still under investigation.

%Several UHPC products are commercially available worldwide, and Ductal{\textregistered} is one widely recognized in the US. The material properties of
%Ductal{\textregistered} were fully investigated by Graybeal \cite{22009}, with the high strength and good durability confirmed based on the experimental results. Compressive strengths as high as 221 MPa are ensured by applying a manufacturer-recommended heat treatment process, otherwise lower strengths will result. The tensile strength was experimentally investigated by Chanvillard \cite{22026} using both uniaxial and four-point bending tests. The ultimate tensile strength was found to be around 10.8 MPa, and a scale factor was introduced for adjusting the flexural test results. The fibers blended in the UHPC matrix provide a bridging effect across the micro cracks and thus increase the tensile strength and ductility. The punching shear resistance of the UHPC slab was investigated by Harris \cite{22003}, and the minimum deck thickness to prevent the punching shear failure under the factored wheel load patch is predicted to be 25 mm. Ductal{\textregistered} has also been used in several bridge applications in the US \cite{08-13-1} and in Europe \cite{22021}.

        \subsection{Review on the available high strength steel reinforcement}
            %% ====text====
% Last updated @ [2010/10/6]
% Discuss about the available reinforcement
%% stainless steel rebar
%The stainless steel reinforcement is used in the normal concrete to prevent potential corrosion \cite{26008}. It has a yielding stress as high as 75 ksi, good ductility and corrosion resistance to atmospheric and can be useful in other less severe environments.
%Bar diameters could go from No 3 to No 10. The strain-stress relation of stainless steel is come from /cite{26001}, a report from The University of Sydney.
%
%
%in which, $\sigma _{0.2}$ is 0.2\% strain stress,$\sigma _{0.01}$ is 0.01\% strain stress (here just assume only 70\% of the $\sigma _{0.2} $,
%$E_0 $ is initial Young's modulus and 193GPa $28*10^6 psi$. $E_{0.2}  = \frac{{E_0 }}{{1 + 0.002n/e}}$ soften stiffness
%      in which $n = \frac{{\ln (20)}}{{\ln (\sigma _{0.2} /\sigma _{0.01} )}}$,$e = \frac{{\sigma _{0.2} }}{{E_0 }}$,$m = 1 + 3.5\frac{{\sigma _{0.2} }}{{\sigma _u }}$
%
%Expression for ultimate stress and strain (Difficult to get in the coupon test, thus, use best fit formula based on 'n','e' value ) $\frac{{\sigma _{0.2} }}{{\sigma _u }} = 0.2 + 185e$ (only apply to austenitic and duplex alloys), based on table in \cite{26002} use 84 ksi $\varepsilon _u  = 1 - \frac{{\sigma _{0.2} }}{{\sigma _u }}$


%% Here start to talk about MMFX rebar
%Microcomposite steel rebar (MMFX2) is an uncoated, high strength, high corrosion resistant rebar made from a low-carbon, chromium alloy steel \cite{26104} as show in Figure 5. The mechanical properties tests were conducted by NCSU in 2002. The typical stress and strain curve is shown in Figure 5 \cite{26111}. The mechanical test results meet the requirement of ASTM A615 Grade 75 \cite{ASTM-A615} and ASTM \cite{ASTM-A1035} Grade 100. The photo and typical uniaxial stress and strain relation of MMFX2 rebar is shown in Fig.~\ref{Fig_intro_MMFX}. The MMFX2 rebar was treated as a promising replacement of the normal carbon rebar in the bridge structure because of its superior corrosion resistant which was confirmed by the accelerated lab corrosion tests \cite{26102}, \cite{26106}. Meanwhile the bond strength and splice of MMFX2 rebar were investigated by Raafat \cite {27004} and Marcus \cite{27014} respectively. The same conclusions were the MMFX2 has similar bond behavior with convention concrete as normal carbon steel rebar, but due to its high strength, more bond strength needed to fully yield the bar and thus utilize its high strength. A longer splice length, and longer extension of the hook were required if the high strength of MMFX2 rebar is considered in the design. Besides the nonlinear behavior of MMFX2 rebar after yielding would weak the bond strength and cause bond failure.
%The MMFX2 rebar already been used in the field applications, The Iowa and Kentucky Department of Transportation have used MMFX rebar as reinforcement in bridge decks, and Delaware Department of Transportation was in consideration using MMFX2 rebar in I-95 Service Road Bridge 1-712-B in 2005 \cite{26101}.
%The experiment on using the MMFX2 rebar in bridge deck was performed by Hatem \cite{26108}. Three two-span double-cantilever decks were tested under wheel load to failure. The first and third specimens were reinforced by MMFX2 rebars with the same and 2/3 reinforcement ratio respectively compared to the second specimen which was reinforced by Grade 60steel rebars. The second and third deck specimens showed similar load-deflection behavior and the first deck reach higher load level. The replacement of carbon steel reinforcement with 67\%-100\% ratio of MMFX2 rebar was considered suitable if the minimum reinforcement ratio meet the requirement of AASHTO specification.


Stainless steel rebars with yield stresses up to 690 MPa are also commercially available \cite{26009}. Their superior corrosion resistance makes them the best choice for deck applications, except that their cost is usually several times higher than normal-strength rebar. High-strength microcomposite steel rebar (MMFX2) is an uncoated, high strength rebar made from a low-carbon, chromium alloy steel with a yielding stress of 690 MPa and ultimate strength as high as 1200 MPa as shown in Fig.~\ref{Fig_intro_MMFX}. According to Cleme\~{n}a \cite{26106}, MMFX2 has better corrosion resistance than normal black steel rebar but is not as good as several other types of stainless steel rebar. The research focused only on the mechanical behavior of the UHP-FRC and HSS composite system; therefore, MMXF2 rebars were selected as a typical high strength Grade 100 rebar. They were embedded in the UHP-FRC as longitudinal reinforcement. Usually no shear reinforcement is needed for the beam due to the high tensile strength of UHP-FRC with the embedded steel fibers; however, several shear transfer mechanisms were investigated in this paper, including the addition of transverse reinforcement. 
            %% ====table====

            %% ====figure====


    \section{PRELIMINARY DESIGN AND OPTIMIZATIONS}
        %\subsection{Estimation of load demand}



        \subsection{Preliminary section selection}
            \subsubsection{General concept and selfweight limit}

                %% ====text====
% Last updated @ [2010/10/6]
% Discuss about the deck system concept
The conceptual drawing of the two-way waffle deck is shown in Fig.~\ref{Fig_preliminary_section}. The two reference directions as well as the typical parameters for both directions were specified. The four important parameters of the deck section in the transverse direction are: the width of the flange $b_e$ (the spacing of the transverse webs); the thickness of the flange $h_f$ , (the thickness of the slab); the height of the webs $h
_w$; The width of the webs $b_w$. the weight limit of the deck is 122 kg/m$^2$ (25 psf), and the unit weight of the UHP-FRC and MMFX2 steel are 2400 kg/m$^3$ (150 lb/ft$^3$) and 7800 kg/m$^3$ (487 lb/ft$^3$) respectively.
For the top UHP-FRC slab thickness, the maximum is 50.8 mm (2 in.) due to selfweight restraint while the minimum set as 25.4 mm (1 in.) to avoid punching shear failure \cite{22003}. In order to find the most appropriate T sections, total 420 trial sections were created. The reinforcement was selected based on the width of webs to adopt the largest reinforcement fit to the web. Spacing of the transverse webs were adopted form half in. to 381 mm (15 in.) The height of the webs varied form 76 mm (3 in.) to 114 mm (4.5 in.). The secondary web remains 38 mm (1.5 in.) wide and 44.5 mm (1.75 in.) deep. The self weights of all these 420 sections were calculated including the weight of steel reinforcement, within which 60 sections were selected with weight up to 141 kg/m$^2$ (29 psf) to go through the next selection stage.



                %% ====figure====

            \subsubsection{Preliminary section based on positive moment capacity}
                %% ====text====
% Last updated by Jun @ [2010/10/23]
% Last updated @ [2010/10/6]
% Discuss about the available reinforcement
The moment capacities per foot as well as the deck dimensions were shown Fig.~\ref{Fig_preliminary_sensitive} in the way that each dot represents particular section result. The color of the dot represents the moment capacity in unit of `kip-in' (1 kip-in=0.113 kN-m). While the size of the dot represents the thickness of the top UHP-FRC slab. The X,Y,Z axles represent the spacing of T unit, the thickness of web and total height, respectively. The most promising section was selected based on the allowable dimension of the deck due to selfweight limits and punching shear demand as shown in Fig.~\ref{Fig_preliminary_finaldimension}. The negative reinforcement were selected based on the minimum cover and provides sufficient negative moment capacity.

                %% ====figure====

                %% ====figure====


    \section{EXPERIMENT EXPLORATION AND VERIFICATION OF FEM MODELS}
        \subsection{Experiment work and discussion about the test results}
% Last updated by Jun @ [2010/10/24]
% Last updated @ [2010/10/7]
% Discuss about the experiment matrix
In order to verify the structural response of the preliminary design, a series of tests were designed as shown in Table~\ref{table_deck_test_matrix} at several investigation levels. The material characterization tests were performed to confirm the mechanical properties and to investigate the interface between the high strength reinforcement and UHP-FRC. The compressive tests results were also used as the material quality control purpose for each casting. At the component level, four deck component types with single or multi-unit were tested as simply supported or two span continuous beams. In order to identified the specimens, the name label 'nTmS' were used to denote the particular specimen with 'n' typical unit and tested with 'm' loading span with the unit span length of 1219 mm (48 in.). The detailed experiment design, specimen casting methods, and test procedure are documented somewhere else. The drawings of the instrumentation plan is shown in Fig.~\ref{Fig_Tsection_instrumentation} to locate the corresponding experiment results. 

            %% ====table====


            %% ====figure====


% Last updated @ [2010/10/7]
% Discuss about the experiment results
The load versus displacement curves are shown in Fig~\ref{Fig_Tsection_loaddisp}. The load capacity of the final 3T2S specimen exceeded the load demand and thus treated as structural sufficient. From the load versus displacement curves, it was found that the stiffness of 1T2S specimen is obviously less than that of 1T1S specimen, which means the negative bending region is actually more critical. The failure modes of all specimens are shown in Fig~\ref{Fig_Tsection_failuremode}, most of the specimens failed due to openings of shear cracks or two way punching shear cracks. From the load versus middle span rebar strain plot shown in Fig~\ref{Fig_Tsection_loaddisp}, it is obvious that the reinforcement at the primary web direction for the 4T1S specimen is not fully utilized. The strain value is less than the yielding strain. While the specimen 1T1S is actually not appropriately tested due to the twisting deformation observed during the experiment. 
            %% ====figure====

            %% ====figure====

        \subsection{FEM results and comparison to the experiment results}
            \subsubsection{General information about the finite element analysis}
% Last updated by Jun @ [2010/10/24]
% Last updated @ [2010/10/7]
% Discuss about the general information about the fem analysis
The finite element analysis were performed using MSC.Marc$^{\textregistered}$, a software package with nonlinear analysis abilities. The three dimensional models using 8 node hex elements were built for all type of specimens are shown in Fig.~\ref{Fig_FEM_Model_wiredisplay}. The loading and boundary conditions were in such a way that they matches the real boundary conditions in experiment. The rebars were modeled with the truss elements and share the nodes with UHP-FRC element to simulate the perfect bond. UHP-FRC is modeled with low-tension material model on the tensile side and elastic plastic material on the compression side. The stress-strain curve used in the analysis as well as the corresponding parameters were shown in Fig.~\ref{Fig_FEM_low_tension} and Table~\ref{table_FEM_para}.

Cracking is judged by comparing the principal stress with predefined first crack stress. After Cracking, the material was treated as orthotropic material and the stress at principal direction will decrease with cracking strain increase until the residual stress went to zero. The dropping process was controlled by the softening modulus. This cracking mechanism was applied to all integration points. After the stiffness went zero, the element lose rigidity at that integration point.


            \subsubsection{FEM results on one span specimens}
% Last updated by Jun @ [2010/10/24]
The simulation results, including the crack pattern, load versus displacement curve, and load versus strain curves for 1T1S and 4T1S specimen are shown in Fig.~\ref{Fig_FEM_res_1T1S} and Fig.~\ref{Fig_FEM_res_4T1S}, respectively. Both models catches the initial stiffness and general strain responses under load conditions. However, due to the fact that the rebar is perfectly bonded with the concrete, both models overestimated the load capacity of the specimens and did not catch the load redistribution during the loading after the opening of the shear cracks. Based on the destructive image analysis applied on the specimens, except for the location where the shear crack developed, the rebar and UHP-FRC still at perfectly bond condition after failure. The coupled local debonding between rebar and concrete, as well as the decrease of compressive strength of concrete, thus the shear resistance are the main reasons for the shear failure.  


            \subsubsection{FEM results on two span specimens}
% Last updated by Jun @ [2010/10/24]
The simulation results, including the crack pattern, load versus displacement curve, and load versus strain curves for 1T2S and 3T2S specimen are shown in Fig.~\ref{Fig_FEM_res_1T2S} and Fig.~\ref{Fig_FEM_res_3T2S}, respectively. This set of simulation results have larger discrepancy than the previous group because of the responses at internal support. At this critical location, not only the moment is the largest along the span, there is also high shear force demand if considering the continuation of the deck panel. The reinforcement at negative moment region was designed based on the moment capacity, however, their shear resistance are not verified and greatly weakened due to the crack at the top surface near the rebar. The concrete web in compression at internal support can not provide enough shear resistance and thus the moment transfer at this location will diminish and cause the load redistribution to the middle span. This responses is very obvious in the load versus displacement curve shown in Fig.~\ref{Fig_FEM_res_3T2S}, where the stiffness degradation is obvious. On the other hand, at peak load, the US {\#}4 rebar in the transverse web approached the ultimate tensile stress, while the stress in the longitudinal rebar still less than yielding stress. This observation revealed the final failure of the 3T2S specimen is because of the transverse connection between the deck unit rather than the load capacity of individual unit.   

    \section{SHEAR FAILURE MODE AND DESIGN EQUATIONS}

        \subsection{Introduction}
            %% ====text====
% Last updated by Jun @ [2010/10/24]
% Last updated @ [2010/10/7]
% Discuss about the introduction on shear design
Shear strength of the beam is usually checked based on the geometry and reinforcement ratio determined by the flexural design. Although quite a few shear design methodologies are available \cite{27602,27603,22005}, their applications to UHP-FRC reinforced beams are all limited by the maximum applicable compressive strength of the concrete and thus need verification based on experimental results. One available shear strength design formula
is from the French code \cite{22005} and has been used to predict the shear strength of the pre-stressing bridge girders \cite{22008}. The formula includes several terms to represent the contribution from the UHP-FRC compression zone, distributed fibers in the matrix bridging the cracks, stirrups, and bent rebar, respectively. Dowel action contribution is not included in the formula but it is believed by Reineck and He \cite{27504,27505} to be an important shear transfer mechanism in the beam, especially those beams without shear reinforcement.

        \subsection{Methods: Experimental investigation}
            %% ====text====
% Last updated by Jun @ [2010/10/24]
Experimental investigations were performed at a variety of length scales to better characterize material properties and behavior under load. Four categories of tests were performed:
material characterization, bond strength, small-scale beams, and full-scale beams. The specimen matrix is summarized in Table~\ref{table_test_matrix}. All of the specimens mentioned in the table were not heat treated; however, the influence of lack of heat treatment are discussed with the experimental results.

The cylinder compression tests were based on the ASTM standard C39 \cite{ASTM-C39} using the universal testing machine (UTM). The cylinders from batch 2 and 3 were tested at the Florida Department of Transportation's Structural Materials Laboratory with fine end grinding, while the ones from batch 1 were tested in the University of Central Florida laboratory with limited surface treatment. The uniaxial tension test of UHP-FRC on the dog-bone coupon specimens was carried out on an MTS 50kN UTM machine. The rebar uniaxial test was carried out following the ASTM-E8 testing method. Detailed procedures and results of the material characterization tests were reported
elsewhere \cite{Paper-2}, and the corresponding properties are summarized in this paper.

The compression bond specimens were made by casting UHP-FRC around MMFX2 rebar in a PVC pipe. Rigid foam was used to stop the UHP-FRC flow and small diameter PVC tubes were used to maintain the designated bond length. The specimen sketch is shown in Fig.~\ref{Fig_compout_setup}. The specimens were loaded in the UTM with a steel cap and neoprene pad on top of the rebar portion extending out from the specimens. The bond length and the cover size are summarized in Table~\ref{table_bond_cover}. The load and table movement were recorded during the test.
Two small-scale reinforced prisms were cast for preliminary experimentation. Both of them were reinforced with one US {\#}3 MMFX2 rebar. In the first specimen, the rebar was bent to form a 90 degree hook at both ends while the other one was cast with a straight bar protruding from the specimen ends. The bottom and side covers for both specimens were set equal to the diameter of the rebar itself, which is approximately 10 mm. Three strain gauges were attached to each prism with the locations shown in Fig.~\ref{Fig_flex_setup}. Two strain gauges were used on the rebar to capture the strain variation along the prism, while the third gage was mounted on the top surface of the concrete in the mid-span. Deflections on both sides of the prisms were measured by linear potential meters that were mounted to an aluminum reference frame fixed between the supports. The four-point loading fixture is shown in Fig.~\ref{Fig_flex_setup}. 
            %% ====table====
            %table 1

            %table 2

            %% ====figure====



            %% ====text====
The full-scale beam specimen design drawings with specific section dimensions are shown in Fig.~\ref{Fig_Tsection_sec_2}. The deck strip was treated as a simply-supported beam reacting on the steel girders that are typically spaced 1219 mm apart on moveable bridges. The total length of the beam is 1372 mm allowing for construction of the end anchorage. The test results for the single-span single-unit (1T1S) deck strip without any end anchorage and with the 180-degree hook anchorage were reported by Saleem et al \cite{Paper-2}. Because both specimens failed with shear cracks rather than flexural cracks, four more deck strip specimens were cast to investigate the efficacy of several shear strengthening mechanisms. To allow for comparison, the details for all six specimens are summarized in Table~\ref{table_Tsec_matrix}. The loading configuration and the instrumentation plan were similar for all specimens and are shown in Fig.~\ref{Fig_Tsection_instru}. Special rebar layouts for those using multiply longitudinal rebars are also shown in Fig.~\ref{Fig_Tsection_instru}. All full-scale experiments were conducted at the Titan America Structures and Construction Testing Laboratory of the Florida International University. 


            %table 3


        \subsection{Methods: Analytical investigation}
            %% ====text====
The ultimate shear strength formula from the French code \cite{22005} is $V_u=V_{Rb} +V_a +V_f $. The contribution from concrete can be expressed as,

in which $k$ is the factor that considers the prestressing effects within the concrete and is equal to one for passively-reinforced beams. All formulas in this paper adopt `MPa' and `mm' as basic units except those terms specified explicitly. The contribution from the reinforcing fibers is,



in which $\sigma_P $ is the average post crack strength. The parameter $S$ is the shear resistance area that equals $0.9b_0 d{\rm{ }}$ for rectangular section and equals $b_0 z$ for T-sections. The parameter $z$ is the distance between the tensile and compressive resultant forces. For preliminary analysis, the design factors $\gamma _E ,\gamma _b ,\gamma _{bf}$ were set to 1 and the average of the post-crack strength $\sigma _p$ was estimated using the lower and upper bounds of 7.7MPa and 10.3 MPa, respectively. $V_a$ refers to the contribution from transverse stirrups or bent rebars. It equals to zero if none of them exists.

The `tooth model' summarized by Reineck \cite{27504} allow the estimation of the shear strength based on the interfacial bond strength between concrete and rebar. The model divided the beam into segments connected only at top compression region as shown in Fig.~\ref{Fig_model_tooth}. The width of the segments is determined by the crack spacing $S_{cr}$. The shear strength was derived by analyzing the moment equilibrium of the segment, resulting in equation $\Delta T \cdot z = V \cdot S_{cr}$. The crack width can be estimated as $S_{cr}  = 0.7(d - c)$, in which $d$ is the structure depth, $c$ is the height of the un-cracked portion, and the term $
\Delta T = \tau \pi d_s S_{cr}$ is related to the interfacial bond stress. The shear strength limited by the interfacial bond strength can be expressed as,



            %% ====figure====


            %% ====text====
% Last updated by Jun @ [2010/10/24]
which is now independent of the crack space $S_{cr}$. Based on the normal and shear stress distribution at the un-cracked section, the shear force contribution from the concrete part is,


Because the value of $c/z$ is usually small, the shear strength contribution from the concrete compression zone is only a small portion of the total, therefore, the friction force along the crack and the dowel action force are important contributors to the shear strength.

The strut and tie model of the beam with no shear reinforcement is also represented in Fig.~\ref{Fig_model_tooth}. According to Reineck \cite{27504}, if there is enough interfacial bond stress between the rebar and concrete, a concrete tension tie will form between points B and C. If shear reinforcement exists (shear stirrups or bent rebars), it will help dissipate the tension force as well. The dowel action of the longitudinal rebar is an important secondary mechanism once the concrete tension tie fails. From the experiment, it was seen that after the shear crack opens, the rebar portion at the crack location was forced to bend instead of shear and lead a rotation of the cracked section as shown in Fig.~\ref{Fig_model_struttie}. This non-traditional dowel action is actually one of the primary load transfer mechanisms at the stage close to failure of the UHPC-HSS beam. This is especially obvious on the specimen without end anchorage of the longitudinal rebar. This beam failed with excessive localized bending of the longitudinal rebar at the cracked section and the concrete around the rebar was split apart due to the dowel action force. A strut and tie model was developed based on this observation and it is show in Fig.~\ref{Fig_model_struttie}. Two extreme cases can be represented by the same model. When the $\theta _d$ angle equals zero, Force $F_{c2}$ equals zero as well, and all load is transferred through the concrete compression zone. When the $\theta _c$ angle equals zero, dowel force $F_{c2}$ equals the full external shear force $V$, and only a portion of the load is transferred through the top compression region (technically, load transferred equals to $F_{c3}$. Between the two extreme cases, the load is distributed between the two mechanisms. 
            %% ====figure====

            %% ====text====
% Last updated by Jun @ [2010/10/24]


From geometry,

If it is assumed the support cannot provide any lateral resistance, then the shear strength $V_u $ can be estimated as follows using the stress in the rebar by assuming $\cos (\theta _d )$ is unity.

The results include effects from both load transfer mechanisms. Although the shear strength is actually determined by the compressive strength of the concrete and the yielding/ultimate stress of the rebar, the final achievable rebar stress reflects both restraints.

Muttoni \cite{27602} estimated the ultimate shear strength dependent on the maximum shear crack width that equals $\varepsilon  \cdot d$
, in which $d$ is the structure depth and $\varepsilon$ is the longitudinal strain measured at depth of $0.6d$. This assumption related the external moment with the shear strength of the section. The equation can be generalized in the form of,



The unit of parameter $\beta$ is `1/mm' to achieve the unit compatibility. For normal concrete according to Muttoni \cite{27602}, the values for these parameters are $\alpha =$ 3 and $\beta  = 22.5/{\rm{mm}}$. Parameter $\phi$ is the curvature calculated based on the external moment as,



Parameter $c$ is derived from the equilibrium of the axial force by assuming a linear distribution of the stresses within the concrete compression zone and leads to,



From Eq.8, it is clear that the shear strength of the section without the influence of bending should be



        \subsection{Results: Experimental}
            \subsubsection{Compression out specimen}
            %% ====text====
% Last updated by Jun @ [2010/10/24]
The concrete properties of UHP-FRC for three separate batches are shown in Table~\ref{table_UHPC_prop}. The relatively low compressive strength for batch 1 was due to the fact that the specimens were tested without perfect end grinding. The bond test results are summarized in Table~\ref{table_bond_res}. All specimens failed due to longitudinal concrete splitting as shown in Fig.~\ref{Fig_comp_failmode}. Only the specimen shown in Fig.~\ref{Fig_comp_failmode} was tested with weakened bond sections and thus ended up with lower than the average bond strength.
The bond between UHP-FRC and MMFX2 is good even with minimal cover, as evident from the test results. The rebar stress exceeded the 827 MPa yielding stress for the US {\#}3 rebar with bond length as low as eight times the bar diameter. The good bond between UHP-FRC and MMFX2 rebar is due to the material properties of the highly-dense UHP-FRC matrix \cite{23103} and is actually an important factor in the behavior of the passively-reinforced beams. The compression bond test caused compressive stress in the UHP-FRC as well as in the rebar. The bond strength in the real beam may be lower than the experimental results. The experiment results on US {\#}3 rebars shown that the average bond strengths for different bond lengths are very close, thus the cracks may have more impact on the global bond stiffness rather than the ultimate local bond strength. More experiments are needed to obtain the complete understanding of the bond between the two materials, especially tests with both materials loaded under tensile stress. 
                %% ====table====
                %table 4

                %table 5

                %% ====figure====


            \subsubsection{Small prisms}
                %% ====text====
The test results on the two small-scale reinforced prisms are shown in Fig.~\ref{Fig_prism_resplot}. The average height and width of the no anchorage prism is 63.8 mm and 39.4 mm, respectively, and for the hooked end prism the values are 69.0 mm and 40.6 mm. The cover for both specimens was approximately 9.5 mm, which equals the diameter of the rebar. The slight difference in the dimensions explains the different initial slope of the load versus deflection curves. The rebar strains near support (SG1) behave significantly differently before and after the load level around 12 kN for both specimens as shown in Fig.~\ref{Fig_prism_resplot}. The no-anchorage prism failed due to bond failure and thus the crack completely opened. The rebar bent locally at crack location, which caused the strain SG1 on top of the rebar became negative due to compression. The specimen with a 90 degree hook did not have one primary crack, rather several distributed small cracks developed. The strain reading (SG1) for this specimen keeps increasing with degraded stiffness. The final failure modes of the two specimens are shown in Fig.~\ref{Fig_prism_failure}.
In order to confirm the good bond between UHP-FRC and MMFX2 rebar, the prisms were cut transversely along the length after failure. The section from the no-anchorage prism that is 25 mm away from the shear crack was shown in Fig.~\ref{Fig_prism_failure_end}. The rebar was tightly bonded with the concrete and no sign of bond failure exists. The 90-degree hook specimen showed the same good bond at the middle section while the bond failure was found at the anchorage region as shown in Fig.~\ref{Fig_prism_failure_end} as well. Because the middle portion of the prism with 90-degree hook still remained intact, a 140 mm long middle portion was cut from the hooked specimen and tested with a smaller shear span of 127 mm. The ultimate load was approximately 53.4 kN, and the strain versus load curves are shown in Fig.~\ref{Fig_prism_smallspan}. It is evident that the moment in the longer-span specimen affected the ultimate shear strength. 
                %% ====figure====




            \subsubsection{T section specimens}
                %% ====text====
The small specimens demonstrated good bond between UHP-FRC and MMFX2 rebar with cover thickness equal to the diameter of the rebar. Therefore, preliminary analysis of the full-scale deck strip were performed based on the perfect bond assumption, and the deck section dimensions and reinforcement ratio were determined. Two full-scale deck strip specimens were cast and tested. The results of these two tests were reported by Saleem et al \cite{Paper-2} along with the system-level experimental results. Several additional one-span
specimens were tested for shear strength improvement. Detailed information regarding specimen configurations is summarized in Table~\ref{table_Tsec_matrix}.

Test results for the specimens utilizing US {\#}7 rebar as longitudinal reinforcement are shown in Fig.~\ref{Fig_tsec_resplot} and Fig.~\ref{Fig_tsec_failmode}. Test results for the rest of the specimens are shown in Fig.~\ref{Fig_tsec_strainplot} and Fig.~\ref{Fig_tsec_failmode_shear}. All specimens failed with diagonal shear cracks that propagated to the support region, but the crack widths and strain responses are different for each specimen. 
                %% ====table====
                % table 12

                % table 13

                % table 14

                % table 15

                %% ====text====
The load versus displacement curves for the specimens reinforced with US No 7 rebar have the similar initial slope and diverged at higher load level due to the different rebar type and end anchorage conditions. All three specimens with MMFX2 rebar showed a gradual transition point on the SG1 plot of Fig.~\ref{Fig_tsec_resplot}, while the transition is more abrupt for the specimen with normal Grade 60 rebar. Strain readings of SG1 on the top surface of concrete between the support and the loading point reflect the load transfer ratio between concrete compression zone and the dowel action. The change of the slope means the redistribution of the load. The transition point occurred much earlier in the specimen using two US No 4 rebars and was not clearly exhibited in the specimens with bent up rebars. The shear cracks were observed to appear initially at load levels that correspond directly to these transition points (zone) during the experiments. The structural responses at this particular load level and at the ultimate load level are summarized in Table~\ref{table_tsec_failtype} for all six specimens. The post behaviors after the transition point were different for the three specimens with similar 180-degree hook. While the two without transverse reinforcement exhibited stiffness degradation of SG1 curve in Fig.~\ref{Fig_tsec_resplot}, the one with stirrups continued with approximately the same slope. 

                %table 6

                %% ====text====
% Last updated by Jun @ [2010/10/24]

Both specimens with small flexural reinforcement ratio failed with shear crack. Although the flexural cracks were visible during the test, they closed along with the opening of the diagonal shear cracks. Therefore, reducing the flexural reinforcement ratio did not change the failure mode. The high bond strength and the relatively small cover ensured that the flexural crack cannot widen without total interfacial bond failure. Due to the loading configuration, the interfacial shear stress is relatively small at the middle span and thus, the only chance for the flexural crack to widen is when the rebar starts to yield and the lateral shrinkage of rebar breaks the local bond mechanism.

The strut and tie model discussed previously was used to interpret the beam response that is notably different from the normal strength rebar reinforcing concrete. The good bond between the UHP-FRC and HSS rebar enables the shear transfer mechanism by the tension tie in concrete. Once the tensile stress exceeded the crack strength of UHP-FRC, diagonal shear cracks will appear. The shear force will be transferred and carried by the dowel action of the rebar. Additional load on the rebar causes a stress increase and the local bond failure. The rebar is then forced to bend locally at the unbounded region. The transition point/region is related to the nonlinear response of the rebar surpassing the yielding point. The smooth nonlinear response of the MMFX2 rebar after yielding is the reason for the gradual load redistribution (degradation of the SG1 slope) while the normal strength rebar has a plastic plateau on stress and strain curve and thus an abrupt transition point exhibited.

Although the dowel action can produce high load capacity due to the superior material properties of the MMFX2 rebar and UHP-FRC, it involves excessive section rotation and beam deflection. Dowel action may also cause lateral force to the supporting girders if the reinforcement is mechanically anchored (to the superstructure), so this failure mode is not desirable for the design of bridge decks.

Three different failure types were classified based on the strain response on the top of the concrete surface. Type I failure is defined as when the SG1 value is close to the crush strain of the concrete. Type II failure is defined as when the concrete strain at mid-span reaches the crush strain prior to SG1, which only happened on the specimen strengthened with shear stirrups. Type III failure is defined as when the SG1 value is larger than the crushing strain of UHP-FRC at failure, which means almost all the vertical load is transferred through dowel action. The failure types for all six specimens were listed in Table~\ref{table_tsec_failtype}. The heat treatment will increase the compressive strength of UHP-FRC and thus will increase the shear strength for those specimens that failed because of concrete crushing (Type I and II). The positive impact from the heat treatment on the bond strength is also expected as the heat treatment will increase the tensile strength and the concrete splitting is the only failure mode observed in the bond tests.

        \subsection{Results: Analytical}
            %% ====text====
% Last updated by Jun @ [2010/10/24]
The shear resistances for the small-scale and full-scale beam specimens were estimated based on Eq.1-2. The results are summarized in
Table~\ref{table_tsec_shearstrength}. Although the average fiber contribution is difficult to calculate, the lower and upper bound estimation predict the right range of the shear strength for all specimens. In the case of the T-section beam, 80{\%} of the flange width was used to count for an
estimated shear lag effect. The external moment and the flexural reinforcement ratio are not taken into consideration in the formula from the French code. However, these two parameters were important for predicting the shear strength based on the experiment results.


Shear strength estimation regarding the interfacial stress limits was based on Eq. 3-4. The maximum bond strengths from the bond compression test were used as the stress limits. If assuming parameter $z$ equals to$0.9d$, the estimated shear strengths were 36 kN and 117 kN, respectively for the small-scale prism and the full-scale T-section beam specimens. Both values are higher than the experiment results, showing the interfacial bond strength between the cracks is not the reason for the final shear failure.

The shear strength based on Eq. 7 by assuming the $\theta _c $ angle equals to zero is listed in Table~\ref{table_tsec_estimation}. For small-scale prism specimens, the rebar stresses were calculated based on the rebar strain measured directly during the test. For full-scale T-section beam specimens, the rebar stresses were estimated based on the strain results of the rebar at mid-span. For those specimens without any recorded rebar strain, yielding stress of the specific longitudinal reinforcement was used in the calculation. Most of the estimated shear strengths are close to the test results, except for the small-scale prism with small shear span. The comparison showed that the strut and tie model can represent the load transfer at the final stage of the specimens with big shear span ($d/L_1  < 5$).

The ultimate shear strength is also estimated using Eq.8. The recommended values for parameters $\alpha $ and $\beta$ predicted conservative results. Therefore, parameter $\alpha $ and $\beta$ for use on the UHP-FRC beam were calibrated based on the small prism test results. The shear strength without flexural influence is 27 kN and the shear strength under a $2.54 \times 10^6 {\rm{N}} \cdot {\rm{mm}}$ moment is 20 kN. The two parameters were determined to be 1 and 0.3/mm, respectively. The moment-shear interaction curves for both groups of parameters are plotted in Fig.~\ref{Fig_moment_shear} for comparison. The strength ratio is shown on the ordinate and is simply the value of $V_u /\left( {bd\sqrt {f_c ^\prime  } } \right)$. The fitted formula underestimated the results for the 1T1S specimen with 1 US No 7 rebar and overestimated the capacity of the specimen reinforced with 2 US No 4 rebar. More full-scale specimens with different shear spans need to be tested to further calibrate the parameters used in this method.

            %% ====table====
            %table 7

            %table 8


            %% ====figure====


        \subsection{FEM analysis with interface modeling }
            %% ====text====
% Last updated by Jun @ [2010/10/24]
Based on the investigation of the shear failure mode, it is clear that the interface between UHP-FRC and the rebar is critical. The local bond failure and plastic flexural deformation of the rebar shall be able to reflect in the model in order to catch the realistic failure mode. Therefore, the finite element model utilizing 3D solid element to represent the rebar is created as shown in Fig.~\ref{Fig_FEM_interface_model}. The critical part of this type of simulation is to setup the contact scheme or create the interface layer element. In Marc$^{\textregistered}$, the contact is handled in such a way that no additional elements are needed between the contact surface. The positions of the nodes on the two contacting surfaces will be checked during the increment/iteration, and the normal and shear stress can be calculated based on the contact status, which is either glued together, slipping in the tangential direction, or totally separated. The "glue" to "slip" scheme is used in this research, and the tangential shear stress was choose to be the slipping criteria. Once the tangential shear stress exceed the limits, the rebar will start to slip in concrete body with the friction force dependent on the normal stress and friction coefficient. In order to simplified the analysis, a small friction force is specified by applying a small number of the friction coefficient. The influence of the loading condition is also taken into account by adding two layers of solid elements on top of the specimen to represent the steel plate and neoprene pad.  
            %% ====figure====

% Last updated by Jun @ [2010/10/24]
The load versus displacement from the finite element analysis considering the interface is shown in Fig.~\ref{Fig_FEM_inteface_res}. The model with perfect glue connections and the glue-slipping responses are compared to the test results. Due to the fact that the shear stress is theoretically uniform in the shear span due to the two point loading configurations, the rebar-concrete interface de-glued almost at the same time and cause the load drop as shown on the load versus displacement curve. The crack greatly widen neared the loading edge due to the fact that pure concrete section can not hold the moment and shear force. However, due to the elastic end joint between the rebar and the supports, the load can continue increase until the rebar reach the ultimate tensile strength. Through the interface model did not give exact load capacity and failure mode, it reveal the realistic interaction between the rebar and the concrete, and shown that the widen of cracks will not concentrate at certain location until the bond start to break locally. 

\clearpage
\newpage
\bibliographystyle{elsarticle-num}
\bibliography{../../lib/UHPC} 
% ========== check Chapter 5

\chapter{SYSTEM LEVEL ANALYSIS}

    \section{AVAILABLE ALTERNATIVES AND CORRESPONDING MODELING EFFORTS}

        \subsection{Steel open grid decks}
            %% ====text====
% updated by Jun @ [2010/10/19]
% Last updated @ [2010/10/8]
% Discuss about the open grid steel decks
Besides the UHP-FRC deck system that are designed to be implemented on the moveable bridges, there are several other alternatives. The unit open grid decks are commonly 127 mm(5 in.) deep, 2.3 m (90 in.) square panel. It has two major lattice patterns, the diagonal pattern and rectangular pattern. Partially or fully concrete-filled grid decks are sometimes used to increase the rigidity of the deck panels and to provide a solid riding surface. Due to the selfweight limit, only open grid steel decks without concrete filling can be used on moveable bridges. The structural responses of the open grid steel deck have been investigated by Huang \cite{41002}. A standard unit open grid steel deck panel was tested with two opposite edges simply supported. The deck was loaded with a 508 mm (20 in.) by 101 mm (4 in.) neoprene pad at the geometric center of the deck to simulate the wheel load patch. The strains and deflections on various locations at the load level of 70 kN (10 kips) were summarized \cite{41002}, and these results were compared with beam element-based finite element results. In order to model this deck for varying overall dimensions and boundary conditions, a plate finite element model was built in MSC.Marc using 3D four node thin shell elements. The dimensions of all components are summarized in Table~\ref{table_steel_grid_comp}. The thickness of the main bar elements are different along the height to reflect its real shape and the correct location of the neutral axis under bending. The illustration of the finite element model is shown in Fig.~\ref{fig_steel_grid_model}. All nodes at the same location were merged to represent the welded connections. Concentrated loads were applied on the geometric center of the model to simulate the applied load during the experiment. The distribution of the load greatly affects the local strain response of the deck portion that is directly under the load point. Therefore, a line load of 381 mm (15 in.) wide was used and it shown the best correlation with experimental results. The strain values at the top and bottom of the main bars with locations at the center and at the quarter of the span were compared to the reported experiment results as shown in Fig.~\ref{fig_FEM_model_res}. Based on the calibrated full scale deck models, the responses of the component level deck model can be simulated as shown in Fig.~\ref{Fig_opengrid_component}. 
            %% ====table====

            %% ====figure====




        \subsection{Aluminum deck system}
            %% ====text====
% updated by Jun @ [2010/10/19]
The first aluminum deck system was built in 1933 in Pittsburgh. They were reinvestigated in late 1990s and applied in the field in Europe \cite{Conf_lingtweight-18}. Its application in US was limited due to the concerns on the corrosion and wear surface issues, and most of the applications in US were limited to the pedestrian bridges. The SAPA Corporation in Sweden has a market available deck system, which has been used in the traffic bridges in Sweden. The 280 mm wide Aluminum decks can be placed side by side on the supporting girders and can be connected together with girders using aluminum mechanical clamps. The fatigue and failure behavior performed in the FDOT structure lab confirmed the deck behavior under 2 million traffic load cycles and the residual strength of the deck system were still exceed the AASHTO requirement \cite{Paper-1}. The experiment results from several component level deck tests were used in the model calibration discussion. 
            %% ====figure====




        \subsection{GFRP deck system}
            %% ====text====
% updated by Jun @ [2010/10/19]
% Last updated @ [2010/10/8]
% Discuss about the open grid steel decks
The GFRP composite deck is another promising deck systems that meets the selfweight requirement. One typical GFRP deck system made by Zellcom is shown in Fig.~\ref{Fig_deck_FRP}. It has been deeply investigated and was implemented in Florida. The deck featured the prefabricated bottom deck panel with four T shape vertical webs. The deck panels can be placed side by side longitudinally with edge lips overlapped. The top GFRP plate with the wearing surface can then be assembled to the bottom panels mechanically with screws. The laboratory fatigue and failure tests indicator a good load capacity and ductility of this deck system \cite{44001}. However, one of the draw backs of the GFRP deck system is their high initial cost. 
            %% ====figure====


    \section{SYSTEM LEVEL ANALYSIS, FRAME BASED MODEL CALIBRATION}
        \subsection{General information about the system level models}
% updated by Jun @ [2010/10/19]
% Last updated @ [2010/10/8]
% Discuss about the general information about the system level model
The purpose of creating the system level finite element model is to utilize the component level test results to predict the global behavior of the deck system regarding the serviceability and load distribution. Usually the grillage method was used to predict the system level response under any loading conditions \cite{Edmund}. The live load distribution factor for shear and moment for interior and exterior girders can be estimated for all possible wheel load placements \cite{NCHRP_592,36003} if the the unit influence surface for all degree of freedom of all nodes are available. The simplified load distribution equations were usually drew based on the analysis results to convert a 3D loading configuration to a two-dimensional problem, therefore, the supporting girders can be designed. The simplified equations for open steel grid deck from AAHSTO LRFD code is shown in Table~\ref{table_DF_AASHTO}. The moment distribution factor for the moveable bridge using W24x68 steel girders with 1219 mm (48 in.) girder spacing is about 0.533 and 0.5 for one lane load and two or more lanes, respectively. The other factors can be acquired using lever rule, which assuming the hinge connections at the interior span of the transverse deck strips at the girder location. The lever rule forms a simplified while conservative load distribution scenario in the direction perpendicular to the traffic direction thus the load distribution factors can be easily determined. 




However, the traditional grillage method is usually used on the concrete deck system that are isotropic. For moveable bridge applications, due to the stringer selfweight limit, the deck systems are mostly orthotropic with strong direction perpendicular to the traffic direction. Further more, in the traditional modeling approaches, the deck was discretized without the physical meaning of the deck dimensions. There is a also a lack of controls on the deck-to-deck, and deck-to-girder joints of the system, which is very important to the global behaviors.

The model was built with typical deck unit in order to easily represent various deck systems by simply change the parameters that calibrated from component level deck tests. The typical model unit uses 2-node displacement based beam elements and end rotational springs as shown in Fig.~\ref{Fig_system_typical_unit}. Only the flexural deformation of the unit deck is directly considered by using appropriate flexural rigidity of the frame elements. The additional deformation due to shear or degradation of the section rigidity will be considered by apply a reduction factor to the flexural second moment of inertia and this reduction factor is calibrated by the single unit, single span test results. In order to consider the asymmetric flexural rigidity of the deck unit at the negative moment region, a spring with constant $Kd_{in}$ is used in the model and was calibrated by the multi-span test results. Regarding the interaction between the neighboring deck units, the deck unit rotational stiffness and the stiffness of the cross links are the two main influence factors. The combined effects of these two factors can be represent by a flexural only link member with appropriate second moment of inertia $I_{yy,link}$. This parameters can be calibrated based on the multi-unit test results. Spring $K_{dg}$ is used to represent various types of the deck to girder connections thus the different degree of moment transfer and/or the vertical separation due to uplifting at the deck corner can be considered. The steel girders were modeled as elastic beam elements with the ability to deform in shear and torsion.

The system level FEM tool was built in the way that the deck unit can be assembled with assigned connection properties. Therefore, several typical deck units can be connected transversely and longitudinally in order to represent the precast decks that usually equals to the lane width. While between the precast deck, the transverse and longitudinal deck joints can be represented by the spring $K_{f,dd,t}$ and $K_{f,dd,l}$. The connection between steel girder elements $K_{f,gg}$ is always rigid.

The system level analysis were carried out at different scale levels. The model was first calibrated based on the component level deck tests. After the determination of the model parameters, the precast decks made of the four deck systems can be created. A virtual bridge with a deck area of 10.7 m by 7.3 m was designed using these precast deck of different dimensions and panel layout. And then the unit influence surface can be obtained for these four deck system. The type of the connection between precast deck panels will be determined based on the connection details with respect to the specific deck type. Besides the connections mentioned, all the other joints between deck unit and precast decks are treated as a rigid connection with full moment transfer abilities.





            %% ====figure====


        \subsection{Model calibration}

            \subsubsection{Calculation of the sectional properties}
% updated by Jun @ [2010/10/19]
% Last updated @ [2010/10/12]
% Discuss about the calculation of the sectional properties of the four typical deck system

The flexural rigidity for all four deck systems were estimated based on the typical deck section shown in Fig.~\ref{Fig_model_calib_section}. The calculated section properties are shown in Table~\ref{table_model_calib_section_property}. The $I_{yy}$ value for UHP-FRC deck section was calculated based on the cracked section. The shear and the torsional deformation are considered indirectly with the appropriate spring constants. The parameters used for steel girders are also shown in Fig.~\ref{Fig_model_calib_section}, additional torsional and shear deformation are specified by setting $J$=1.87, $G=E/2/(1+\mu)$. For GFRP deck section, due to the low modulus of elasticity of the webs, only the top and bottom plate counted towards the flexural rigidity. 
                %% ====figure====

                %% ====table====


            \subsubsection{Model calibrations}
% updated by Jun @ [2010/10/19]
% Last updated @ [2010/10/12]
% Discuss about the calibration of the UHP-FRC deck system
The calibration of UHP-FRC deck system were based on the component level test results on specimen 1T1S, 1T2S, and 5T1S. The load versus deflection curves are shown in previous chapter. The displacement at load level of 100 kN (22.5 kip) were collected and shown in Table~\ref{table_calib_all}. The model used for the calibration are shown in Fig.~\ref{Fig_model_calib_UHPC}. The deck to girder connection was assumed to be hinge connections. The detailed calibration results as well as the parameters used are shown in Table~\ref{table_calib_all} along with the simulated results for comparison purpose.
                %% ====figure====

                %% ====table====

% Last updated @ [2010/10/12]
% Discuss about the calibration of steel grid decks
The component level results of steel grid deck were obtained from the calibrated plate based finite element model. The same type of component level results as UHP-FRC deck system were used as shown in Fig.~\ref{Fig_model_calib_UHPC}. The simulated component level test results are shown in Table~\ref{table_calib_all} with the results obtained from system level finite element model. 
% updated by Jun @ [2010/10/19]
% Last updated @ [2010/10/12]
% Discuss about the calibration on Aluminum deck system
The calibration on the aluminum deck system is based on the available experiment results as shown in Fig.~\ref{Fig_model_calib_Alum}. The experiment results are summarized in Table~\ref{table_calib_all} along with the calibrated parameters. In the two span specimen tests, the clamps were used and thus the rotation at the internal support is restraint, therefore, a fixed support was used for this set of results. The 3T2S test results were obtained from the deck residual strength tests after 2 million cycles fatigue load test, thus some degree of degradation was reflected in the results.                 %% ====figure====

% updated by Jun @ [2010/10/19]
% Last updated @ [2010/10/12]
% Discuss about the calibration of the GFRP deck system
The calibration on the GFRP deck system were based on the component level in the two directions. The results and corresponding model are shown in Fig.~\ref{Fig_model_calib_FRP}. It was found that the estimated section properties lead to a stiffer results when compare to the 1T1S test results. Considering the slip between the top and bottom plate due to the deformation of the screws and the shear deformation due to the webs. The experiment results are summarized in Table~\ref{table_calib_all} with the calibrated parameters.                   %% ====figure====


    \section{APPLICATION OF SYSTEM LEVEL MODEL}
        %\subsection{Sensitivity analysis}
        %    \subsubsection{Sensitivity analysis on the precast parameters}

        %    \subsubsection{Sensitivity analysis on the deck to deck connection}



        \subsection{Description of the demo bridge deck system}
% updated by Jun @ [2010/10/21]
% updated by Jun @ [2010/10/19]
All four bridge deck solutions were applied to the demo bridge with the deck dimensions as shown in Fig.~\ref{Fig_system_analysis_demo}. While the precast UHP-FRC deck can be as wide as the bridge lane width, the steel grid deck has typical dimensions around 2.3 m (90 in.). The aluminum deck and GFRP deck has nature deck width and were reflected in the placement scheme. Regarding the deck joints, the connection for aluminum deck and GFRP deck units are actually the internal joints that were previously calibrated, therefore, a rigid joint was assigned to the deck joints and the stiffness of the link member were used to represent the deck panel to panel interaction. For open grid steel deck, the deck panels were welded on the steel girder and the interlock between different deck panels were small thus the joint is treated as separated. For UHP-FRC deck, the detailed connection methods are still under investigation, thus hinged joint were assumed to be on the conservative side. The steel girders were set as fixed at both ends while the deck to girder connections were set as hinge for all four deck systems. A moving unit nodal load was placed at the center of the basic deck units and the vertical displacement as well as the internal shear and moment of all steel girder members were recorded. The maximum values with respect to each loading positions were extracted and plotted in a surface contour chart as shown in Fig.~\ref{Fig_sys_Surf_Dy}, Fig.~\ref{Fig_sys_Surf_M}, and Fig.~\ref{Fig_sys_Surf_V}.

It was found that the Aluminum deck system lead to the smallest steel girder responses regarding the vertical deflection and the maximum moment. The responses of UHP-FRC deck system is comparable to that of steel open grid deck. The GFRP deck system causes a larger girder deflections and causes a higher moment in the girder elements.  







\clearpage
\newpage
\bibliographystyle{elsarticle-num}
\bibliography{../../lib/UHPC}







% ========== check Chapter 6

\chapter{CONCLUSION, DISCUSSION, AND FUTURE RESEARCH DIRECTION}

\section{CONCLUSIONS AND DISCUSSION}
The problems on the deck system of the moveable bridges cause the bridge engineering community start looking for alternatives to the open steel grid decks. The solution using UHP-FRC material is very promising due to its high strength-weight ratio and superior durabilities. However, due to the production issues, the deck system can not be prestressed as commonly practiced when casting UHP-FRC bridge girders. Therefore, the passively reinforced UHP-FRC system with high strength steel rebars was investigated in this research, and the experimental and analytical results were used to estimate the system level responses of the moveable bridge that may utilize the type of deck in the future.



% for chapter 2
Ultra-high performance fiber reinforced concrete (UHP-FRC) has macro-mechanical properties, particularly ultimate strength, that depend on the type and alignment of steel fibers embedded in the cement matrix. This paper focuses on quantifying the anisotropic fiber orientation distribution. The estimation of fiber orientation factors for anisotropic cases weak-principal (WP) and strong-principal (SP) were derived as shown in Eq.~\ref{eq_orient_kappa_case1} and Eq.~\ref{eq_orient_kappa_case2} based on statistical derivations and numerical simulations. For case WP, the properties in the t-t plane are independent of the cut plane directions, so the orientation factor was only affected by the uniformity parameter $\kappa$. While for case SP, the properties in the p-t plane were of interest and the orientation factor is affected by the direction of cut plane represented by the parameter $\varphi$. Furthermore, the orientation distribution for the fibers having intersections with any particular plane was modeled. The general expression is shown in Eq.~\ref{eq_ft_fii} with parameters listed in Table ~\ref{Table_ft_cf_case1} and Table ~\ref{Table_ft_cf_case2}, which were based on the cut plane directions and level of anisotropy denoted by parameter $\kappa$. The equations of the orientation factors and PDF of crossing fiber orientations make it possible to estimate the properties of materials under certain assumptions. By using the material properties for a particular UHP-FRC, it was found that the elastic modulus does not change drastically with the anisotropic orientation distributions, while the ultimate tensile strength can change significantly with respect to the section considered and level of anisotropy. The paper also illustrated the method to quantify the uniformity parameter $\kappa$ based on image analysis of actual specimen fiber counts at perpendicular cut directions without considering the fiber orientation distribution details.

% For chapter 3

The uniaxial stress and strain are the basis for the section capacity based flexural member design procedure. The model available in the literature different form each other with regarding the post-crack responses as well as the consideration on the size dependent nature of concrete material. The influence of other factors, such as heat treatment, anisotropic fiber orientation distributions, and the influence form shear deformations also need to take into account. Based on the analytical work performed in this research, it was conclude that for general bridge applications, both the size effect and post-crack responses have only limited impact on the flexural responses when compared to the impact from using different types of tensile members. Use high strength reinforcement made of HSS will fully utilized the high compressive strength of UHP-FRC without causing an excessive reinforcement ratio. The proposed Hardening-Plastic material model is accurate enough when compared to the experiment results on reinforced prisms and beams. It is also complicated enough to represent the necessary influence factors as mentioned before. By adding the estimation scheme of the shear force demanding and resisting to the flexural analysis, the stiffness degradation and premature failure due to shear can be reflected.

% for chapter 4
The high bond strength between UHP-FRC and HSS (MMFX2) rebar was validated experimentally. Even with a short embedment length and small cover, UHP-FRC can provide sufficient bond to ensure the yielding of the embedded high strength rebar. The highly nonlinear behavior of HSS reinforcement also provide the ductility of the composite system even with a commonly brittle shear failure. Although all deck strip specimens (T-section beams) failed in shear, this specific shear failure is not abrupt and catastrophic. Shear cracks, additional beam deflection, and rotation at the cracked section were seen as signs prior to the ultimate shear failure. The ultimate structural response is largely dependent on the end anchorage of the longitudinal reinforcement or it's nonlinear stress-strain response if the end anchorage is strong enough. The fibers within the UHP-FRC can help prevent opening of the diagonal cracks besides the contribution from the end anchorage and transverse strengthening reinforcement. The supplemental shear stirrups help bridge the force across the shear cracks and changed the redistribution of the load after the forming of shear cracks. The bent rebar introduced an additional tension force transfer mechanism and thus caused a more gradual load redistribution even before the shear crack happened. Although the two shear strengthening methods did not increase the shear resistance, they improved the load transfer mechanism and can be treated as crack width control solutions. Based on the experiment results, the shear failure is regarded as an acceptable failure mode for the UHP-FRC/HSS beams if the longitudinal reinforcement is anchored using 180-degree hooks. This beam end anchorage was used in the system-level deck panel specimens that also exhibited similar shear failure mechanisms.


% For chapter 5
The system level finite element model created in this research was calibrated for four potential deck systems of moveable bridge, individually. Based on the extreme responses surface that obtained by placing unit load at all deck nodes, It was found that the Aluminum deck system has the highest stiffness and therefore lead to the smallest defection, moment, and shear in the grides. The girder responses is comparable when use open steel grid deck and the proposed UHP-FRC/HSS deck system. The girder largest girder deflection and moment happens when use the GFRP deck systems.


\section{FUTURE WORKS}

\subsection{Research on the advanced cement based composite material}
The goal is to fully characterize the mechanical properties of the material/structure under certain micro-structure conditions (fiber volume fraction, fiber orientation anisotropy, and curing method, etc). In order to achieve this, the finite element analysis procedure shall be able to reflect the microstructure condition influenced by cast flow direction and cement paste viscosity. From the experiment point of view, the cyclic, fatigue, and impact tests are interested. The conclusion out of this research will lead to an optimized design for the UHP-FRC structure, for example the blast resisting panel or a light weight concrete driven pile.

\subsection{Research on the advanced computational mechanics}
The uniaxial material model for UHP-FRC was already built in OpenSees and the multi-dimensional material model is the next target. While the plasticity based material model is usually used for this type of material, the low-tension-cracking concept is the one will be pursuit. This model can explicitly distinguish the crack direction and is easier to adopt the different post-crack response based on different lateral stress conditions. Along with the fluid flow simulations, the properties of the local orthotropic material can be related to the casting condition and curing methods based on the theory about fiber distribution and curing process while the global properties will then be obtained via the transformations.

\subsection{Research on modular bridge superstructure for quick construction}
The time for the construction or repair work is very sensitive parameter to achieve the economy target for all project. By using the modular bridge superstructure, the construction time can be greatly reduced. The lightweight precast column made by hollow UHP-FRC tube can be erect in the field and then post tensioned together with the foundation. After the installation of the precast concrete or steel girders, the precast deck can be placed piece by piece on these girders. Surely, the structural behavior of the individual component needs experimental examination and analytical simulations, on the other side, the joints between the connection parts are also important and sometimes critical. By utilizing the early strength of field casted UHP-FRC and the ultra-high bond strength between UHP-FRC and the steel reinforcement, both the dimensions and the construction time of the joints are expected to be greatly reduced.







%====================start export tables================
            \begin{table}
            \caption{Available UHP-FRC material}
            %\centering \includegraphics[scale=1]{table/chapter1/Table_available_UHPFRC.eps}
            \input{table/chapter1/Table_available_UHPFRC.tex}\\
            \label{Table_available_UHPFRC}
            \end{table}

\clearpage
\newpage
            \begin{table}
            \caption{Constitution list of typical UHP-FRC}
            %\centering \includegraphics[scale=1]{table/chapter1/Table_constitution.eps}
            \input{table/chapter1/Table_constitution.tex}
            \label{Table_constitution}
            \end{table}

\clearpage
\newpage
            \begin{table}
            \caption{Different scale levels of research works}
            %\centering \includegraphics[scale=1]{table/chapter1/Table_constitution.eps}
            \input{table/chapter1/Table_research_scales.tex}
            \label{Table_research_scales}
            \end{table}

\clearpage
\newpage
            \begin{table}
            \caption{Description and values of fiber properties}
            \centering
            %\includegraphics[scale=1]{table/chapter2/Table_fiber_para.eps}
            \input{table/chapter2/Table_fiber_para.tex}
            \label{Table_fiber_para}
            \end{table}

\clearpage
\newpage
            \begin{table}
            \caption{Explicit expression of fiber orientation PDFs}
            %\includegraphics[scale=0.7]{table/chapter2/Table_Transverse_ft.eps}
            \input{table/chapter2/Table_Transverse_ft.tex}
            \label{Table_Transverse_ft}
            \end{table}

\clearpage
\newpage
            \begin{table}
            \caption{Summary of curve fitting results: case WP}
            \centering
            %\includegraphics[scale=0.6]{table/chapter2/Table_ft_cf_case1.eps}
            \input{table/chapter2/Table_ft_cf_case1.tex}
            \label{Table_ft_cf_case1}
            \end{table}

\clearpage
\newpage
            \begin{table}
            \caption{Summary of curve fitting results: case SP}
            %\includegraphics[scale=0.6]{table/chapter2/Table_ft_cf_case2.eps}
            \input{table/chapter2/Table_ft_cf_case2.tex}
            \label{Table_ft_cf_case2}
            \end{table}

\clearpage
\newpage
            \begin{table}
            \caption{Parameters and typical values for UHP-FRC}
            \centering
            %\includegraphics[scale=1]{table/chapter2/Table_para_cox.eps}
            \input{table/chapter2/Table_para_cox.tex}
            \label{Table_para_cox}
            \end{table}

\clearpage
\newpage
                \begin{table}
                \caption{The post crack parameters used in the simplified stress-strain curve}
                %\centering \includegraphics[scale=1]{table/chapter3/para_french.eps}
                \input{table/chapter3/Table_para_french.tex}
                \label{para_french}
                \end{table}

\clearpage
\newpage
                \begin{table}
                \caption{Parameters and typical values for think model}
                %\centering \includegraphics[scale=0.9]{table/chapter3/para_MIT.eps}\
                \input{table/chapter3/Table_para_MIT.tex}
                \label{Table_para_MIT}
                \end{table}

\clearpage
\newpage
                \begin{table}
                \caption{Parameters used for size independent models}
                \centering \input{table/chapter3/Table_independent_para.tex}
                \label{Tabel_sizeindep_para}
                \end{table}

\clearpage
\newpage
                \begin{table}
                \caption{Maximum allowable shear stress}
                \input{table/chapter3/Table_ta_allow.tex}
                \label{Table_ta_allow}
                \end{table}

\clearpage
\newpage
                \begin{table}
                \caption{Results for unit width unreinforced section with different heights}
                %\centering \includegraphics[scale=0.7]{table/res_unreinforced.eps}
                \input{table/chapter3/Table_res_unreinforced.tex}
                \label{t_res_unreinforced}
                \end{table}

\clearpage
\newpage
                \begin{table}
                \caption{Results for unit width reinforced sections with normal strength rebar}
                %\centering \includegraphics[scale=0.6]{table/res_reinforced_EPP.eps}
                \input{table/chapter3/Table_res_reinforced_EPP.tex}
                \label{Table_res_reinforced_EPP}
                \end{table}

\clearpage
\newpage
                \begin{table}
                \caption{Results for unit width reinforced sections with high strength rebar}
                %\centering \includegraphics[scale=0.6]{table/res_reinforced_MMFX.eps}
                \input{table/chapter3/res_reinforced_MMFX.tex}
                \label{Table_res_reinforced_MMFX}
                \end{table}

\clearpage
\newpage
%            \begin{table}
%            \caption{Description and values of fiber properties}
%            \centering
%            \input{table/chapter4/table_highstrength_steel_select.tex}
%            \label{Table_HSS_select}
%            \end{table}

\clearpage
\newpage
            \begin{table}
            \caption{Deck specimen matrix}
            \centering
            \input{table/chapter4/table_deck_test_matrix.tex}
            \label{table_deck_test_matrix}
            \end{table}

\clearpage
\newpage
                \begin{table}
                \caption{Parameters used in the finite element analysis}
                \centering
                \input{table/chapter4/table_FEM_para.tex}
                \label{table_FEM_para}
                \end{table}

\clearpage
\newpage
            \begin{table}
            \caption{Matrix of experimental program}
            \input{table/chapter4/table_test_matrix.tex}
            \label{table_test_matrix}
            \end{table}

\clearpage
\newpage
            \begin{table}
            \caption{Bond length and cover for each test group}
            \input{table/chapter4/table_bond_cover.tex}
            \label{table_bond_cover}
            \end{table}

\clearpage
\newpage
            \begin{table}
            \caption{T-section beam configuration summary}
            \input{table/chapter4/table_Tsec_matrix.tex}
            \label{table_Tsec_matrix}
            \end{table}

\clearpage
\newpage
                \begin{table}
                \caption{Tested UHPC properties}
                \input{table/chapter4/table_UHPC_prop.tex}
                \label{table_UHPC_prop}
                \end{table}

\clearpage
\newpage
                \begin{table}
                \caption{Bond test results}
                \input{table/chapter4/table_bond_res.tex}
                
                \label{table_bond_res}
                \end{table}

\clearpage
\newpage
                \begin{table}
                \caption{Corresponding responses at the transition point and ultimate point}
                \input{table/chapter4/table_tsec_failtype.tex}
                \label{table_tsec_failtype}
                \end{table}

\clearpage
\newpage
            \begin{table}
            \input{table/chapter4/table_tsec_shearstrength.tex}
            \caption{Estimated shear strength using Eq.~\ref{eq_V_f}}
            \label{table_tsec_shearstrength}
            \end{table}

\clearpage
\newpage
            \begin{table}
            \caption{Shear resistance based on dowel action strut and tie model}
            \input{table/chapter4/table_tsec_estimation.tex}
            \label{table_tsec_estimation}
            \end{table}

\clearpage
\newpage
            \begin{table}
            \caption{Simulated results and calibration on steel open grid deck system}
            %\centering \includegraphics[scale=0.9]{table/chapter5/table_steel_grid_comp.eps}
            \input{table/chapter5/table_steel_grid_comp.tex}
            \label{table_steel_grid_comp}
            \end{table}

\clearpage
\newpage
            \begin{table}
            \caption{Simplified equation for load distribution factor}
            \input{table/chapter5/table_DF_AASHTO.tex}
            \label{table_DF_AASHTO}
            \end{table}

\clearpage
\newpage
                \begin{table}
                \caption{Calculated section properties of four deck units and steel girders}
                \input{table/chapter5/table_model_calib_section_property.tex}
                \label{table_model_calib_section_property}
                \end{table}

\clearpage
\newpage
                \begin{table}
                \caption{Calibration of parameters for the four deck systems}
                \input{table/chapter5/table_calib_all.tex}
                \label{table_calib_all}
                \end{table}

\clearpage
\newpage

%====================start export Figures================
            \begin{figure}
            \centering \includegraphics[scale=0.6]{image/chapter1/fig_flowtable.eps}
            \caption{Cement paste casting and flow table test}
            \label{fig_flowtable}
            \end{figure}

\clearpage
\newpage
            \begin{figure}
            \centering \includegraphics[scale=0.5]{image/chapter1/Fig_modeling_scale_work.eps}
            \caption{Modeling and experiments at different scale levels}
            \label{Fig_modeling_scale_work}
            \end{figure}

\clearpage
\newpage
            \begin{figure}
            \centering\includegraphics[scale=0.9]{image/chapter2/transversely_isotropic.eps}
            \caption{Local coordinates for two cases}
            \label{Fig_coord_transverse}
            \end{figure}

\clearpage
\newpage
            \begin{figure}
            \centering\includegraphics[scale=0.9]{image/chapter2/angle_coordinates.eps}
            \caption{Coordinates and angles}
            \label{fig_angle_coordinates}
            \end{figure}

\clearpage
\newpage
            \begin{figure}
            \centering\includegraphics[scale=0.75]{image/chapter2/orientation-factor.eps}
            \caption{Orientation factors for principle and transverse direction}
            \label{Fig_orient_factor}
            \end{figure}

\clearpage
\newpage
            \begin{figure}
            \centering\includegraphics[scale=0.5]{image/chapter2/fiber_place_case_all.eps}
            \caption{Fiber placement for the two anisotropic cased for $\kappa=0.2$}
            \label{fiber_place_case_all}
            \end{figure}

\clearpage
\newpage
            \begin{figure}
            \centering\includegraphics[scale=0.5]{image/chapter2/crossing_PDF_case_all.eps}
            \caption{PDF of orientations of fiber crossing $\kappa=0.2$}
            \label{fiber_crossing_PDF_case_all}
            \end{figure}

\clearpage
\newpage
            \begin{figure}
            \centering\includegraphics[scale=0.5]{image/chapter2/orient_factor.eps}
            \caption{Plot of orientation factors for two anisotropic cases}
            \label{orientation_factors}
            \end{figure}

\clearpage
\newpage
            \begin{figure}
            \centering\includegraphics[scale=0.5]{image/chapter2/simplify_formula.eps}
            \caption{Comparison of simulation results versus equations}
            \label{simplify_formula}
            \end{figure}

\clearpage
\newpage
            \begin{figure}
            \centering\includegraphics[scale=0.6]{image/chapter2/cf_k_01_ft.eps}
            \caption{Comparison of curve fitting and simulation results case SP}
            \label{Fig_cf_k_01}
            \end{figure}

\clearpage
\newpage
            \begin{figure}
            \centering\includegraphics[scale=0.6]{image/chapter2/fig_inner_distance.eps}
            \caption{Minimum inter-fiber distance with respect to the orientation scenarios}
            \label{fig_inner_distance}
            \end{figure}

\clearpage
\newpage
            \begin{figure}
            \centering\includegraphics[scale=1.5]{image/chapter2/fig_magnetic_prism.eps}
            \caption{Electro-Magnetic field treatment mold for prisms}
            \label{fig_magnetic_prism}
            \end{figure}

\clearpage
\newpage
            \begin{figure}
            \centering\includegraphics[scale=0.9]{image/chapter2/Fig_imageanalysis_proc.eps}
            \caption{Procedures for image analysis on spatial and directional distributions}
            \label{Fig_imageanalysis_proc}
            \end{figure}

\clearpage
\newpage
            \begin{figure}
            \centering\includegraphics[scale=0.6]{image/chapter2/Fig_image_uniformity_check.eps}
            \caption{Spatial distribution check: grid and results}
            \label{Fig_image_uniformity_check}
            \end{figure}

\clearpage
\newpage
            \begin{figure}
            \centering\includegraphics[scale=0.9]{image/chapter2/fig15.eps}
            \caption{Cutting planes for prisms, dimension shown in 'inch'}
            \label{fig15}
            \end{figure}

\clearpage
\newpage
            \begin{figure}
            \centering\includegraphics[scale=0.9]{image/chapter2/fig16.eps}
            \caption{Typical photo of unit area: a) cut 1-5 at [0,0], b) cut 6-8 at [0, $\pi$/2]}
            \label{fig16}
            \end{figure}

\clearpage
\newpage
            \begin{figure}
            \centering\includegraphics[scale=0.9]{image/chapter2/fig_exp_fiber_counts.eps}
            \caption{Fiber count for two flow cast prisms}
            \label{fig_exp_fiber_counts}
            \end{figure}

\clearpage
\newpage
            \begin{figure}
            \centering\includegraphics[scale=0.6]{image/chapter2/Fig_mechanical_res.eps}
            \caption{Cracks for the two types of prisms}
            \label{Fig_mechanical_res}
            \end{figure}

\clearpage
\newpage
            \begin{figure}
            \centering\includegraphics[scale=1.1]{image/chapter2/interfacial_shearstress.eps}
            \caption{Interfacial shear stress distribution}
            \label{interfacial_shear}
            \end{figure}

\clearpage
\newpage
        \begin{figure}
        \centering\includegraphics[scale=0.6]{image/chapter2/fig_modulus_elasticity.eps}
        \caption{Estimation of modulus of elasticity versus various anisotropic cases}
        \label{fig_modulus_elasticity}
        \end{figure}

\clearpage
\newpage
        \begin{figure}
        \centering\includegraphics[scale=0.6]{image/chapter2/fig_ultimate_eff.eps}
        \caption{Estimation of efficacy factor for ultimate tensile stress}
        \label{fig_ultimate_eff}
        \end{figure}

\clearpage
\newpage
                \begin{figure}
                \centering \includegraphics[scale=0.45]{image/chapter3/model_french.eps}
                \caption{Stress-strain curve based on crack opening}
                \label{model_french}
                \end{figure}

\clearpage
\newpage
                \begin{figure}
                \centering \includegraphics[scale=0.6]{image/chapter3/model_MIT.eps}
                \caption{Stress-strain curve from Think model}
                \label{Fig_model_MIT}
                \end{figure}

\clearpage
\newpage
                \begin{figure}
                \centering \includegraphics[scale=1.2]{image/chapter3/Fig_opensees_model.eps}
                \caption{Generalized uniaxial model built in OpenSees}
                \label{Fig_opensees_mode}
                \end{figure}

\clearpage
\newpage
                \begin{figure}
                \centering \includegraphics[scale=0.6]{image/chapter3/stress_strain_MMFX.eps}
                \caption{Uniaxial stress-strain of high strength steel rebars}
                \label{Fig_stress_strain_MMFX}
                \end{figure}

\clearpage
\newpage
                \begin{figure}
                \centering \includegraphics[scale=0.6]{image/chapter3/Fig_E_fc_relation.eps}
                \caption{Relation between modulus of elasticity and compressive strength}
                \label{Fig_E_Fc_relation}
                \end{figure}

\clearpage
\newpage
                \begin{figure}
                \centering \includegraphics[scale=0.5]{image/chapter3/Fig_exp_tensile.eps}
                \caption{Experimental results on post-crack responses }
                \label{Fig_exp_tensile}
                \end{figure}

\clearpage
\newpage
                \begin{figure}
                \centering \includegraphics[scale=1.2]{image/chapter3/Fig_French_Tension.eps}
                \caption{Tensile stress-strain relation for French model}
                \label{Fig_French_Tension}
                \end{figure}

\clearpage
\newpage
                \begin{figure}
                \centering \includegraphics[scale=0.6]{image/chapter3/Fig_model_independent.eps}
                \caption{Tensile stress-strain relation for size independent model (Treated)}
                \label{Fig_model_independent}
                \end{figure}

\clearpage
\newpage
                \begin{figure}
                \centering \includegraphics[scale=1.2]{image/chapter3/model_treated_untreated.eps}
                \caption{Stress-strain relation for treated and untreated material}
                \label{model_treated_untreated}
                \end{figure}

\clearpage
\newpage
                \begin{figure}
                \centering \includegraphics[scale=0.8]{image/chapter3/model_sheardesign_flow.eps}
                \caption{The flow chart to consider the shear deformation within flexural analysis}
                \label{model_sheardesign_flow}
                \end{figure}

\clearpage
\newpage
                \begin{figure}
                \centering \includegraphics[scale=0.6]{image/chapter3/Fig_ta_allowable.eps}
                \caption{The allowable shear stress based on the normal stress}
                \label{Fig_ta_allowable}
                \end{figure}

\clearpage
\newpage
                \begin{figure}
                \centering \includegraphics[scale=0.6]{image/chapter3/res_unreinforced.eps}
                \caption{Moment curvature relations a) h=50.8mm, b) h=1066.8 mm}
                \label{res_unreinforced}
                \end{figure}

\clearpage
\newpage
                \begin{figure}
                \centering \includegraphics[scale=0.6]{image/chapter3/Fig_h5_EPP_simple_all_0.01.eps}
                \caption{Moment curvature curves for section h=127 mm, EPP reinforced}
                \label{Fig_res_reinforced_EPP}
                \end{figure}

\clearpage
\newpage
                \begin{figure}
                \centering \includegraphics[scale=0.6]{image/chapter3/Fig_h5_MMFX_simple_all_0.01.eps}
                \caption{Moment curvature curves for section h=127 mm, HSS reinforced}
                \label{Fig_res_reinforced_MMFX}
                \end{figure}

\clearpage
\newpage
                \begin{figure}
                \centering \includegraphics[scale=0.6]{image/chapter3/res_unreinroced_lit.eps}
                \caption{Analytical results compare to experimental results, un-reinforced prism}
                \label{Fig_res_unreinforced_lit}
                \end{figure}

\clearpage
\newpage
                \begin{figure}
                \centering \includegraphics[scale=0.7]{image/chapter3/res_orient.eps}
                \caption{Comparison between analytical and experimental results: orientation impact}
                \label{Fig_res_orient}
                \end{figure}

\clearpage
\newpage
                \begin{figure}
                \centering \includegraphics[scale=0.8]{image/chapter3/res_reinforced_prism_setting.eps}
                \caption{Test setup of the small reinforced prisms}
                \label{res_reinforced_prism_setting}
                \end{figure}

\clearpage
\newpage
                \begin{figure}
                \centering \includegraphics[scale=0.65]{image/chapter3/specimen_UHPC_FRP.eps}
                \caption{Prisms with and without CFRP strengthening}
                \label{prism_frp}
                \end{figure}

\clearpage
\newpage
                \begin{figure}
                \centering \includegraphics[scale=0.6]{image/chapter3/res_reinforced_prism.eps}
                \caption{Experiment and analytical results for reinforced prisms}
                \label{res_reinforced_prism}
                \end{figure}

\clearpage
\newpage
                \begin{figure}
                \centering \includegraphics[scale=0.6]{image/chapter3/res_FRP_prism.eps}
                \caption{Experiment and analytical results for CFRP reinforced prisms}
                \label{res_FRP_prism}
                \end{figure}

\clearpage
\newpage
                \begin{figure}
                \centering \includegraphics[scale=1.0]{image/chapter3/Fig_Tsection_sec.eps}
                \caption{Section dimensions and reinforcement of two deck strips}
                \label{Fig_Tsection_sec}
                \end{figure}

\clearpage
\newpage
                \begin{figure}
                \centering \includegraphics[scale=0.6]{image/chapter3/Model_Tsection_pad.eps}
                \caption{FEM model of T section deck strip considering the pad redistribution}
                \label{Model_Tsec}
                \end{figure}

\clearpage
\newpage
                \begin{figure}
                \centering \includegraphics[scale=0.6]{image/chapter3/neoprene_pad_SI.eps}
                \caption{Pressure versus deformation curve for neoprene pad}
                \label{res_neoprene}
                \end{figure}

\clearpage
\newpage
                \begin{figure}
                \centering \includegraphics[scale=0.6]{image/chapter3/Fig_res_Tsec_pad.eps}
                \caption{Load versus displacement curves for three loading assumptions}
                \label{Fig_res_Tsec_pad}
                \end{figure}

\clearpage
\newpage
                \begin{figure}
                \centering \includegraphics[scale=0.55]{image/chapter3/Fig_res_Tsec_shear.eps}
                \caption{Load versus displacement curves considering the shear deformations}
                \label{Fig_res_Tsec_shear}
                \end{figure}

\clearpage
\newpage
            \begin{figure}
            \centering \includegraphics[scale=0.65]{image/chapter4/Fig_intro_MMFX.eps}
            \caption{The MMFX2 Rebar and its typical stress and strain curve}
            \label{Fig_intro_MMFX}
            \end{figure}

\clearpage
\newpage
                \begin{figure}
                \centering \includegraphics[scale=0.6]{image/chapter4/Fig_preliminary_section.eps}
                \caption{The configuration of the proposed deck system}
                \label{Fig_preliminary_section}
                \end{figure}

\clearpage
\newpage
                \begin{figure}
                \centering \includegraphics[scale=1]{image/chapter4/Fig_preliminary_sensitive.eps}
                \caption{Selection of the section dimensions}
                \label{Fig_preliminary_sensitive}
                \end{figure}

\clearpage
\newpage
                \begin{figure}
                \centering \includegraphics[scale=0.9]{image/chapter4/Fig_preliminary_finaldimension.eps}
                \caption{The deck dimensions of the final preliminary design}
                \label{Fig_preliminary_finaldimension}
                \end{figure}

\clearpage
\newpage
            \begin{figure}
            \centering \includegraphics[scale=0.18]{image/chapter4/Fig_Tsection_instrumentation.eps}
            \caption{The instrumentation plan for the specimens}
            \label{Fig_Tsection_instrumentation}
            \end{figure}

\clearpage
\newpage
            \begin{figure}
            \centering \includegraphics[scale=0.6]{image/chapter4/Fig_Tsection_loaddisp.eps}
            \caption{Load versus displacement curves for all specimens}
            \label{Fig_Tsection_loaddisp}
            \end{figure}

\clearpage
\newpage
            \begin{figure}
            \centering \includegraphics[scale=0.4]{image/chapter4/Fig_Tsection_failuremode.eps}
            \caption{Failure mode for all specimens}
            \label{Fig_Tsection_failuremode}
            \end{figure}

\clearpage
\newpage
                \begin{figure}
                \centering \includegraphics[scale=0.9]{image/chapter4/Fig_FEM_Model_wiredisplay.eps}
                \caption{FEM model of all four types of specimens}
                \label{Fig_FEM_Model_wiredisplay}
                \end{figure}

\clearpage
\newpage
                \begin{figure}
                \centering \includegraphics[scale=0.9]{image/chapter4/Fig_FEM_model_lowtension.eps}
                \caption{FEM model of low tension material \cite{Marc_Volume_A}}
                \label{Fig_FEM_low_tension}
                \end{figure}

\clearpage
\newpage
                \begin{figure}
                \centering \includegraphics[scale=0.6]{image/chapter4/Fig_FEM_res_1T1S.eps}
                \caption{FEM results on 1T1S specimen}
                \label{Fig_FEM_res_1T1S}
                \end{figure}

\clearpage
\newpage
                \begin{figure}
                \centering \includegraphics[scale=0.6]{image/chapter4/Fig_FEM_res_4T1S.eps}
                \caption{FEM results on 4T1S specimen}
                \label{Fig_FEM_res_4T1S}
                \end{figure}

\clearpage
\newpage
                \begin{figure}
                \centering \includegraphics[scale=0.6]{image/chapter4/Fig_FEM_res_1T2S.eps}
                \caption{FEM results on 1T2S specimen}
                \label{Fig_FEM_res_1T2S}
                \end{figure}

\clearpage
\newpage
                \begin{figure}
                \centering \includegraphics[scale=0.6]{image/chapter4/Fig_FEM_res_3T2S.eps}
                \caption{FEM results on 3T2S specimen}
                \label{Fig_FEM_res_3T2S}
                \end{figure}

\clearpage
\newpage
            \begin{figure}
            \centering\includegraphics[scale=0.9]{image/chapter4/Fig_compout_setup.eps}
            \caption{Specimen sketch of the compression bond test}
            \label{Fig_compout_setup}
            \end{figure}

\clearpage
\newpage
            \begin{figure}
            \centering\includegraphics[scale=0.9]{image/chapter4/Fig_flex_setup.eps}
            \caption{The small reinforced prism and loading configuration}
            \label{Fig_flex_setup}
            \end{figure}

\clearpage
\newpage
            \begin{figure}
            \centering\includegraphics[scale=0.9]{image/chapter4/Fig_Tsection_sec_2.eps}
            \caption{The T-section beam dimensions}
            \label{Fig_Tsection_sec_2}
            \end{figure}

\clearpage
\newpage
            \begin{figure}
            \centering\includegraphics[scale=0.9]{image/chapter4/Fig_Tsection_instru.eps}
            \caption{The T-section beam instrumentation plan}
            \label{Fig_Tsection_instru}
            \end{figure}

\clearpage
\newpage
            \begin{figure}
            \centering\includegraphics[scale=0.9]{image/chapter4/Fig_model_tooth.eps}
            \caption{'Tooth' model and load transfer through struts and ties\cite{27504}}
            \label{Fig_model_tooth}
            \end{figure}

\clearpage
\newpage
            \begin{figure}
            \centering\includegraphics[scale=0.9]{image/chapter4/Fig_model_struttie.eps}
            \caption{Strut and tie model for dowel action}
            \label{Fig_model_struttie}
            \end{figure}

\clearpage
\newpage
                \begin{figure}
                \centering\includegraphics[scale=0.9]{image/chapter4/Fig_comp_failmode.eps}
                \caption{The failure mode and weakening of the full section(view from rear)}
                \label{Fig_comp_failmode}
                \end{figure}

\clearpage
\newpage
                \begin{figure}
                \centering\includegraphics[scale=0.9]{image/chapter4/Fig_prism_resplot.eps}
                \caption{Load-strain and load-displacement results from the small-scale prisms}
                \label{Fig_prism_resplot}
                \end{figure}

\clearpage
\newpage
                \begin{figure}
                \centering\includegraphics[scale=0.9]{image/chapter4/Fig_prism_failure.eps}
                \caption{Failure modes for the two small-scale prisms}
                \label{Fig_prism_failure}
                \end{figure}

\clearpage
\newpage
                \begin{figure}
                \centering\includegraphics[scale=0.9]{image/chapter4/Fig_prism_failure_end.eps}
                \caption{Failure modes for the two small-scale prisms}
                \label{Fig_prism_failure_end}
                \end{figure}

\clearpage
\newpage
                \begin{figure}
                \centering\includegraphics[scale=0.9]{image/chapter4/Fig_prism_smallspan.eps}
                \caption{Test results of specimen with small shear span }
                \label{Fig_prism_smallspan}
                \end{figure}

\clearpage
\newpage
                \begin{figure}
                \centering\includegraphics[scale=0.9]{image/chapter4/Fig_tsec_resplot.eps}
                \caption{Test results of 1T1S specimen with US No 7 rebar}
                \label{Fig_tsec_resplot}
                \end{figure}

\clearpage
\newpage
                \begin{figure}
                \centering\includegraphics[scale=0.9]{image/chapter4/Fig_tsec_failmode.eps}
                \caption{Failure mode of the four 1T1S specimens}
                \label{Fig_tsec_failmode}
                \end{figure}

\clearpage
\newpage
                \begin{figure}
                \centering\includegraphics[scale=0.9]{image/chapter4/Fig_tsec_strainplot.eps}
                \caption{Test results of 1T1S specimens compared to shear strengthened specimens}
                \label{Fig_tsec_strainplot}
                \end{figure}

\clearpage
\newpage
                \begin{figure}
                \centering\includegraphics[scale=0.9]{image/chapter4/Fig_tsec_failmode_shear.eps}
                \caption{Failure mode of the two strengthened 1T1S specimens}
                \label{Fig_tsec_failmode_shear}
                \end{figure}

\clearpage
\newpage
            \begin{figure}
            \centering\includegraphics[scale=0.9]{image/chapter4/Fig_moment_shear.eps}
            \caption{The moment-shear interaction curves}
            \label{Fig_moment_shear}
            \end{figure}

\clearpage
\newpage
            \begin{figure}
            \centering\includegraphics[scale=0.3]{image/chapter4/Fig_FEM_interface_model.eps}
            \caption{The finite element model including interface}
            \label{Fig_FEM_interface_model}
            \end{figure}

\clearpage
\newpage
            \begin{figure}
            \centering\includegraphics[scale=0.6]{image/chapter4/Fig_FEM_inteface_res.eps}
            \caption{The results from finite element analysis considering the interface}
            \label{Fig_FEM_inteface_res}
            \end{figure}

\clearpage
\newpage
            \begin{figure}
            \centering \includegraphics[scale=0.4]{image/chapter5/fig_steel_grid_model.eps}
            \caption{Finite element model of typical unit of full-scale deck}
            \label{fig_steel_grid_model}
            \end{figure}

\clearpage
\newpage
            \begin{figure}
            \centering \includegraphics[scale=0.8]{image/chapter5/fig_FEM_model_res.eps}
            \caption{Finite element model results compared to experiment results}
            \label{fig_FEM_model_res}
            \end{figure}

\clearpage
\newpage
            \begin{figure}
            \centering \includegraphics[scale=0.6]{image/chapter5/Fig_opengrid_component.eps}
            \caption{3D models used for calibration purpose: steel open grid deck components}
            \label{Fig_opengrid_component}
            \end{figure}

\clearpage
\newpage
            \begin{figure}
            \centering \includegraphics[scale=0.9]{image/chapter5/Fig_deck_alum.eps}
            \caption{Aluminum deck under static loading}
            \label{Fig_deck_alum}
            \end{figure}

\clearpage
\newpage
            \begin{figure}
            \centering \includegraphics[scale=0.9]{image/chapter5/Fig_deck_FRP.eps}
            \caption{GFRP deck and assembling configuration}
            \label{Fig_deck_FRP}
            \end{figure}

\clearpage
\newpage
            \begin{figure}
            \centering \includegraphics[scale=0.5]{image/chapter5/Fig_system_typical_unit.eps}
            \caption{The basic unit of the system level model}
            \label{Fig_system_typical_unit}
            \end{figure}

\clearpage
\newpage
                \begin{figure}
                \centering \includegraphics[scale=1.2]{image/chapter5/Fig_model_calib_section.eps}
                \caption{The transverse (primary) sections of four deck systems}
                \label{Fig_model_calib_section}
                \end{figure}

\clearpage
\newpage
                \begin{figure}
                \centering \includegraphics[scale=0.75]{image/chapter5/Fig_model_calib_UHPC.eps}
                \caption{Model used in calibration  of UHP-FRC deck}
                \label{Fig_model_calib_UHPC}
                \end{figure}

\clearpage
\newpage
                \begin{figure}
                \centering \includegraphics[scale=0.6]{image/chapter5/Fig_model_calib_Alum.eps}
                \caption{Parameter calibration for Aluminum deck system}
                \label{Fig_model_calib_Alum}
                \end{figure}

\clearpage
\newpage
                \begin{figure}
                \centering \includegraphics[scale=0.5]{image/chapter5/Fig_model_calib_FRP.eps}
                \caption{Parameter calibration for GFRP deck system}
                \label{Fig_model_calib_FRP}
                \end{figure}

\clearpage
\newpage
                \begin{figure}
                \centering \includegraphics[scale=1.4]{image/chapter5/Fig_system_analysis_demo.eps}
                \caption{Demo of four type of bridge deck systems (1 in.=25.4 mm)}
                \label{Fig_system_analysis_demo}
                \end{figure}

\clearpage
\newpage
    `           \begin{figure}
                \centering \includegraphics[scale=0.4]{image/chapter5/Fig_sys_Surf_Dy.eps}
                \caption{Maximum girder vertical displacement surface}
                \label{Fig_sys_Surf_Dy}
                \end{figure}

\clearpage
\newpage
                \begin{figure}
                \centering \includegraphics[scale=0.4]{image/chapter5/Fig_sys_Surf_M.eps}
                \caption{Maximum girder moment surface}
                \label{Fig_sys_Surf_M}
                \end{figure}

\clearpage
\newpage
                \begin{figure}
                \centering \includegraphics[scale=0.4]{image/chapter5/Fig_sys_Surf_V.eps}
                \caption{Maximum girder shear force surface}
                \label{Fig_sys_Surf_V}
                \end{figure}

\clearpage
\newpage

%====================start export equations================
\begin{equation}\label{eq_uniform_ft}
f_{\Theta } (\theta ) = \sin (\theta)
\end{equation}

\begin{equation}\label{eq_general_ft_cross}
f_{\Theta }(\theta _{{\rm{crossing}}} ) = \frac{{\cos (\theta )f_{\Theta }\left({\theta } \right)}}{{\int\limits_0^{\pi /2} {\cos (\theta)f_{\Theta }\left({\theta } \right)d\theta } }}
\end{equation}

\begin{equation}\label{eq_uniform_ft_cross}
f_{\Theta }(\theta _{{\rm{crossing}}} )_{uniform}  = 2\sin (\theta)\cos (\theta)
\end{equation}

\begin{equation}\label{eq_Nv}
N_V  = 4V_f /(\pi d_f ^2 l_f)
\end{equation}

\begin{equation}\label{eq_Ns_alpha}
N_s  = \alpha _{orient} V_f /A_f=\alpha _{orient} N_V l_f
\end{equation}

\begin{equation}\label{eq_Ns_expression}
E(N_s ) = P(\theta\le \theta _{crit} |x = x_0)P( - \frac{{l_f }}{2} \le x_0 \le \frac{{l_f }}{2}) = \int\limits_{ - \frac{{l_f }}{2}}^{\frac{{l_f }}{2}} {\int\limits_0^{\theta _{crit} } {f_{\Theta }(\theta) d\theta dx} }
\end{equation}

\begin{equation}\label{eq_ss2}
\mathbf{C} = \left[ {\begin{array}{*{20}c}
   {\frac{1}{{E_t }}} & { - \frac{{v_t }}{{E_t }}} & { - \frac{{v_{pt} }}{{E_p }}} & 0 & 0 & 0  \\
   { - \frac{{v_t }}{{E_t }}} & {\frac{1}{{E_t }}} & { - \frac{{v_{pt} }}{{E_p }}} & 0 & 0 & 0  \\
   { - \frac{{v_{tp} }}{{E_t }}} & { - \frac{{v_{tp} }}{{E_t }}} & {\frac{1}{{E_p }}} & 0 & 0 & 0  \\
   0 & 0 & 0 & {\frac{1}{{2G_{pt} }}} & 0 & 0  \\
   0 & 0 & 0 & 0 & {\frac{1}{{2G_{pt} }}} & 0  \\
   0 & 0 & 0 & 0 & 0 & {\frac{{1 + v_t }}{{E_t }}}  \\
\end{array}} \right]
\end{equation}

\begin{equation}\label{eq_plane_stress_1}
\mathbf{C_{WP}} = \left[ {\begin{array}{*{20}c}
   {\frac{1}{{E_t }}} & { - \frac{{v_t }}{{E_t }}} & 0  \\
   { - \frac{{v_t }}{{E_t }}} & {\frac{1}{{E_t }}} & 0  \\
   0 & 0 & {\frac{{1 + v_t }}{{E_t }}}  \\
\end{array}} \right]
\end{equation}

\begin{equation}\label{eq_plane_stress_2}
\mathbf{C_{SP}} = \left[ {\begin{array}{*{20}c}
   {\frac{1}{{E_t }}} & { - \frac{{v_{pt} }}{{E_p }}} & 0  \\
   { - \frac{{v_{tp} }}{{E_t }}} & {\frac{1}{{E_p }}} & 0  \\
   0 & 0 & {\frac{1}{{2G_{pt} }}}  \\
\end{array}} \right]
\end{equation}

\begin{equation}\label{eq_s1s2}
\begin{array}{l}
 f(S_{WP}) = \beta (1 - s_{WP})^{\beta  - 1}  \\
 f(S_{SP}) = \beta (s_{SP})^{\beta  - 1}  \\
\end{array}
\end{equation}

\begin{equation}\label{eq_fp1_fp2}
\begin{array}{l}
\begin{array}{l}
\mathrm{WP:~}f_{\Theta _{p} } (\theta _{p} ) = \beta (1 - \cos (\theta _{p} ))^{\beta  - 1} \sin (\theta _{p} )  \\
\mathrm{SP:~}f_{\Theta _{p} } (\theta _{p} ) = \beta \cos ^{^{\beta  - 1} } (\theta _{p} )\sin (\theta _{p} ) \\
\end{array}
\end{array}
\end{equation}

\begin{equation}\label{eq_fp1_fp2_crossing}
\begin{array}{l}
\mathrm{WP:~}f_{\Theta _{p} ,{\rm{crossing}}} (\theta _{p} ) = c\sin (\theta _{p} )\cos (\theta _{p} )(1 - \cos (\theta _{p} ))^{\beta  - 1}  \\
\mathrm{SP:~}f_{\Theta _{p} ,{\rm{crossing}}} (\theta _{p} ) = (\beta  + 1)\sin (\theta _{p} )\cos (\theta _{p} )^\beta   \\
 \end{array}
\end{equation}

\begin{equation}\label{eq_angle_relation}
\left\{ \begin{array}{l}
 \cos (\Theta _3 ) = \cos (\Theta _1 )\cos (\Theta _2 ) \\
 \cos (\Theta _t ) = \sin (\Theta _p )\cos (\Theta _{tp} ) \\
 \tan (\Theta _1 ) = \tan (\Theta _p )\cos (\Theta _{tp} ) \\
 \Theta _3  = \Theta _p  \\
 \end{array} \right.
\end{equation}

\begin{equation}\label{eq_fii_angle}
\cos (\Theta _\varphi  ) = \cos (\Theta _p )\cos (\varphi ) + \sin (\Theta _p )\cos(\Theta_{tp})\sin (\varphi )
\end{equation}

\begin{equation}\label{eq_orient_principle}
\begin{array}{l}
\mathrm{WP:~}\alpha _{orient,p}  = \frac{\kappa }{{\kappa  + 1}} \\
\mathrm{SP:~}\alpha _{orient,p}  = \frac{1}{{\kappa  + 1}} \\
 \end{array}
\end{equation}

\begin{equation}\label{eq18}
\left\{ \begin{array}{l}
 l_i  = \cos (\theta _i )\cos (\theta _i ) \\
 m_i  = \cos (\theta _i )\sin (\theta _i ) \\
 n_i  = \sin (\theta _i ) \\
 \end{array} \right.
\end{equation}

\begin{equation}\label{eq19}
\left\{ \begin{array}{l}
 l_c x + m_c y + n_c z - (l_c  + m_c  + n_c ) = 0{\rm{\ldots\ldots cut plane}} \\
 \frac{{x - x_{0,i} }}{{l_i }} = \frac{{y - y_{0,i} }}{{m_i }} = \frac{{z - z_{0,i} }}{{n_i }}{\rm{\ldots\ldots\ldots\ldots\ldots\ldots\ldots fiber line}} \\
 \end{array} \right.
\end{equation}

\begin{equation}\label{eq20}
\left\{ \begin{array}{l} x_{{\mathop{\rm int}} ,i}  = \frac{{\left( {m_c m_i  + n_c n_i } \right)x_0  + l_i (m_c  + n_c  + l_c  - z_0 n_c  - y_0 m_c )}}{{m_c m_i  + n_c n_i  + l_c l_i }} \\ y_{{\mathop{\rm int}} ,i}  = \frac{{\left( {l_c l_i  + n_c n_i } \right)y_0  + m_i (m_c  + n_c  + l_c  - z_0 n_c  - x_0 l_c )}}{{m_c m_i  + n_c n_i  + l_c l_i }} \\ z_{{\mathop{\rm int}} ,i}  = \frac{{\left( {l_c l_i  + m_c m_i } \right)z_0  + n_i (m_c  + n_c  + l_c  - y_0 m_c  - x_0 l_c )}}{{m_c m_i  + n_c n_i  + l_c l_i }} \\ \end{array} \right.
\end{equation}

\begin{equation}\label{eq21}
\begin{array}{l}
 {\mathrm{Criterion~I:~}}\sqrt[{}]{{(x_{{\mathop{\rm int}} ,i}  - 1)^2  + (y_{{\mathop{\rm int}} ,i}  - 1)^2  + (z_{{\mathop{\rm int}} ,i}  - 1)^2 }} \le 0.5641 \\
 {\mathrm{Criterion~II:~}}\max ({\rm{distance}}(P_{{\mathop{\rm int}} ,i} ,P_{start,i} ),{\rm{distance}}(P_{{\mathop{\rm int}} ,i} ,P_{end,i} )) \le lf \\
 \end{array}
\end{equation}

\begin{equation}\label{eq22}
\left\{ \begin{array}{l}
 P_{start,i}  = P_{0,i}  - l_f [l_i ,m_i ,n_i ] \\
 P_{end,i}  = P_{0,i}  + l_f [l_i ,m_i ,n_i ] \\
 \end{array} \right.
\end{equation}

\begin{equation}\label{eq23}
\sin (\theta _i ) = \frac{{l_c l_i  + m_c m_i  + n_c n_i }}{{\sqrt[{}]{{(l_c ^2  + m_c ^2  + n_c ^2 )(l_i ^2  + m_i ^2  + n_i ^2 )}}}}
\end{equation}

\begin{equation}\label{eq24}
l_{{\rm{distance}}}  = l_c a + m_c b + n_c c - (l_c  + m_c  + n_c )
\end{equation}

\begin{equation}\label{eq_inter_distance}
l_{dist,i}  = \min (l_{{\rm{distance}}} (P_{start,i} ),l_{{\rm{distance}}} (P_{end,i} ))
\end{equation} 

\begin{equation}\label{eq25}
l_{em,i}  = l_{dist,i} /\cos (\theta _i )
\end{equation} 
\begin{equation}\label{eq_orient_kappa_case2}
\alpha _{orient,SP} (\kappa ,\varphi ) = \frac{1}{{1 + \kappa ^{1 - \varphi } }}{\rm{  }}
\end{equation}

\begin{equation}\label{eq_orient_kappa_case1}
\alpha _{orient,WP} (\kappa ,\varphi  = \frac{\pi }{2}) = \frac{2}{\pi }(1 - \kappa ) + \frac{1}{2}\kappa
\end{equation} 
\begin{equation}\label{eq_ft_fii}
f_{\Theta ,{\rm{crossing}}} (\theta ) = C_{sc} \sin(\theta )^{r_s} \cos (\theta )^{r_c}
\end{equation}

\begin{equation}\label{eq_lamda}
C_{sc}  = \frac{{2\Gamma (\frac{{r_s  + r_c  + 2}}{2})}}{{\Gamma (\frac{{r_s  + 1}}{2})\Gamma (\frac{{r_c  + 1}}{2})}}
\end{equation}

\begin{equation}\label{eq26}
2R = d_f \sqrt {\frac{\pi }{{V_f }}}
\end{equation}

\begin{equation}\label{eq_spatial_functions}
\left\{ \begin{array}{l}
 F(r): N(g\in A_r|{D_{min}(g_0,f\in A)<r})/N(g \in A_r)\\
 G(r): N(f \in A_r|{D_{min}(f_0,f\in A)<r)}/N(f \in A_r)\\
 K(r): A_r*\sum {N(f \in A_r|{D_{min}(f_0,f\in A)<r}})/{N(f \in A_r)}^2 \\
 \end{array} \right.
\end{equation}

\begin{equation}\label{eq_E_effective}
\overline {E_f } = E_f \eta _l
\end{equation}

\begin{equation}\label{eq_eta_l}
\eta _l  = (1 - \frac{{{\rm{tanh}}(\lambda l_f /2)}}{{\lambda l_f /2}})
\end{equation}

\begin{equation}\label{eq_lambda}
\lambda  = \left[ {\frac{{2G_m }}{{E_f r_f ^2 \ln (R/r_f )}}} \right]^{1/2}
\end{equation}

\begin{equation}\label{eq_incline_reduction}
\eta_{\theta_{i}} = \cos ^{f_d} (\theta_{i})
\end{equation}

\begin{equation}\label{eq29}
\begin{array}{l}
 \eta _{\theta ,i}  =  - \frac{{12\theta _i ^2 }}{{\pi ^2 }} + \frac{{4\theta _i }}{\pi } + 1 \\
 \eta _{l,i}  = 1 \\
 \end{array}
\end{equation}  
\begin{equation}\label{eq_composite_modulus}
E_c  = E_m (1 - N_s A_f) + E_f A_f\eta _l\sum\limits_{i = 1..Ns} {\eta_{\theta_{i}} }
\end{equation}

\begin{equation}\label{eq_E_expectation}
E(E_c ) = E_m (1 - \alpha _{orient} V_f ) + \eta _l \eta _\theta  E_f V_f \alpha _{orient}
\end{equation}

\begin{equation}\label{eq_eta_theta}
\eta _\theta   = \frac{{\Gamma (\frac{{r_s }}{2})\Gamma (\frac{{r_c }}{2} + 1)}}{{f_d \Gamma (\frac{{r_c  + 1}}{2})\Gamma (\frac{{r_s  + 1}}{2})}}
\end{equation}

\begin{equation}\label{eq27}
\sigma _{t,ult}  = \eta_g \sum\limits_{i = 1}^{N_s } {P_{ult} (z_i ,\theta _i )}  = \eta_g \pi d_f \sum\limits_{i = 1}^{N_s } {\left[ {\tau (l_{em,i} ,\theta _{crossing,i} )l_{em,i} } \right]}
\end{equation}

\begin{equation}\label{eq28}
\sum\limits_{i = 1}^{N_s } {\left[ {\tau (l_{em,i} ,\theta _{crossing,i} )l_{em,i} } \right]}  = \sum\limits_{i = 1}^{N_s } {\left[ {\eta _{\theta ,i} \eta _{l,i} \tau _{fu} l_{em,i} } \right]}
\end{equation}

\begin{equation}\label{eq30}
E\left[\sum\limits_{i = 1}^{N_s } {\left[ {\eta _{\theta ,i} \eta _{l,i} \tau _{fu} l_{em,i} } \right]}\right]=\tau _{fu}\frac{{l_f }}{4}
\frac{{V_f }}{A_f}\alpha _{orient}E\left[- \frac{{12\theta _i ^2 }}{{\pi ^2 }} + \frac{{4\theta _i }}{\pi } + 1  \right]
\end{equation}

\begin{equation}\label{eq31}
E[ {\sigma _{t,ult} } ] = \eta _{eff} \eta _{g} \tau _{fu} \frac{{l_f }}{{d_f }}V_f
\end{equation}

\begin{equation}\label{eq_eff}
\eta _{eff}  = \alpha _{orient} ( - \frac{{12 ( E^2[\theta] + Var[\theta] ) }}{{\pi ^2 }} + \frac{{4 E[\theta] }}{\pi } + 1)
\end{equation}

\begin{equation}\label{eq32}
E[\theta]  = \int\limits_0^{\pi /2} {\theta~C_{sc} \cos ^{r_c } (\theta )\sin ^{r_s } (\theta )} d\theta
\end{equation}

\begin{equation}\label{Var_t}
Var[\theta] = \int\limits_0^{\pi /2} {(\theta  - E[\theta] )^2 }{C_{sc} \cos ^{r_c } (\theta )\sin ^{r_s } (\theta )} d\theta
\end{equation}

\begin{equation}\label{Model_para_MIT}
\left\{ \begin{array}{l}
K_1  = C_M  + C_F,K_2  = C_F  + \frac{{C_M M}}{{M + C_M }}  \\
\sigma _1^ -   = (1 + \frac{{C_F }}{{C_M }})(f_t  + k_M )  \\
\sigma _1^ +   = (1 + \frac{{C_F }}{{C_M }})(f_t  + k_M ) - \frac{{C_M }}{{C_M  + M}}f_t  \\
\sigma _2  = k_M  + f_y  \\
 \end{array} \right.
\end{equation}

\begin{equation}\label{para_comp}
\begin{array}{l}
 \sigma _{c0}  = f_c ^\prime   \\
 \varepsilon _{c0}  = f_c ^\prime  /E_c  \\
 \sigma _{c1}  = f_c ^\prime   \\
 \varepsilon _{c1}  =  - 0.004 \\
 \varepsilon _{c2}  =  - 0.01 \\
 \end{array}
\end{equation}

\begin{equation}\label{s_w_relation}
\sigma  = \frac{1}{\gamma }(6.9\ln (680w + 1) - 8.8w)
\end{equation}

\begin{equation}\label{para_French}
\begin{array}{l}
 \sigma _{t0}  = f_t  \\
 \varepsilon _{t0}  = f_t /E_t  \\
 \sigma _{t1}  = \sigma (w_{0.3} ) \\
 \varepsilon _{t1}  = \varepsilon _{t0}  + w_{0.3} /l_c  \\
 \sigma _{t2}  = \sigma (w_{1\% } ) \\
 \varepsilon _{t2}  = \varepsilon _{t0}  + w_{1\% } /l_c  \\
 \varepsilon _{t3}  = \varepsilon _{t0}  + w_{\lim } /l_c  \\
 \end{array}
\end{equation}

\begin{equation}\label{eq_sig_1}
\sigma _{0}(\varphi,\kappa )  = \eta _1 \sigma _{0}
\end{equation}

\begin{equation}\label{eq1_sig_2}
\sigma _{1}(\varphi,\kappa )  = f_{t,m}  + \eta _2 f_{t,f}
\end{equation}

\begin{equation}\label{eq_eta_1}
\eta _1  = 2^{1-\kappa}  [1 - \sin (\varphi )\sqrt {1-\kappa}  ]
\end{equation}

\begin{equation}\label{eq_eta_2}
\eta _2  = 2^{1-\kappa}  [1 - \sin (\varphi )^4 \sqrt {1-\kappa}  ]
\end{equation}

\begin{equation}
\label{eq_orient_kappa_case2_2}
\alpha (\kappa ,\varphi ) = \frac{1}{{1 + \kappa ^{1 - \varphi } }}{\rm{  }}
\end{equation}

\begin{equation}\label{eq_ta_allow}
\tau _{\max }  = \sqrt {f_{t,\max } (f_{t,\max }  - \sigma _N )}
\end{equation}

\begin{equation}\label{eq_G_compre}
G_c = G_0 (1 - \frac{{\tau _{\max } }}{{\tau _{\max ,f'_c } }})
\end{equation}

%\begin{equation}\label{eq_stainless_ss}
%\varepsilon  = \left\{ \begin{array}{l}
% \frac{\sigma }{{E_0 }} + 0.002\left( {\frac{\sigma }{{\sigma _{0.2} }}} \right)^n {\rm{                         for }}\sigma  \le \sigma _{0.2}  \\
% \frac{{\sigma  - \sigma _{0.2} }}{{E_{0.2} }} + \varepsilon _u \left( {\frac{{\sigma  - \sigma _{0.2} }}{{\sigma _u  - \sigma _{0.2} }}} \right)^m  + \varepsilon _{0.2} {\rm{      for }}\sigma  > \sigma _{0.2} {\rm{     }} \\
% \end{array} \right.
%\end{equation}

\begin{equation}\label{eq_V_rb}
V_{Rb}  = \frac{1}{{\gamma _E }}\frac{{0.21}}{{\gamma _b }}k\sqrt {f_c } b_0 d
\end{equation}

\begin{equation}\label{eq_V_f}
V_f  = \frac{{S\sigma _P }}{{\gamma _{bf} \tan (\beta _u )}}
\end{equation}

\begin{equation}\label{eq_V_u}
V_u  = \tau _u \pi d_s z
\end{equation} 
\begin{equation}\label{eq_V_c}
V_c  = \frac{2}{3}cb_w v_n  = \frac{2}{3}c\tau \pi d_s  = \frac{2}{3}\frac{c}{z}V
\end{equation}

\begin{equation}\label{eq_dowel}
\left\{ \begin{array}{l}
 V = F_{c2}  + F_c \sin (\theta _c ) \\
 F_{c2}  = F_s \sin (\theta _d ) \\
 F_c \cos (\theta _c ) - F_s \cos (\theta _d ) = N_L  \\
 \end{array} \right.
\end{equation}

\begin{equation}\label{eq_geometry_eq1}
\tan (\theta _c ) + \tan (\theta _{_d } ) = d/L1
\end{equation}

\begin{equation}\label{eq_geometry_eq2}
V_u  = F_{s,u} \cos (\theta _d )\frac{d}{{L1}}
\end{equation}

\begin{equation}\label{eq_V_general}
\frac{{V_u }}{{bd\sqrt {f_c ^\prime  } }} = \frac{1}{{\alpha  + \beta (0.6d - c)\phi d}}
\end{equation}

\begin{equation}\label{eq_phi}
\phi  = \frac{M}{{A_s E_s (d - c/3)(d - c)}}
\end{equation}

\begin{equation}\label{eq_c}
c = \frac{{A_s E_s }}{{E_c b}}(\sqrt {1 + \frac{{2E_c bd}}{{E_s A_s }}}  - 1)
\end{equation}

\begin{equation}\label{eq_Vu_new}
V_u  = \frac{1}{\alpha }bd\sqrt {f_c ^\prime  }
\end{equation} 
\end{document}